\begin{abstract}
Knowledge graphs offer a structured representation of real-world entities and their relationships, enabling a wide range of applications from information retrieval to automated reasoning. In this paper, we conduct a systematic comparison between traditional rule-based approaches and modern deep learning methods for link prediction. We focus on KBGAT, a graph neural network model that leverages multi-head attention to jointly encode both entity and relation features within local neighborhood structures. To advance this line of research, we introduce \textbf{GCAT} (Graph Collaborative Attention Network), a refined model that enhances context aggregation and interaction between heterogeneous nodes. Experimental results on four widely-used benchmark datasets demonstrate that GCAT not only consistently outperforms rule-based methods but also achieves competitive or superior performance compared to existing neural embedding models. Our findings highlight the advantages of attention-based architectures in capturing complex relational patterns for knowledge graph completion tasks.
\end{abstract}
