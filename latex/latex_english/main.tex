\documentclass[12pt,a4paper]{extreport} % Lớp tài liệu: báo cáo mở rộng, khổ giấy A4, cỡ chữ 12pt
\usepackage[utf8]{inputenc} % Mã hóa đầu vào UTF-8
%\usepackage[T5]{fontenc} % Mã hóa font T5 (hỗ trợ tiếng Việt - thường dùng khi không dùng gói vietnam)
%\usepackage[utf8]{vietnam} % Gói hỗ trợ tiếng Việt với mã UTF-8
%\usepackage{times} % Font Times New Roman
\usepackage{enumerate} % Hỗ trợ liệt kê theo số thứ tự
\usepackage{enumitem} % Tùy chỉnh danh sách liệt kê
\usepackage{multicol} % Gói hỗ trợ chia nhiều cột
\usepackage{listings} % Gói hiển thị mã nguồn
\usepackage[a4paper, top=3.5cm,bottom=3cm,left=3.5cm,right=2cm]{geometry} % Cài đặt lề trang
\usepackage{verbatim} % Cho phép chèn đoạn văn bản giữ nguyên định dạng
\usepackage{graphicx} % Hỗ trợ chèn hình ảnh
\usepackage{url} % Hiển thị URL
\usepackage{fancyhdr} % Tùy chỉnh header và footer
\usepackage{fancybox,framed} % Tạo khung hộp cho nội dung
\linespread{1.3} % Giãn dòng 1.3
\usepackage{lastpage} % Để biết số trang cuối cùng
\usepackage{floatrow} % Gói hỗ trợ bố trí hình và bảng
\usepackage{array} % Gói hỗ trợ bảng nâng cao
\usepackage{longtable} % Hỗ trợ bảng kéo dài nhiều trang

\pagenumbering{arabic} % Đánh số trang bằng số tự nhiên
%\pagestyle{fancy} % (Đang tắt) Kiểu trang fancy
\newfloatcommand{capbtabbox}{table}[][\FBwidth] % Tạo môi trường bảng đặc biệt với floatrow
\usepackage{caption} % Tùy chỉnh caption của hình và bảng
%\captionsetup[figure]{font=normalsize} % (Đang tắt) Thiết lập cỡ chữ cho caption hình
\usepackage{blindtext} % Tạo văn bản mẫu (dummy text)
\usepackage{titlesec} % Tùy chỉnh tiêu đề chương, mục
\usepackage[nottoc]{tocbibind} % Cho phép hiện danh mục tài liệu tham khảo trong mục lục nhưng không hiện mục lục chính
\usepackage{bookmark} % Quản lý bookmark PDF
%\usepackage[backend=bibtex,style=numeric,defernumbers=true,sorting=none]{biblatex} % (Đang tắt) Cấu hình trích dẫn BibLaTeX
% \usepackage[backend=biber]{biblatex} % (Đang tắt) Cấu hình dùng biber cho biblatex
%\bibliographystyle{plain} % (Đang tắt) Kiểu hiển thị tài liệu tham khảo
%\bibliography{References/references.bib} % (Đang tắt) Nguồn tài liệu tham khảo

% ******************************************************************************
%\setlength{\topmargin}{-0.5cm} % (Đang tắt) Lề trên
%\setlength{\oddsidemargin}{-0.5cm} % (Đang tắt) Lề trái
%\setlength{\evensidemargin}{-0.5cm} % (Đang tắt) Lề phải
\setlength{\textwidth}{15.5cm} % Chiều rộng vùng văn bản
\setlength{\textheight}{24cm} % Chiều cao vùng văn bản
\setlength{\footskip}{1cm} % Khoảng cách từ nội dung đến footer (số trang)

% Giãn dòng 1.5 (tùy chọn)
%\usepackage{setspace}
%\renewcommand{\baselinestretch}{1.5}

% ******************************************************************************
\usepackage{hyperref} % Hỗ trợ liên kết siêu văn bản (hyperlink)

% Cấu hình cho \paragraph để nó xuống dòng sau tiêu đề
\titleformat{\paragraph}[block]
{\normalfont\normalsize\bfseries}{\theparagraph}{1em}{}

\titlespacing*{\paragraph}
{0pt}{3.25ex plus 1ex minus .2ex}{1.5ex plus .2ex}

\setcounter{secnumdepth}{4} % Đánh số đến cấp độ subsubsubsection
\setcounter{tocdepth}{4} % Hiển thị mục lục đến cấp độ subsubsubsection

\usepackage{array} % Gói cần thiết cho cột m{} trong bảng

% Định dạng tiêu đề các cấp
\titleformat*{\section}{\LARGE\bfseries}
\titleformat*{\subsection}{\Large\bfseries}
\titleformat*{\subsubsection}{\large\bfseries}

\usepackage{mdframed} % Hộp khung có thể đổ bóng

\usepackage{alltt} % Hiển thị văn bản định dạng có thể lồng lệnh LaTeX
\usepackage{color} % Hỗ trợ màu sắc
\usepackage{tocloft} % Tùy chỉnh mục lục

\usepackage{tikz} % Gói vẽ hình
\usetikzlibrary{calc} % Thư viện con hỗ trợ tính toán vị trí vẽ trong TikZ
\newcommand\HRule{\rule{\textwidth}{1pt}} % Định nghĩa dòng kẻ ngang
\fancyfoot[C]{\thepage} % Hiển thị số trang ở giữa footer

% Toán học
\usepackage{amsfonts} % Gói font toán học (ví dụ: \mathbb)

\usepackage[numbers]{natbib} % Gói trích dẫn tài liệu theo số (vd: [1], [2])

% Chèn thêm các tệp cấu hình:
\newcommand{\tenSV}{Hoàng~Minh~Thanh~-~Phan~Minh~Tâm} % Dấu ~ là khoảng trắng không được tách (các chữ nối với nhau bằng dấu ~ sẽ nằm cùng 1 dòng
\newcommand{\mssv}{18424062-18424059}
\newcommand{\tenKL}{ĐỰ~ĐOÁN~LIÊN~KẾT~TRONG~ĐỒ~THỊ~TRI~THỨC} % Chú ý dấu ~ trong tên khóa luận
\newcommand{\tenGVHD}{ThS.~Lê~Ngọc~Thành}
\newcommand{\tenBM}{BM. Khoa~Học~Máy~Tính}
 % Metadata: tên tác giả, luận văn,...
\usepackage{amsmath}
\usepackage{amssymb}
\usepackage{stmaryrd}

\usepackage[font=small,labelfont=bf]{subcaption}
\usepackage{listings}
\usepackage{color}

\usepackage{tocloft}
\usepackage{tabularx}
\usepackage{booktabs}
\usepackage{multirow}

\usepackage{tocloft}
\usepackage{tabularx}
\usepackage{booktabs}


% Chèn và định dạng mã giả
\usepackage{algorithm}
% \usepackage[ruled,vlined,linesnumbered]{algorithm2e}
\usepackage[noend]{algpseudocode}
\makeatletter
\def\BState{\State\hskip-\ALG@thistlm}
\makeatother


\usepackage{float}



\usepackage{cleveref}


% Vẽ hình
\usetikzlibrary{arrows,shapes,positioning, decorations, babel, quotes}
\usepackage[babel,german=quotes]{csquotes}

% Coordinate
\usepackage{pgfplots, pgfplotstable}

\usepackage{ltablex} % Gói mở rộng tùy chọn bổ sung (nếu có)
% Redefine \tiny
\renewcommand{\tiny}{\fontsize{6.01}{7.21}\selectfont}    % MS Word 6pt ≈ LaTeX 6.01pt

% Redefine \scriptsize
\renewcommand{\scriptsize}{\fontsize{7.01}{8.41}\selectfont}  % MS Word 7pt ≈ LaTeX 7.01pt

% Redefine \footnotesize
\renewcommand{\footnotesize}{\fontsize{8.03}{9.63}\selectfont} % MS Word 8pt ≈ LaTeX 8.03pt

% Redefine \small
\renewcommand{\small}{\fontsize{12.03}{14.43}\selectfont}  % MS Word 12pt ≈ LaTeX 12.03pt

% Redefine \normalsize
\renewcommand{\normalsize}{\fontsize{13.04}{15.65}\selectfont} % MS Word 13pt ≈ LaTeX 13.04pt

% Redefine \large
\renewcommand{\large}{\fontsize{14.08}{16.90}\selectfont}  % MS Word 14pt ≈ LaTeX 14.08pt

% Redefine \Large
\renewcommand{\Large}{\fontsize{16.16}{19.26}\selectfont}  % MS Word 16pt ≈ LaTeX 16.16pt

% Redefine \LARGE
\renewcommand{\LARGE}{\fontsize{18.19}{21.83}\selectfont}  % MS Word 18pt ≈ LaTeX 18.19pt

% Redefine \huge
\renewcommand{\huge}{\fontsize{20.22}{24.26}\selectfont}   % MS Word 20pt ≈ LaTeX 20.22pt

% Redefine \Huge
\renewcommand{\Huge}{\fontsize{24.27}{29.12}\selectfont}   % MS Word 24pt ≈ LaTeX 24.27pt




% \titleformat{\section}[block]{\normalfont\Large\bfseries}{\thesection}{1em}{}

% \titleformat{\chapter}[display]
%   {\normalfont\LARGE\bfseries}   % Font size and weight for chapter title
%   {\thechapter}                  % Chapter number formatting
%   {1em}                          % Space between number and title
%   {}

% \makeatletter
% \def\@makechapterhead#1{
%   \vspace*{50pt}  % Vertical space before the chapter title
%   {\normalfont\Huge\bfseries\MakeUppercase{#1}}  % Font size and uppercase
%   \par\nobreak
%   \vskip 40pt
% }
% \makeatother


% \makeatletter
% \def\@makechapterhead#1{
%   \vspace*{50pt}  % Vertical space before the chapter title
%   {\normalfont\Huge\bfseries \MakeUppercase{Chapter \thechapter: #1}}  % "Chapter" text and uppercase title
%   \par\nobreak
%   \vskip 40pt
% }
% \makeatother

%********************* Custom *********************

% \Huge
% Customize chapter title (equivalent to Word styles)
\titleformat{\chapter} % Adjust the chapter heading
  {\LARGE\bfseries}     % Font size: Huge (24.27pt), bold
  {\chaptername\ \MakeUppercase\thechapter.}       % Chapter number with a period
  {20pt}               % Space between the number and title
  {\MakeUppercase}     % Title format

% Customize section title (equivalent to Word styles)
\titleformat{\section} % Adjust the section heading
  {\Large\bfseries}    % Font size: LARGE (18.19pt), bold
  {\thesection}        % Section numbering
  {15pt}               % Space between the number and title
  {\LARGE\bfseries}    % Title format

% Customize subsection title
\titleformat{\subsection} % Adjust the subsection heading
  {\large\bfseries}      % Font size: Large (16.16pt), bold
  {\thesubsection}       % Subsection numbering
  {10pt}                 % Space between the number and title
  {\Large\bfseries}      % Title format

% Customize subsubsection title
\titleformat{\subsubsection} % Adjust the subsubsection heading
{\normalsize\bfseries}      % Font size: Large (16.16pt), bold
{\thesubsubsection}       % Subsection numbering
{10pt}                 % Space between the number and title
{\normalsize\bfseries}      % Title format

%********************* Custom *********************

\definecolor{customlinkcolor}{HTML}{0064E0}
\definecolor{customcitecolor}{HTML}{00AF19}
\definecolor{custompinkcolor}{HTML}{AF00E6}


% custom color
\definecolor{awesome}{rgb}{1.0, 0.13, 0.32}
\definecolor{azure}{rgb}{0.0, 0.5, 1.0}
\definecolor{amber}{rgb}{1.0, 0.75, 0.0}
\definecolor{deepmagenta}{rgb}{0.8, 0.0, 0.8}
\definecolor{darkpastelgreen}{rgb}{0.01, 0.75, 0.24}
\definecolor{deepskyblue}{rgb}{0.0, 0.75, 1.0}
\definecolor{deepcarminepink}{rgb}{0.94, 0.19, 0.22}
\definecolor{gray}{rgb}{0.5, 0.5, 0.5}


\definecolor{codegreen}{rgb}{0,0.6,0}
\definecolor{codegray}{rgb}{0.5,0.5,0.5}
\definecolor{codepurple}{rgb}{0.58,0,0.82}
\definecolor{backcolour}{rgb}{0.95,0.95,0.92}
\lstdefinestyle{mystyle}{
    backgroundcolor=\color{backcolour},   
    commentstyle=\color{codegreen},
    keywordstyle=\color{magenta},
    numberstyle=\tiny\color{codegray},
    stringstyle=\color{codepurple},
    basicstyle=\footnotesize,
    breakatwhitespace=false,         
    breaklines=true,                 
    captionpos=b,                    
    keepspaces=true,                 
    numbers=left,                    
    numbersep=5pt,                  
    showspaces=false,                
    showstringspaces=false,
    showtabs=false,                  
    tabsize=2
}
\lstset{style=mystyle}


\hypersetup{
	colorlinks=true,
	linkcolor=customlinkcolor,
	citecolor=customcitecolor,
	filecolor=magenta,
	urlcolor=custompinkcolor,
	citebordercolor=red,
	breaklinks=true,
	bookmarksopen=true,
	frenchlinks=true,
	linkbordercolor={0 0 1},
	menubordercolor={0 0 1},
	plainpages=false,
	urlbordercolor={0 0 1},
	pdftitle={Link Prediction In Knowledge Graphs Using Graph Collaborative Attention Network},
	pdfauthor={Phan Minh Tam - Hoang Minh Thanh},
	pdfpagemode=FullScreen,
	pdfborder={0 0 0}
}


% Abstract
\usepackage{changepage}
 % Cấu hình riêng (tiêu đề, định dạng trang,...)
\newcommand{\der}[2]{\frac{d#1}{d#2}}
\newcommand{\nder}[3]{\frac{d^#1 #2}{d #3 ^ #1}}
\newcommand{\pder}[2]{\frac{\partial #1}{\partial #2}}
\newcommand{\npder}[3]{\frac{\partial ^#1 #2}{\partial #3^#1}}
\newcommand{\sentencelist}{def}
\newcommand{\overbar}[1]{\mkern 1.5mu\overline{\mkern-1.5mu#1\mkern-1.5mu}\mkern 1.5mu}
\newcommand{\lined}{\overbar}
\newcommand{\perm}[2]{{}^{#1}\!P_{#2}}
\newcommand{\comb}[2]{{}^{#1}C_{#2}}
\newcommand{\intall}{\int_{-\infty}^{\infty}}
\newcommand{\Var}[1]{\text{Var}\left(#1\right)}
\newcommand{\E}[1]{\text{E}\left(#1\right)}
\newcommand{\define}{\equiv}
\newcommand{\diff}[1]{\mathrm{d}#1}
\newcommand{\empy}[1]{{\color{darkorange}\emph{#1}}}
\newcommand{\empr}[1]{{\color{cardinalred}\emph{#1}}}

%********************* Custom ********************* 
\newcommand{\vardbtilde}[1]{\tilde{\raisebox{0pt}[0.85\height]{$\tilde{#1}$}}}
\newcommand{\defeq}{\coloneqq}
\newcommand{\grad}{\nabla}
%\newcommand{\E}{\mathbb{E}}
%\newcommand{\Var}{\mathrm{Var}}
\newcommand{\Cov}{\mathrm{Cov}}
\newcommand{\Ea}[1]{\E\left[#1\right]}
\newcommand{\Eb}[2]{\E_{#1}\!\left[#2\right]}
\newcommand{\Vara}[1]{\Var\left[#1\right]}
\newcommand{\Varb}[2]{\Var_{#1}\left[#2\right]}
\newcommand{\kl}[2]{D_{\mathrm{KL}}\!\left(#1 ~ \| ~ #2\right)}
\newcommand{\pdata}{{p_\mathrm{data}}}
\newcommand{\bA}{\mathbf{A}}
\newcommand{\bI}{\mathbf{I}}
\newcommand{\bJ}{\mathbf{J}}
\newcommand{\bH}{\mathbf{H}}
\newcommand{\bL}{\mathbf{L}}
\newcommand{\bM}{\mathbf{M}}
\newcommand{\bQ}{\mathbf{Q}}
\newcommand{\bR}{\mathbf{R}}
\newcommand{\bzero}{\mathbf{0}}
\newcommand{\bone}{\mathbf{1}}
\newcommand{\bb}{\mathbf{b}}
\newcommand{\bu}{\mathbf{u}}
\newcommand{\bv}{\mathbf{v}}
\newcommand{\bw}{\mathbf{w}}
\newcommand{\bx}{\mathbf{x}}
\newcommand{\by}{\mathbf{y}}
\newcommand{\bz}{\mathbf{z}}
\newcommand{\bxh}{\hat{\mathbf{x}}}
\newcommand{\btheta}{{\boldsymbol{\theta}}}
\newcommand{\bphi}{{\boldsymbol{\phi}}}
\newcommand{\bepsilon}{{\boldsymbol{\epsilon}}}
\newcommand{\bmu}{{\boldsymbol{\mu}}}
\newcommand{\bnu}{{\boldsymbol{\nu}}}
\newcommand{\bSigma}{{\boldsymbol{\Sigma}}}


% Định nghĩa
\newtheorem{definition}{Definition}

\renewcommand{\chaptername}{Chapter}
\renewcommand{\figurename}{Figure}
\renewcommand{\tablename}{Table}
\renewcommand{\contentsname}{Table of Contents}
\renewcommand{\listfigurename}{List of Figures}
\renewcommand{\listtablename}{List of Tables}
\renewcommand{\listalgorithmname}{List of Algorithms}
\renewcommand{\appendixname}{Appendix}
\renewcommand{\algorithmname}{Algorithm}


% ******************************************************************************
% Chapter and section structure
\renewcommand{\thechapter}{\arabic{chapter}} % Use natural numbers for chapter numbering
\renewcommand{\thesection}{\thechapter.\arabic{section}} % Use natural numbers for section numbering
%\renewcommand{\thesubsection}{\arabic{section}.\arabic{subsection}} % Use for subsection numbering
%\renewcommand{\thesubsubsection}{\arabic{section}.\arabic{subsection}.\arabic{subsubsection}} % Use for subsubsection numbering

% Number figures, tables, and equations by chapter
\renewcommand{\thefigure}{\thechapter.\arabic{figure}} % Number figures as Chapter.Figure
\renewcommand{\thetable}{\thechapter.\arabic{table}}   % Number tables as Chapter.Table
\renewcommand{\theequation}{\thechapter.\arabic{equation}} % Number equations as Chapter.Equation
% \renewcommand{\thealgorithm}{\thealgorithm.\arabic{algorithm}}

% \renewcommand{\thefigure}{\arabic{figure}} % Use flat numbering for figures
% \renewcommand{\theequation}{\arabic{equation}} % Use flat numbering for equations
\renewcommand{\thealgorithm}{\arabic{algorithm}} % Number algorithms with natural numbers
% \thealgorithm.

% Define acronym list command
\newcommand{\acronym}[2]{\textbf{#1} (\textit{#2})}

% Customize figure list title and spacing
\renewcommand{\cftfigpresnum}{Figure~} % Prefix before figure number in list
% \renewcommand{\cftfigpresnum}{Figure~} % Vietnamese: Hình~
\renewcommand{\cftfigaftersnum}{:} % Character after figure number
\addtolength{\cftfignumwidth}{3em} % Increase space between number and caption in figure list

% Configure table list title and spacing
\renewcommand{\cfttabpresnum}{Table~}
% \renewcommand{\cfttabpresnum}{Table~} % Vietnamese: Bảng~
\renewcommand{\cfttabaftersnum}{:}
\addtolength{\cfttabnumwidth}{3em} % Increase space between number and caption in table list

% Create list of algorithms
\makeatletter
\begingroup
\let\newcounter\@gobble
\let\setcounter\@gobbletwo
\globaldefs\@ne
\let\c@loadepth\@ne
\newlistof{algorithms}{loa}{\listalgorithmname}
\endgroup
\let\l@algorithm\l@algorithms
\makeatother

% Customize algorithm list
\renewcommand\cftalgorithmsaftersnum{:}
\renewcommand\cftalgorithmspresnum{Algorithm~}
% \renewcommand\cftalgorithmspresnum{Algorithm~} % Vietnamese: Thuật toán~
\cftsetindents{algorithms}{1.5em}{7em}

% \settocstyle{algorithm}{%
	%   \tocstyle{list}[\textbf{Algorithms}]{\thealgorithm}%
	%   \contentsline{algorithm}{\ignorespaces\thealgorithm}{\thealgorithm}%
	% }
% \tocstyle{algorithm}{\listofalgorithms}

% ******************************************************************************
% Autoreference labels (used by \autoref)
\renewcommand{\figureautorefname}{Figure}
\renewcommand{\tableautorefname}{Table}
\renewcommand{\sectionautorefname}{Section}
\renewcommand{\chapterautorefname}{Chapter}
\renewcommand{\subsectionautorefname}{Subsection}
\renewcommand{\subsubsectionautorefname}{Subsubsection}
\renewcommand{\equationautorefname}{Equation}
\renewcommand{\appendixautorefname}{Appendix}

\providecommand{\algorithmautorefname}{Algorithm}
\renewcommand{\algorithmautorefname}{Algorithm}
 % Định nghĩa các lệnh tùy chỉnh

\graphicspath{ {images/} } % Thư mục chứa hình ảnh


\begin{document}

\begin{titlepage}

\begin{center}
%ĐẠI HỌC QUỐC GIA THÀNH PHỐ HỒ CHÍ MINH\\
TRƯỜNG ĐẠI HỌC KHOA HỌC TỰ NHIÊN\\
\textbf{KHOA CÔNG NGHỆ THÔNG TIN}\\[2cm]

\begin{figure}[htp]
\centering
\includegraphics[width=8 cm]{images/logo-khtn.png}
% \caption{Hình ví dụ 1}
%\label{fig:vd1}
{\\[1cm]}
\end{figure}

{ \Large \bfseries \tenSV \\[1cm] } 

%Tên đề tài Khóa luận tốt nghiệp/Đồ án tốt nghiệp

{ \Large \bfseries BÁO CÁO KHÓA LUẬN TỐT NGHIỆP ĐỰ ĐOÁN LIÊN KẾT TRONG ĐỒ THỊ TRI THỨC\\[2cm]} 


%Chọn trong các dòng sau
\large KHÓA LUẬN TỐT NGHIỆP CỬ NHÂN\\
%\large ĐỒ ÁN TỐT NGHIỆP CỬ NHÂN\\
%\large THỰC TẬP TỐT NGHIỆP CỬ NHÂN\\
%Đưa vào dòng này nếu thuộc chương trình Chất lượng cao, hoặc lớp Cử nhân tài năng
\large CHƯƠNG TRÌNH HOÀN CHỈNH\\
% \large CHƯƠNG TRÌNH CHÍNH QUY\\
%\large CHƯƠNG TRÌNH CHẤT LƯỢNG CAO\\
%\large CHƯƠNG TRÌNH CỬ NHÂN TÀI NĂNG\\[2cm]


\begin{tikzpicture}[remember picture, overlay]
  \draw[line width = 2pt] ($(current page.north west) + (2cm,-2cm)$) rectangle ($(current page.south east) + (-1.5cm,2cm)$);
\end{tikzpicture}

\vfill
Tp. Hồ Chí Minh, tháng 10/2020

\end{center}

\pagebreak



\begin{center}

TRƯỜNG ĐẠI HỌC KHOA HỌC TỰ NHIÊN\\
\textbf{KHOA CÔNG NGHỆ THÔNG TIN}\\[2cm]


{\large \bfseries Phan Minh Tâm - 18424059\\} 
{\large \bfseries Hoàng Minh Thanh - 18424062\\[2cm]}

%Tên đề tài Khóa luận tốt nghiệp/Đồ án tốt nghiệp

{ \Large \bfseries  BÁO CÁO KHÓA LUẬN TỐT NGHIỆP ĐỰ ĐOÁN LIÊN KẾT TRONG ĐỒ THỊ TRI THỨC\\[2cm] } 


%Chọn trong các dòng sau
\large KHÓA LUẬN TỐT NGHIỆP CỬ NHÂN\\
%\large ĐỒ ÁN TỐT NGHIỆP CỬ NHÂN\\
%Đưa vào dòng này nếu thuộc chương trình Chất lượng cao, hoặc lớp Cử nhân tài năng
\large CHƯƠNG TRÌNH HOÀN CHỈNH\\[2cm]
%\large CHƯƠNG TRÌNH CHẤT LƯỢNG CAO\\[2cm]
%\large CHƯƠNG TRÌNH CỬ NHÂN TÀI NĂNG\\[2cm]

\textbf{GIÁO VIÊN HƯỚNG DẪN}\\
\tenGVHD\\
\tenBM\\

\begin{tikzpicture}[remember picture, overlay]
  \draw[line width = 2pt] ($(current page.north west) + (2cm,-2cm)$) rectangle ($(current page.south east) + (-1.5cm,2cm)$);
\end{tikzpicture}

\vfill
Tp. Hồ Chí Minh, tháng 10/2020

\end{center}

\end{titlepage}


\vspace{2cm}


\section*{\centering  \Large DECLARATION}
\phantomsection
\addcontentsline{toc}{section}{Declaration}

\vspace{2cm}

{
	We hereby declare that this is our own research work. The data and research results presented in this thesis are truthful and have not been duplicated from any other projects.
	
	All research results presented in this thesis are entirely truthful and accurate.
	
	\begin{table}[H]
		\centering
		\begin{tabularx}{\textwidth}{X X}
			\textbf{\begin{tabular}[c]{@{}c@{}} \end{tabular}} &
			\textbf{\begin{tabular}[c]{@{}c@{}}
					\textit{Ho Chi Minh City, day ... month ... year ... } \\
					CANDIDATE \\
					\textit{(Signature and full name)}
			\end{tabular}}
		\end{tabularx}
	\end{table}
	\pagebreak
}



\pagebreak

\chapter*{Lời cảm ơn}
\label{thanks}

%Tôi xin chân thành cảm ơn các thầy cô 

Chúng tôi xin chân thành cảm ơn thầy Lê Ngọc Thành đã tận tình hướng dẫn, truyền đạt kiến thức và kinh nghiệm, và đưa ra các giải pháp cho chúng tôi trong suốt quá trình thực hiện đề tài luận văn tốt nghiệp này.

Xin gửi lời cảm ơn đến quí thầy cô Khoa Công Nghệ Thông Tin trường Đại Học Khoa Học Tự Nhiên - Đại Học Quốc Gia Thành Phố Hồ Chí Minh, những người đã truyền đạt kiến thức quý báu cho em trong thời gian học tập vừa qua.

Đồng thời cảm ơn các nhà khoa học đã nghiên cứu về đề tài mà chúng tôi đã trích dẫn để có thể có những kiến thức hoàn thiện luận văn của chúng tôi.

Sau cùng chúng tôi xin gửi lời cảm ơn đến gia đình, bạn bè,.. những người luôn động viên, giúp đỡ em trong quá trình làm luận văn. 

Một lần nữa, xin chân thành cảm ơn !



% Ngắt việc thêm mục lục vào chính nó
\addtocontents{toc}{\protect\setcounter{tocdepth}{-1}}
% Căn giữa chữ "Mục lục"
% Bỏ qua tiêu đề tự động của \tableofcontents
\renewcommand{\contentsname}{}


%\renewcommand{\cftchappresnum}{\chaptername222 \quad \ \MakeUppercase} 
% Điều chỉnh cách hiển thị số chương trong TOC
\renewcommand{\cftchappresnum}{\chaptername\ \MakeUppercase} 
% Điều chỉnh khoảng cách giữa số chương và tiêu đề trong TOC
\setlength{\cftchapnumwidth}{6em} % Điều chỉnh độ rộng số chương
\setlength{\cftchapindent}{0em}     % Điều chỉnh độ thụt đầu dòng

\begin{center}
\textbf{\huge Mục lục}
\end{center}
%\addcontentsline{toc}{chapter}{Mục lục}

\tableofcontents
% Khôi phục lại thêm các mục vào mục lục sau đó
\addtocontents{toc}{\protect\setcounter{tocdepth}{2}}


\pagebreak


% Manually add "List of Figures" to the table of contents without using \listoffigures
\phantomsection
\addcontentsline{toc}{section}{List of Figures}

% Temporarily disable adding entries to the table of contents
\addtocontents{toc}{\protect\setcounter{tocdepth}{-1}}

% Use \listoffigures without automatically adding it to the table of contents
%LIST OF FIGURES
\renewcommand{\listfigurename}{\makebox[\linewidth]{\Large LIST OF FIGURES}}

\listoffigures

% Restore table of contents depth for subsequent sections
\addtocontents{toc}{\protect\setcounter{tocdepth}{2}}

\pagebreak

% Manually add "List of Tables" to the table of contents without relying on \listoftables
\phantomsection
\addcontentsline{toc}{section}{List of Tables}

% Temporarily disable adding entries to the table of contents
\addtocontents{toc}{\protect\setcounter{tocdepth}{-1}}

% Use \listoftables without automatically adding it to the table of contents
% LIST OF TABLES
\renewcommand{\listtablename}{\makebox[\linewidth]{\Large LIST OF TABLES}}

\listoftables

% Restore table of contents depth for subsequent sections
\addtocontents{toc}{\protect\setcounter{tocdepth}{2}}

\pagebreak
% Add "List of Algorithms" to the table of contents manually (without using \listoftables)
\phantomsection
\addcontentsline{toc}{section}{List of Algorithms}
% List of Algorithms

% Temporarily disable adding entries to the table of contents
\addtocontents{toc}{\protect\setcounter{tocdepth}{-1}}

\pagebreak
\phantomsection

% Use \listofalgorithms without automatically adding it to the table of contents
% LIST OF ALGORITHMS
\renewcommand{\listalgorithmname}{\makebox[\linewidth]{\Large LIST OF ALGORITHMS}}

\listofalgorithms

% Restore table of contents depth for subsequent sections
\addtocontents{toc}{\protect\setcounter{tocdepth}{2}}

\pagebreak

\pagebreak
\phantomsection
\addcontentsline{toc}{section}{Danh mục các ký hiệu, các chữ viết tắt}
%List of Symbols and Abbreviations
%LIST OF SYMBOLS AND ABBREVIATIONS
\section*{\textbf{ \Large DANH MỤC CÁC KÝ HIỆU, CÁC CHỮ VIẾT TẮT}}


%\renewcommand{\listalgorithmname}{\makebox[\linewidth]{\Large DANH MỤC THUẬT TOÁN}}
%
%\listofalgorithms
%
%\addtocontents{toc}{\protect\setcounter{tocdepth}{2}}

\begin{center}
	\begin{longtable}{|p{2cm}|p{14cm}|}
		\caption*{Danh sách Ký hiệu} \\
		\hline
		\textbf{Ký hiệu} & \textbf{Mô tả} \\
		\hline
		$\mathcal{G}$ & Đồ thị \\
		\hline
		$\mathcal{G}_{\text{mono}}$ & Đồ thị đồng nhất \\
		\hline
		$\mathcal{G}_{\text{hete}}$ & Đồ thị không đồng nhất \\
		\hline
		$\mathcal{G}_{\text{know}}$ & Đồ thị tri thức \\
		\hline
		$V, E$ & Tập hợp đỉnh, Tập hợp cạnh \\
		\hline
		$e, e_i$ & Thực thể, thực thể thứ $i$ \\
		\hline
		$r, r_k$ & Quan hệ, quan hệ thứ $k$ \\
		\hline
		$t_{ijk}, t_{ij}^k$ & Một cạnh/bộ ba \\
		\hline
		$\overrightarrow{e}, \overrightarrow{r}$ & Thực thể nhúng, quan hệ nhúng \\
		\hline
		$\langle h, r, t \rangle$ & Bộ ba gồm thực thể đỉnh (head), quan hệ (relation), thực thể đuôi (tail) \\
		\hline
		$T^v, T^e$ & Tập hợp loại đỉnh, loại cạnh \\
		\hline
		$N_e, N_r$ & Số lượng tập thực thể, tập quan hệ \\
		\hline
		$N_{\text{head}}$ & Số lượng đỉnh tự chú ý \\
		\hline
		$\mathbb{R}$ & Tập số thực \\
		\hline
		$\mathbf{E}, \mathbf{R}$ & Ma trận nhúng thực thể, quan hệ \\
		\hline
		$\mathbf{S}$ & Tập dữ liệu huấn luyện \\
		\hline
		$\ast$ & Phép tính tích chập \\
		\hline
		$\sigma$ & Hàm biến đổi phi tuyến tính \\
		\hline
		$\mathbf{W}$ & Ma trận trọng số \\
		\hline
		$\bigparallel_{k=1}^{K}$ & Phép ghép chồng từ $1$ đến $K$ lớp \\
		\hline
		$||$ & Phép ghép chồng \\
		\hline
		${.}^T$ & Phép chuyển vị \\
		\hline
		$|| W ||^2_2$ & Chuẩn hóa L2 \\
		\hline
		$\vee, \wedge$ & Phép hội, phép giao \\
		\hline
		$\oplus$ & Phép toán hai ngôi \\
		\hline
		$\cap$ & Phép hợp \\
		\hline
		$\neg$ & Phép phủ định \\
		\hline
		$\bigwedge^n_{i=1}$ & Phép nối liền \\
		\hline
		$\gamma$ & Biên lề \\
		\hline
		$\mu$ & Tốc độ học \\
		\hline
		$\omega$ & Số lượng lớp tích chập \\
		\hline
		$\Omega$ & Bộ lọc tính tích chập \\
		\hline
	\end{longtable}
\end{center}


\pagebreak

\pagebreak
\phantomsection
\addcontentsline{toc}{section}{Bảng chú thích thuật ngữ}
%GLOSSARY OF TERMS
\section*{\textbf{ \Large BẢNG CHÚ THÍCH THUẬT NGỮ}}

%\begin{center}
%\begin{tabular}{|p{5cm}|p{10cm}|}
%\hline
%\textbf{Abbreviation} & \textbf{Full Meaning} \\
%\hline
%DDPM & Denoising Diffusion Probabilistic Models \\
%\hline
%OHGesture & Proposed model of the thesis \\
%\hline
%Sampling/Inference & The process of sampling or inference in the model \\
%\hline
%KL & Kullback–Leibler Divergence \\
%\hline
%MAE & Mean Absolute Error \\
%\hline
%MSE & Mean Squared Error \\
%\hline
%Forward Diffusion Process & A noise-adding process without weight-based training \\
%\hline
%\end{tabular}
%\end{center}

\begin{center}
	\begin{longtable}{|p{5cm}|p{10cm}|}
		\hline
		\textbf{Thuật ngữ} & \textbf{Ý nghĩa đầy đủ} \\
		\hline
		\endfirsthead
		
		\hline
		\textbf{Thuật ngữ} & \textbf{Ý nghĩa đầy đủ} \\
		\hline
		\endhead
		
		\hline
		\endfoot
		
		\hline
		\endlastfoot
		
		AnyBURL & Anytime Bottom-up Rule Learning - Thuật toán học luật từ dưới lên bất kỳ lúc nào \\
		\hline
		
		Knowledge Graph ($\mathcal{G}_{\text{know}}$) & Đồ thị tri thức - Tập hợp các ground atoms hoặc facts \\
		
		\hline
		
		ConvE / ConvKB & Mô hình CNN cho nhúng đồ thị - Biến thể dùng vector nhúng cho $\langle h, r, t \rangle$ \\
		\hline
		
		Graph Attention Network (GAT) & Mạng đồ thị chú ý - Sử dụng attention để tổng hợp thông tin từ lân cận \\
		\hline
		
		KBGAT & GAT có vector nhúng quan hệ - Biến thể GAT kết hợp thêm nhúng quan hệ \\
		\hline
		
%		Unary Rules ($U_d$) & Quy tắc đơn nguyên kết thúc bằng đỉnh treo - Head atoms chỉ chứa 1 biến \\
%		\hline
%		Unary Rules ($U_c$) & Quy tắc đơn nguyên kết thúc bằng atom - Head atoms chỉ chứa 1 biến \\
%		\hline
%		Confidence Score & Độ tin cậy - Tỷ lệ body atoms dẫn đến head atoms \\
%		\hline
		Saturation (SAT) & Độ bão hòa - Tỷ lệ luật mới học được so với luật đã học \\
		\hline
	\end{longtable}
\end{center}

\pagebreak


\pagenumbering{arabic}

%\chapter{INTRODUCTION}
%\label{Introduction}
\chapter{Introduction}
\label{chap:Introduction}

Nowadays, graphs have been applied in all aspects of life. Social network graphs (e.g., Facebook \cite{ugander2011anatomy}) illustrate how individuals are connected to each other, the places we visit, and the information we interact with. Graphs are also used as core structures in video recommendation systems (e.g., YouTube \cite{baluja2008video}), flight networks, GPS navigation systems, scientific computations, and even brain connectivity analysis. Google’s Knowledge Graph \cite{googlekg:2020}, introduced in 2012 \cite{ji2020survey}, is a notable example of how information can be structured and utilized in knowledge graphs.

Effectively exploiting knowledge graphs provides users with deeper insight into the underlying data, which can benefit many real-world applications. However, in practice, new knowledge is continuously generated, and the acquired information is often incomplete or missing. This leads to the problem of knowledge graph completion or link prediction in knowledge graphs.

Most current approaches aim to predict a new edge connecting two existing nodes. Such methods help make the graph more complete—i.e., denser—by introducing additional connecting edges. However, these approaches primarily address the problem of completion rather than the challenge of integrating new knowledge into the graph, which remains an open question. Currently, research in knowledge graph completion follows two main directions: one is optimizing an objective function to make predictions with minimal error, as in RuDiK \cite{ortona2018robust}, AMIE \cite{galarraga2015fast}, and RuleN \cite{meilicke2018fine}, which are typically used in vertex or edge classification applications. The other approach generates a ranked list of \(k\) candidate triples, where the score reflects decreasing confidence, as seen in studies such as TransE \cite{bordes2013translating} and ConvKB \cite{vu2019capsule}, which are commonly used in recommendation systems. Our approach follows this second direction of producing a candidate list.

Within these approaches, there are two main methodologies: rule-based systems such as AnyBURL \cite{burl}, and embedding-based methods such as ConvE \cite{dettmers2017convolutional}, TransE \cite{bordes2013translating}, and ComplEx \cite{trouillon2016complex}. With the goal of gaining a systematic understanding of these methods, we chose to explore both directions in this thesis. For the rule-based approach, we selected AnyBURL \cite{burl}, and for the graph embedding-based method, we chose KBAT \cite{nathani2019learning}, which employs attention mechanisms.

Our contribution in the AnyBURL method includes a Python implementation \footnote{https://github.com/MinhTamPhan/mythesis}, along with two proposed strategies for adding new knowledge to the graph, which we term *online-to-offline* and *online-to-online*. The *online-to-offline* strategy extends AnyBURL by generating rules when a batch (set) of new knowledge is added. The *online-to-online* strategy generates rules immediately when a single new piece of knowledge (edge) is added.

For the embedding-based method, we present a review of attention mechanisms \cite{vaswani2017attention}, their application in knowledge graphs via Graph Attention Networks (GATs) \cite{velivckovic2017graph}, and the KBAT model \cite{nathani2019learning}.

Our contribution in the deep learning approach includes a publicly available implementation and training process on GitHub \footnote{https://github.com/hmthanh/GCAT}, with both training code and model results openly provided.


%\chapter{RELATED WORK}
%\label{Chapter2}

\chapter{Các công trình liên quan}
\label{chap:RelatedWork}

Trong phần này chúng tôi sẽ trình bày về các định nghĩa cơ bản về đồ thị tri thức, để từ đó hiểu được nhiệm vụ dự đoán liên kết trong đồ thị tri thức là gì cũng như các hướng nghiên cứu bên khác liên quan về đồ thị tri thức.

\section{Định nghĩa đồ thị tri thức}

Các định nghĩa cơ bản về đồ thị tri thức được nhóm tác giả Cai, Hongyun\cite{cai2018comprehensive} và Goyal, Palash\cite{goyal2018graph} tổng hợp và phân loại như sau :

\begin{figure}[htp]
	\centering
	\includegraphics[width=7 cm]{images/graph_emb_1.png}
	\caption{Ví dụ về đồ thị đầu vào}
	\label{fig:graphInput}
\end{figure}

\begin{itemize}
	\item \begin{definition}[Đồ thị]\label{def:defGraph}
		\(\mathcal{G} = (V, E)\), trong đó \(v \in V\) là một đỉnh và \(e \in E\) là một cạnh. \(\mathcal{G}\) được liên kết với hàm ánh xạ loại đỉnh \(f_v: V \to T^v\) và hàm ánh xạ loại cạnh: \(f_e: E \to T^e\) .
	\end{definition}
	
	Trong đó: \(T^v\) và \(T^e\) lần lượt là tập hợp các loại đỉnh và loại cạnh. Mỗi đỉnh \(v_i \in V\) thuộc về một loại cụ thể, tức là, \(f_v(v_i) \in T^v\). Tương tự, đối với \(e_{ij} \in E, f_e (e_{ij}) \in T^e\).
	
	\item
	\begin{definition}[Đồ thị đồng nhất]\label{def:homogeneous}
		Đồ thị đồng nhất (homogeneous graph) : \textit{ $\mathcal{G}_{homo} = (V, E)$ là đồ thị trong đó $\mid T^v \mid = \mid T^e \mid = 1$. Tất cả các đỉnh trong $\mathcal{G}$ thuộc về một loại duy nhất và tất cả các cạnh thuộc về một loại duy nhất}.
	\end{definition}
	
	\item
	\begin{definition}[Đồ thị không đồng nhất]\label{def:heterogeneous}
		Đồ thị không đồng nhất (heterogeneous graph) : \textit{$\mathcal{G}_{hete} = (V, E)$ là một đồ thị trong đó $\mid T^v \mid > 1$ hoặc $\mid T^e \mid > 1$. Tức là có nhiều hơn một loại đỉnh hoặc nhiều hơn một loại cạnh}.
	\end{definition}
	
	\item
	\begin{definition}[Đồ thị tri thức]\label{def:knowledgeGraph}
		Đồ thị tri thức (knowledge graph)
		$\mathcal{G}_{know} = (V, R, E)$ là một đồ thị có hướng, có tập đỉnh là biểu diễn cho các thực thể (entities), tập quan hệ biểu diễn các mối quan hệ (relations) giữa các đỉnh, tập cạnh (edges) biểu diễn các sự kiện $E \subseteq V\times R \times V$ là gồm bộ ba subject-property-object. Mỗi cạnh là một mẫu gồm $(\text{entity}_{\text{head}}, \text{relation}, \text{entity}_{\text{tail}})$ (ký hiệu là $\langle h, r, t \rangle$) biểu thị mối quan hệ của $r$ từ thực thể $h$ đến thực thể $t$ .
	\end{definition}
	Trong đó $h, t \in V$ là các thực thể và $r \in R$ là các quan hệ. Chúng ta gọi $\langle h, r, t \rangle$ một bộ ba (triples) đồ thị tri thức.
	
	Ví dụ: trong \autoref{fig:graphExample} có hai bộ ba: 
	$\langle \text{Tom Cruise, born\_in, New York} \rangle$
và $\langle \text{New York, state\_of, U.S} \rangle$. Lưu ý rằng các t
hực thể và quan hệ trong đồ thị tri thức thường có các loại khác nhau. Do đó, đồ thị tri thức có thể được xem như là một ví dụ của đồ thị không đồng nhất.
\end{itemize}

\section{Dự đoán liên kết trong đồ thị tri thức}

Dự đoán liên kết (link prediction) hay hoàn thiện đồ thị tri thức (knowledge graph completion) là nhiệm vụ khai thác những sự kiện có sẵn trong đồ thị tri thức để suy luận ra sự kiện còn thiếu. Điều này tương đương với việc đoán đúng thực thể đuôi $\langle h, r, ? \rangle$ (dự đoán đuôi) hoặc $\langle ?, r, t \rangle$ (dự đoán đầu). Để đơn giản,  thay vì gọi dự đoán đầu hoặc đuôi, một cách tổng quát chúng ta gọi thực thể nguồn (source) là thực thể đã biết trong việc dự đoán, thực thể đích (target) là cái chúng ta cần dự đoán.


Hầu hết các nghiên cứu hiện tại về việc dự đoán liên kết của đồ thị tri thức đều liên quan đến các phương pháp tiếp cận tập trung vào khái niệm nhúng một đồ thị đã cho trong một không gian vectơ có số chiều thấp. Ngược lại với các tiếp cận này là một phương pháp đựa trên luật được nghiên cứu trong \cite{burl}. Thuật toán cốt lõi của nó dựa trên lấy mẫu một luật bất kỳ, sau đó khái quát  thành các quy tắc Horn\cite{wiki:Horn}. Tiếp đó dùng thống kê để tính độ tin cậy của các luật được khái quát. Khi dự đoán một liên kết mới (cạnh mới) của đồ thị chúng ta dự đoán một đỉnh có cạnh nối với một quan hệ cụ thể (label) với đỉnh còn lại hay không. Cũng đã có rất nhiều phương pháp được nghiên cứu, đề xuất để học các các luật trong đồ thị chẳng hạn như trong  RuDiK\cite{ortona2018robust}, AMIE\cite{galarraga2015fast}, RuleN\cite{meilicke2018fine}. 
Như đã nói trong phần trước có hai cách tiếp cận chính cho bài toán này một là tối ưu hóa hàm mục tiêu. Tìm ra một bộ quy tắc nhỏ bao gồm phần lớn các ví dụ là đúng và ít sai sót nhất có thể như được ngiên cứu trong RuDiK\cite{ortona2018robust}. Còn cách tiếp cận còn lại cũng là cách tiếp cận mà chúng tôi chọn nghiên cứu là cố gắng tìm hiểu mọi quy tắc khả thi có thể sau đó tạo xếp hạng \(k\) ứng viên tiềm năng với một độ tin cậy nhất định được đo trên tập huấn luyện.

Phương pháp đựa trên luật của chúng tôi phần lớn dựa vào phương pháp Anytime Bottom-Up Rule Learning for Knowledge Graph Completion \cite{meilicke2019anytime} mà sau đây chúng tôi gọi là \textbf{AnyBURL}. Như tên của phương pháp này phương pháp chủ yếu chú trọng vào vấn đề hoàn thành đồ thị, điền những phần còn thiếu vào đồ thị. Vấn đề tồn đọng lại ở mô hình này khi có một cạnh mới hay một tri thức mới được thêm vào đồ thị sẽ phải đào tạo lại toàn bộ mô hình. Chúng tôi giải quyết vẫn đề này theo hai chiến lược offline-to-online tức là khi thêm vào đồ thị tập hợp các cạnh thì mới thực hiện lại quá trình đào tạo lại một phần của đồ thị và chiến lược thứ 2 là online-to-online  khi thêm một cạnh mới sẽ thực hiện đào tạo lại ngay một phần có liên quan tới cạnh vừa thêm vào.

% GAT
Trong nhánh các phương pháp về học sâu, rất nhiều kỹ thuật học sâu thành công trong xử lý ảnh và xử lý ngôn ngữ tự nhiên được áp dụng vào đồ thị tri thức như : Mạng Neural Tích Chập (Convolution Neural Network - CNN \cite{lecun1999object}), Mạng Neural Hồi Quy (Recurrent Neural Network\cite{hopfield2007hopfield}), và gần đây như Transformer (\cite{yang2019xlnet}), Mạng Neural Bao Bọc (Capsule Neural Network - CapsNet \cite{sabour2017dynamic}). Bên cạnh đó các nghiên cứu còn sử dụng một số kỹ thuật khác như Random Walks, các mô hình dựa trên cấu trúc phân cấp, .. Ưu điểm chung của nhóm các phương pháp học sâu trên đồ thị tri thức đó là tự động rút trích các đặc trưng và có thể khái quát hóa cấu trúc phức tạp của đồ thị dựa trên một lượng lớn dữ liệu huấn luyện. Tuy nhiên, một số phương pháp chỉ chủ yếu tập trung vào cấu trúc dạng lưới mà không giữ được đặc trưng không gian của đồ thị tri thức. 
Cơ chế chú ý hay lớp chú ý đa đỉnh (multi-head attention layer) đã được áp dụng vào đồ thị bằng mô hình Mạng Đồ Thị Chú Ý (Graph Attention Network - GAT \cite{velivckovic2017graph}) giúp tổng hợp thông tin của một thực thể dựa vào trọng số chú ý của thực thể gốc đối với các thực thể lân cận. Tuy nhiên, mô hình đồ thị chú ý lại thiếu thông tin của vector nhúng quan hệ cũng như các vector nhúng lân cận của một thực thể gốc, một phần rất quan trọng giúp thể hiện vai trò của từng thực thể. Vấn đề đó đã được giải quyết trong báo cáo Learning Attention-based Embeddings for Relation Prediction in
Knowledge Graphs (\textbf{KBAT} \cite{nathani2019learning}), mô hình được chúng tôi chọn làm cơ sở nghiên cứu.
Cơ chế chú ý đang là một trong những cấu trúc học sâu đạt được hiệu quả nhất hiện nay (state-of-the-art) vì nó đã được chứng minh là thay thế cho bất kỳ phương pháp tính tích chập nào \cite{cordonnier2019relationship},
hơn nữa nó cũng nằm trong cấu trúc cơ bản để áp dụng trên các mô hình mới nhất trên ngôn ngữ tự nhiên như mô hình Megatron-LM \cite{shoeybi2019megatron}, và trên phân đoạn hình ảnh như mô hình HRNet-OCR (Hierarchical Multi-Scale Attention \cite{tao2020hierarchical}). Một số phương pháp thú vị \cite{cordonnier2020multi} đã cải tiến dựa trên cơ chế chú ý, tuy nhiên nó lại chưa được áp dụng vào đồ thị tri thức, vì vậy chúng tôi chọn nhóm phương pháp này để áp dụng các cải tiến mới nhất vào đồ thị tri thức.

\section{Các lĩnh vực nghiên cứu về đồ thị tri thức}

\begin{figure}[htp]
	\centering
	\tikzset{
		category/.style  = {draw, font=\sffamily, thin, align=center},
		subcat/.style={rectangle, rounded corners=6pt},
		center/.style = {category, ellipse, fill=blue!60, text width=4em},
		group/.style = {category, subcat, fill=blue!30, rounded corners=6pt, text width=6em},
		yellowbox/.style = {category, subcat, fill=yellow!30},
		greenbox/.style = {category, subcat, fill=green!30},
		redbox/.style = {category, subcat, fill=red!30},
		bluebox/.style = {category, subcat, fill=cyan!30},
		leafbox/.style = {category, subcat, fill=black!10, rounded corners=0mm}
	}
	\resizebox{\textwidth}{!}{%
		\begin{tikzpicture}
			\node[center] (root) {Đồ thị tri thức};
			\node[group][above left=1cm of root] (c1) {Học biểu diễn tri thức};
			\node[group][above right=8mm of root] (c2) {Nhận biết tri thức};
			\node[group][below left=5mm of root] (c3) {Thu nhận tri thức};
			\node[group][below right=20mm of root, xshift=-2cm] (c4) {Đồ thị tri thức thời gian};
			
			\begin{scope}[every node/.style={yellowbox}]
				\node[above=8mm of c1, xshift=-1cm] (c11) {Biểu diễn không gian};
				\node[left=of c1, yshift=9mm, xshift=-10mm] (c12) {Hàm đánh giá};
				\node[left=15mm of c1, yshift=-5mm] (c13) {Mã hóa mô hình};
				\node[left=of c1, yshift=-20mm] (c14) {Thông tin tương tự};
			\end{scope}
			
			\begin{scope}[every node/.style={redbox}]
				\node[above left=of c2, text width=35mm, yshift=1mm, xshift=18mm] (c21) {Hiểu ngôn ngữ tự nhiên};
				\node[above=1cm of c2, yshift=5mm, xshift=13mm] (c22) {Trả lời câu hỏi};
				\node[right=of c2, yshift=15mm] (c23) {Hệ thống hội thoại};
				\node[right=of c2, yshift=-2mm] (c24) {Hệ thống gợi ý};
				\node[below=5mm of c2, xshift=5mm] (c25) {Ứng dụng khác};
			\end{scope}
			
			\begin{scope}[every node/.style={greenbox}]
				\node[left=1cm of c3, yshift=5mm] (c31) {Khai phá thực thể};
				\node[below left=of c3, yshift=1cm, xshift=3mm] (c32) {Rút trích quan hệ};
				\node[below right=5mm of c3, xshift=-4cm] (c33) {Hoàn thiện đồ thị};
			\end{scope}
			
			\begin{scope}[every node/.style={bluebox}]
				\node[below=of c4, xshift=-15mm, yshift=-35mm] (c41) {Lý giải logic thời gian};
				\node[below=of c4, xshift=2cm, yshift=-23mm] (c42) {Quan hệ thời gian độc lập};
				\node[below right=of c4,xshift=-2mm, yshift=-10mm] (c43) {Thực thể động};
				\node[right=of c4, yshift=-2cm, xshift=15mm] (c44) {Nhúng thời gian};
			\end{scope}
			
			\begin{scope}[every node/.style={leafbox}]
				\node[above=5mm of c11] (c11x) {
					\begin{tabular}{@{}l@{}@{}l@{}}
						-Point-wise & -Đa tạp \\
						-Số phức & -Gausian \\
						-Rời rạc & \\
					\end{tabular}
				};
				\node[above=5mm of c12, xshift=-5mm] (c12x) {
					\begin{tabular}{@{}l@{}}
						-Khoảng cách \\
						-Ngữ nghĩa \\
						-Khác \\
					\end{tabular}
				};
				\node[left=of c13, yshift=5mm] (c13x) {
					\begin{tabular}{@{}l@{}}
						-Tuyến tính/ \\song tuyến tính \\
						-Ma trận hóa \\
						-Neural Nets \\
						-CNN \\
						-RNN \\
						-Transformers \\
						-GCN \\
					\end{tabular}
				};
				\node[below left=5mm of c14, xshift=6mm, yshift=-7mm] (c14x) {
					\begin{tabular}{@{}l@{}c@{}l@{}}
						-Văn bản & -Kiểu & -Trực quan \\
					\end{tabular}
				};
				%%%%%%%%%%%% 
				\node[below left=of c31] (c31x) {
					\begin{tabular}{@{}l@{}}
						-Nhận dạng \\
						-Định kiểu \\
						-Phân biệt \\
						-Sắp xếp \\
					\end{tabular}
				};
				\node[below=of c32, xshift=-5mm] (c32x) {
					\begin{tabular}{@{}l@{}}
						-Neural Nets\\
						-Chú ý \\
						-GCN \\
						-GAN \\
						-RL \\
						-Khác \\
					\end{tabular}
				};
				\node[below=5mm of c33] (c33x) {
					\begin{tabular}{@{}l@{}}
						-Nhúng dựa trên xếp hạng\\
						-Lý giải đựa trên đoạn \\
						-Lý giải dựa trên luật \\
						-Học siêu quan hệ \\
						-Phân loại bộ ba \\
					\end{tabular}
				};
				%%%%%%%%%%%%%%%%%
				\node[above=5mm of c22] (c22x) {
					\begin{tabular}{@{}l@{}}
						-Single-fact QA\\
						-Lý giải nhiều bước \\
					\end{tabular}
				};
				\node[below right=5mm of c25, yshift=15mm] (c25x) {
					\begin{tabular}{@{}l@{}}
						-Sinh câu hỏi\\
						-Công cụ tìm kiếm \\
						-Ứng dụng y khoa \\
						-Hồi phục sức khỏe \\
						-Phân loại ảnh zero-shot\\
						-Sinh văn bản\\
						-Phân tích ngữ nghĩa\\
					\end{tabular}
				};
				
			\end{scope}
			
			\foreach \value in {1,...,4}
			\draw[->, line width=0.8mm] (root) -> (c\value);
			
			\foreach \value in {1,...,4}
			\draw[->, ultra thick] (c1) -> (c1\value);
			\foreach \value in {1,...,4}
			\draw[->, thick] (c1\value) -> (c1\value x);
			
			\foreach \value in {1,...,5}
			\draw[->, ultra thick] (c2) -> (c2\value);
			\foreach \value in {2,5}
			\draw[->, thick] (c2\value) -> (c2\value x);
			
			\foreach \value in {1,...,3}
			\draw[->, ultra thick] (c3) -> (c3\value);
			\foreach \value in {1,...,3}
			\draw[->, ultra thick] (c3\value) -> (c3\value x);
			
			\foreach \value in {1,...,4}
			\draw[->, ultra thick] (c4) -> (c4\value);
			
	\end{tikzpicture}}
	\caption{
		Danh mục các lĩnh vực nghiên cứu trên đồ thị tri thức}
	\label{fig:categoriesResearch}
\end{figure}

Biểu diễn tri thức đã từng có lịch sử phát triển suốt chiều dài lịch sử trong lĩnh vực logic và trí tuệ nhân tạo. Trên đồ thị tri thức, có 4 bốn nhóm nghiên cứu chính đã được phân loại và tổng hợp ở báo cáo \cite{ji2020survey} bao gồm : Học Biểu Diễn Tri Thức (Knowledge Representation Learning), Thu Nhận Tri Thức (Knowledge Acquisition), Đồ Thị Tri Thức Về Thời Gian (Temporal Knowledge Graphs), Ứng Dụng Nhận Biết Tri Thức (Knowledge-aware Applications). Tất cả các danh mục nghiên cứu được minh họa ở \autoref{fig:categoriesResearch}.

\textbf{Học biểu diễn tri thức}

Học biểu diễn tri thức là vấn đề tìm hiểu thiết yếu của đồ thị tri thức giúp mở ra rất nhiều ứng dụng trong thực tế. Học biểu diễn tri thức được phân loại thành bốn nhóm con bao gồm : 

\begin{itemize}
	\item \textit{Biểu Diễn Không Gian} (Representation Space) nghiên cứu về cách các thực thể và quan hệ được biểu diễn trong không gian. Biểu diễn không gian bao gồm không gian điểm (point-wise), đa tạp (manifold), không gian vector số phức (complex), phân phối Gaussian và không gian rời rạc.
	
	\item \textit{Hàm Đánh Giá} (Scoring Function) nghiên cứu về hàm đo lường giá trị của một bộ ba trong thực tế, bao gồm các hàm đánh giá dựa trên khoảng cách hoặc dựa trên sự tương đồng.
	
	\item \textit{Mã Hóa Mô Hình} (Encoding Models) nghiên cứu về cách biểu diễn và học các tương tác giữa các mối quan hệ. Đây là hướng nghiên cứu chính hiện nay, bao gồm các mô hình tuyến tính hoặc phi tuyến tính, phân rã ma trận hoặc mạng neural.
	
	\item \textit{Thông Tin Bổ Trợ} (Auxiliary Information) nghiên cứu về cách kết hợp vào các phương pháp nhúng, các thông tin bổ trợ bao gồm văn bản, hình ảnh và loại thông tin .
\end{itemize}

\textbf{Thu nhận tri thức}

Thu nhận tri thức nghiên cứu về cách thu nhận tri thức dựa trên đồ thị tri thức, bao gồm hoàn thiện đồ thị (knowledge graph completion), khai thác quan hệ và khai phá thực thể. Khai thác quan hệ và khai phát thực thể là nhóm phương pháp khai thác tri thức mới (bao gồm các quan hệ hoặc thực thể) trong đồ thị từ văn bản. Hoàn thiện đồ thị là nhiệm vụ mở rộng đồ thị tri thức dựa trên đồ thị đang có. Hoàn thiện đồ thị bao gồm các hướng nghiên cứu như : xếp hạng dựa trên nhúng (embedding-based ranking), dự đoán đường đi quan hệ (relation path reasoning), dự đoán dựa trên luật (rule-based reasoning) và học siêu quan hệ.
Khai phá thực thể bao gồm nhận dạng, phân biệt, định kiểu và sắp xếp. 
Các mô hình khai thác quan hệ sử dụng cơ chế chú ý, mạng đồ thị tích chập (graph
convolutional networks), huấn luyện đối nghịch (adversarial training), học tăng cường (reinforcement learning), học sâu và học chuyển tiến (transfer learning), đây là hướng nghiên cứu trong phương pháp đề xuất của chúng tôi.

Ngoài ra, trên đồ thị tri thức còn có các hướng nghiên cứu như \textbf{đồ thị tri thức về thời gian} và \textbf{ứng dụng nhận biết tri thức}. Đồ thị tri thức về thời gian sẽ kết hợp thêm thông tin thời gian trên đồ thị để học cách biểu diễn, còn ứng dụng nhận biết tri thức bao gồm hiểu ngôn ngữ tự nhiên (natural language understanding), trả lời câu hỏi (question answering), hệ thống gợi ý (recommendation systems) và nhiều nhiệm vụ khác trong thế giới thực mà nó tích hợp tri thức vào để cải thiện quá trình học biểu diễn .
\chapter{RULE-BASED METHOD}
\label{chap:RuleBase}

In this chapter, we describe how the problem is reformulated using the rule-based approach AnyBURL, including the rule (path) sampling algorithm and the rule generalization algorithm used to store learned knowledge in the model. We also present our improvements to the training process when new knowledge (edges) is added to the graph.

\section{Horn Clauses}
In mathematical logic, an \textbf{atomic formula} \cite{wiki:Atomic}, also simply called an \textbf{atom}, is a formula that contains no logical connectives such as conjunction (\(\wedge\)), disjunction (\(\vee\)), or biconditional (\(\Leftrightarrow\)). It is a formula with no proper subformulas—meaning that an atom cannot be decomposed into smaller atoms. Thus, atomic formulas are the simplest expressions used to construct logical rules. Compound formulas are formed by combining atomic formulas using logical connectives.

A \textbf{literal} \cite{wiki:Literal} is either an atomic formula or the negation of one. This concept primarily arises in classical logic theory. Literals are classified into two types: A \textbf{positive literal} is simply an atomic formula (e.g., \(x\)). A \textbf{negative literal} is the negation of an atomic formula (e.g., \(\neg x\)). Whether a literal is considered positive or negative depends on its defined form.

A clause is either a single literal or a disjunction of two or more literals. In \textbf{Horn form}, a clause contains at most one positive literal. Note: Not all propositional logic formulas can be converted into Horn form. A clause with no literals is sometimes referred to as a \textit{unit clause}, and a unit clause without variables is often called a \textit{fact} \cite{wiki:Horn}. An atomic formula is referred to as a \textit{ground} or \textit{ground atom} if it is constructed entirely from unit clauses. All possible ground atoms that can be formed from a set of function and predicate symbols make up the Herbrand base for those symbols \cite{wiki:Term}.





\section{Definition of Logical Graph Language} \label{kg}

Unlike general definitions of knowledge graphs commonly used in graph embedding methods, our rule-based approach treats the graph as a formal language. Below are the formal language definitions of the knowledge graph.

A knowledge graph \(\mathcal{G}_{\text{know}}\) is defined over a vocabulary \(\langle \mathbb{C}, \mathbb{R} \rangle\), where \(\mathbb{C}\) is the set of constants and \(\mathbb{R}\) is the set of binary predicates. Then, \(\mathcal{G}_{\text{know}} = \{r(a, b) \mid r \in \mathbb{R}, a, b \in \mathbb{C}\}\) is the set of \textit{ground atoms} or \textit{facts}. A binary predicate is referred to as a relation, and a constant (or referenced constant) is referred to as an entity, corresponding to a data entry in the training set. In what follows, we use lowercase letters for constants and uppercase letters for variables. This is because we do not learn arbitrary Horn rules; instead, we focus only on those rule types that can be generalized as discussed below.

We define a rule as \(h(c_0, c_n) \gets b_1(c_0, c_1), \dots, b_n(c_{n}, c_{n + 1})\), which is a path of ground atoms of length \(n\). Here, \(h(\dots)\) is referred to as the \textit{head atom}, and \(b_1(c_0, c_1), \dots, b_n(c_{n}, c_{n + 1})\) are referred to as the \textit{body atoms}. We distinguish the following three types of rules:

- \textit{Binary rules} \((\mathbf{B})\): Rules in which the head atom contains two variables.
- \textit{Unary rules ending in a dangling node} \((\mathbf{U_d})\): Rules where the head atom contains only one variable, and the rule ends in a body atom that contains only variables (no constants).
- \textit{Unary rules ending in a constant} \((\mathbf{U_c})\): Rules where the head atom also contains only one variable, but the rule ends with an atom that may link to an arbitrary constant. If that constant matches the constant in the head atom, the rule forms a cyclic path.


\begin{equation}
	\begin{aligned}
		B: \quad & h(A_0, A_n) \gets \bigwedge_{i=1}^{n} b_i(A_{i-1}, A_i) \\
		U_d: \quad & h(A_0, c) \gets \bigwedge_{i=1}^{n} b_i(A_{i-1}, A_i) \\
		U_c: \quad & h(A_0, c) \gets \bigwedge_{i=1}^{n-1} b_i(A_{i-1}, A_i) \wedge b_n(A_{n-1}, c')
	\end{aligned}
\end{equation}


%\[B \hspace{3.7cm} h(A_0,A_n) \gets  \bigwedge^n_{i=1} b_i(A_{i-1}, A_i)\]
%\[U_d \hspace{3.8cm} h(A_0,c) \gets  \bigwedge^n_{i=1} b_i(A_{i-1}, A_i)\]
%\[U_c \hspace{1cm} h(A_0,c) \gets  \bigwedge^{n-1}_{i=1} b_i(A_{i-1}, A_i) \wedge b_n(A_{n-1}, c^{\prime})\]



We refer to rules of these types as path rules because the body atoms (the part after the \(\gets\) symbol) form a path. Note that this also includes variants of rules where the variables are reversed within the atoms. Given a knowledge graph \(\mathcal{G}_{\text{know}}\), a path of length \(n\) is a sequence of \(n\) triples \(p_i(c_i, c_{i+1})\) where either \(p_i(c_i, c_{i+1}) \in \mathcal{G}_{\text{know}}\) or \(p_i(c_{i+1}, c_i) \in \mathcal{G}_{\text{know}}\), with \(0 \leq i \leq n\). The abstract rule patterns presented above are considered to have length \(n\), as their body atoms can be instantiated into a path of length \(n - 1\). For example, in \autoref{fig:burl} \footnote{http://web.informatik.uni-mannheim.de/AnyBURL/2019-05/meilicke19anyburl.pdf},  
when sampling paths of length 3, we can obtain the following two rules: the rule marked in green and the rule marked in red.

\begin{equation*}
	\begin{aligned}
		\text{(green)} \quad & speaks(ed, d) \gets married(ed, lisa) \wedge born(lisa, a) \\
		\text{(red)} \quad & speaks(ed, d) \gets lives(ed, nl) \wedge lang(nl, d)
	\end{aligned}
\end{equation*}

\begin{figure*}[h]
	\centering
	\includegraphics[width=12cm]{images/burl-ago.png}
	\caption{Example of a knowledge graph}
	\label{fig:burl}
	\textit{Source: adapted from \href{http://web.informatik.uni-mannheim.de/AnyBURL/2019-05/meilicke19anyburl.pdf}{Anytime Bottom-Up Rule Learning}}
\end{figure*}

In addition, rules of type \(B\) and \(U_c\) are also referred to as closed-path rules. These are utilized by the AMIE model, described in \cite{AMIE,galarraga2015fast}. Rule \(U_d\) is considered an open rule or an acyclic path rule, since \(A_n\) is a variable that appears only once. For example:



\begin{equation*}
\begin{matrix}
	\textit{speaks}(X, Y ) & \gets & \textit{lives}(X, Y) & \quad (1) \\
	\textit{lives\_in\_city}(X, Y ) & \gets & \textit{lives}(X, A),\textit{within}(Y, A)  & \quad  (2) \\
	\textit{gen}(X, female) & \gets & \textit{married}(X, A), \textit{gen}(A, male)  & \quad  (3) \\
	\textit{profession}(X, actor) &  \gets & \textit{acted\_in}(X, A)  & \quad (4)
\end{matrix}
\end{equation*}


Rule (1) is a \textbf{B}-type rule (binary rule). This rule states that a person (entity) \(X\) speaks language \(Y\) if person \(X\) lives in country \(Y\). Clearly, this is a general rule: whenever entity \(X\) has an edge to entity \(Y\) labeled \textit{lives}, we can infer the existence of another edge labeled \textit{speaks} between \(X\) and \(Y\).

Rules (2) and (3) are both \(U_c\)-type rules. Rule (2) states that a person \(X\) lives in city \(Y\) if person \(X\) lives in country \(A\) and city \(Y\) is located in country \(A\). Rule (3) states that a person \(X\) is female if they are married to person \(A\) and person \(A\) is male. In rule (3), there is no cycle formed in the graph, unlike in rule (2), where node \(Y\) is repeated both in the \textit{head atom} and as the final node in the \textit{body atoms}.

Rule (4) is a \(U_d\)-type rule, which states that a person \(X\) is an actor if they acted in a movie \(A\).

All rules under consideration are filtered based on a score called the rule's confidence, which is computed on the training dataset. This confidence score is defined as the number of \textit{body atom} paths that lead to the \textit{head atom}, divided by the total number of paths that contain only the body atoms.

For example, consider the following rule:  
\(\textit{gen}(X, female) \gets \textit{married}(X, A), \textit{gen}(A, male)\).  
We first count all entity pairs that satisfy the relations \(\textit{married}(X, A), \textit{gen}(A, male)\)—this is the number of paths containing the body atoms. Then we count how many of those pairs also satisfy the inferred relation \(\textit{gen}(X, female)\); this is the number of body atom paths that lead to the head atom. The confidence score of the rule is the ratio of the latter to the former.

\section{AnyBURL Algorithm} \label{myalgorithm}
In this section, we describe the core algorithm of the AnyBURL method, as originally introduced in \cite{burl}, as well as our two extended algorithms designed to handle situations where the graph is incrementally updated with one or more new facts (edges). Additionally, we briefly describe how rules are initialized and how rule confidence is computed using sampling on the training set, including the issue of confidence estimation during prediction when sampling is used.


\subsection{AnyBURL}
\begin{algorithm}[H]
	\caption{Anytime Bottom-up Rule Learning}\label{algorithm1}
	\begin{algorithmic}[1]
		\Procedure{AnyBURL($\mathcal{G}_{\text{know}}$, s, sat, Q, ts)}{}
		\State $\textit{n} = \text{2}$
		\State $R = \emptyset$
		\Loop
		\State $R_s = \emptyset$
		\State $start = currentTime()$
		\Repeat
		\State $p = samplePath(\mathcal{G}_{\text{know}}, n)$
		\State $R_p = generateRules(p)$
		\For {$r \in R_p$}
		\State $score(r, s)$
		\If {$Q(r)$}
		\State $R_s = R_s \cup \{r\}$
		\EndIf
		\EndFor
		\Until {$currentTime() > start + ts$}
		\State $R^{\prime}_s = R_s \cap R$
		\If {$ \mid R^{\prime} \mid / \mid R \mid > SAT$}
		\State $n = n + 1$
		\EndIf
		\State $R = R_s \cap R$
		\EndLoop
		\Return R
		\EndProcedure
	\end{algorithmic}
\end{algorithm}

The input of the algorithm consists of \(\mathcal{G}_{\text{know}}, S, SAT, Q, TS\). The output is the set \(R\) of learned rules. Here, \(\mathcal{G}_{\text{know}}\) is a knowledge graph derived from the training dataset. \(S\) is a parameter indicating the sample size used during each sampling iteration on the training data for confidence computation. \(SAT\) denotes the saturation level of the rules generated in each iteration; this saturation is calculated based on the number of \textbf{new} rules learned in the current iteration relative to the total number of rules already learned. If this value is below the saturation threshold, we consider that there is still potential to discover rules of length \(n\). Otherwise, we increase the rule length and continue the rule mining process. \(Q\) is a threshold used to determine whether a newly generated rule should be added to the result set. \(TS\) indicates the total learning time of the algorithm.

We start with \(n = 2\), which corresponds to rules of path length 2, since a valid path rule requires at least one literal in the head atom and one in the body atoms. In the rule sampling step (\textit{samplePath}), we simply select a random node in the graph, traverse all possible paths from that node with length \(n\), and then randomly select one of the traversed paths.


\subsection{Generate Rules}
\label{subsec:CreateRule}

\begin{algorithm}[H]
	\caption{Generate Rules(p)}
	\label{alg:GenerateRules}
	\begin{algorithmic}[1]
		\Procedure{generate\_rules(p)}{}
		\State $\textit{generalizations} = \emptyset$
		\State $is\_binary\_rule = random.choices([true,false])$
		\If {$is\_binary\_rule$}
		\State $replace\_all\_head\_by\_variables(p)$
		\State $replace\_all\_tail\_by\_variables(p)$
		\State $add(generalizations, p)$
		\Else:
		\State $replace\_all\_head\_by\_variables(p)$
		\State $add(generalizations, p)$
		\State $replace\_all\_tail\_by\_variables(p)$
		\State $add(generalizations, p)$
		\EndIf
		\Return $generalizations$
		\EndProcedure
	\end{algorithmic}
\end{algorithm}

In this algorithm, we substitute constants into the head and tail of all path rules from the sampled rule in the previous step if the rule to be learned is a binary rule. Otherwise, we substitute either the head or the tail and then add the rule to the return set. We then sample a set of rules from the training set and compute their confidence scores as described in \autoref{subsec:CreateRule}. To reduce computational cost, we choose to sample from the training set for this calculation. 

When making predictions for rule candidates, we recompute confidence by incorporating an estimated number of incorrect rules not observed during sampling. For our model, after experimenting with the parameter in the range \([5, 10]\), we found that this yields the best results.

\section{Extended AnyBURL Algorithm}
\subsection{Algorithm 3: Offline-to-Online Learning}


\begin{algorithm}[H]
	\caption{BatchAnyBURL Learning batch size}
	\label{alg:BatchAnyBURL}
	\begin{algorithmic}[1]
		\Procedure{BatchAnyBURL($\mathcal{G}_{know}$, sat, Q, ts, batch\_edge)}{}
		\State $is\_connected = add(\mathcal{G}_{\text{know}}, batch\_edge)$
		\If {$is\_connected$}
		\State  $ G^{\prime} = \mathcal{G}_{\text{know}} \oplus batch\_edge$
		\Else
		\State  $ G^{\prime} = batch\_edge$
		\EndIf
		\State $\textit{n} = \text{2}$
		\State $R = \emptyset$
		\Loop
		\State $R_s = \emptyset$
		\State $start = currentTime()$
		\Repeat
		\State $p = samplePath(G^{\prime}, n)$
		\State $R_p = generateRules(p)$
		\For {$r \in R_p$}
		\State $score(r, G^{\prime})$
		\If {$Q(r)$}
		\State $R_s = R_s \cup \{r\}$
		\EndIf
		\EndFor
		\Until {$currentTime() > start + ts$}
		\State $R^{\prime}_s = R_s \cap R$
		\If {$ \mid R^{\prime} \mid / \mid R \mid > SAT$}
		\State $n = n + 1$
		\EndIf
		\State $R = R_s \cap R$
		\EndLoop
		\Return R
		\EndProcedure
	\end{algorithmic}
\end{algorithm}

This algorithm is our proposed extension to avoid retraining the entire model when a new set of knowledge is added to the graph. When new knowledge is added, we first check whether it is connected to the existing knowledge in the graph (i.e., connectivity). If it is, we perform the \(\oplus\) operation by combining all elements in \(batch\_edge\) with the connected components in the graph, up to a path length of 5. If there is no connectivity, we use all elements in \(batch\_edge\) and repeat the steps of the Anytime Bottom-up Rule Learning algorithm.



\subsection{Algorithm 4: Online-to-Online Learning}



\begin{algorithm}[H]
	\caption{EdgeAnyBURL}
	\label{alg:EdgeAnyBURL}
	\begin{algorithmic}[1]
		\Procedure{EdgeAnyBURL($\mathcal{G}_{\text{know}}$, s, sat, Q, ts, edge)}{}
		\State $is\_connected = add(\mathcal{G}_{\text{know}}, edge)$
		\State $R = \emptyset$
		\If {$is\_connected$}
		\State $\textit{n} = \text{2}$
		\State $R_s = \emptyset$
		\Repeat
		\State $p = samplePath(edge, n)$
		\State $R_p = generateRules(p)$
		\For {$r \in R_p$}
		\State $score(r, s)$
		\If {$Q(r)$}
		\State $R_s = R_s \cup \{r\}$
		\EndIf
		\EndFor
		\Until {$currentTime() > start + ts$}
		\State $R^{\prime}_s = R_s \cap R$
		\If {$ \mid R^{\prime} \mid / \mid R \mid > SAT$}
		\State $n = n + 1$
		\EndIf
		\State $R = R_s \cap R$
		\EndIf
		\State \Return R
		\EndProcedure
	\end{algorithmic}
\end{algorithm}

This algorithm is a complementary component to \autoref{alg:BatchAnyBURL}. We refer to it as online-to-online because when a new edge (i.e., new knowledge) is added to the graph, we immediately perform learning on the path rules related to that edge—unlike in \autoref{alg:BatchAnyBURL}, where learning is triggered only after a sufficient amount of new knowledge has been added.


\input{3_ProposedMethod/ProposedMethod.tex}

%\chapter{EXPERIMENTS}
%\label{Chapter4}

%\chapter{Thực nghiệm}
\chapter{EXPERIMENTS}
\label{chap:Experiment}

%\begin{center}
%\begin{tikzpicture}
%	[every axis/.style={
	%		ybar,
	%		scale only axis,
	%		ymin=0, ymax= 50000,
	%		width=0.5\textwidth,
	%		height=0.4\textwidth,
	%		legend style={at={(20em,5em)}, anchor=east},
	%		bar width=1em,
	%		scaled y ticks=false,
	%		xtick=data,
	%		font=\scriptsize\sffamily,
	%		symbolic x coords={Dataset,FB15k,FB15k-237,WN18,WN18RR},
	%		nodes near coords,
	%		nodes near coords align={vertical},
	%	}]
%	\pgfplotsset{
	%		compat=newest,
	%		major grid style=blue,
	%		xlabel near ticks,
	%		ylabel near ticks
	%	}
%	
%	\begin{axis}[]
	%		\addplot [fill=awesome] coordinates {
		%			(FB15k,14951)
		%			(FB15k-237,14541)
		%			(WN18,40943)
		%			(WN18RR,40559)
		%			(YAGO3-10, 123182)
		%		};
	%		\addplot [fill=azure] coordinates {
		%			(FB15k,1345)
		%			(FB15k-237,237)
		%			(WN18,18)
		%			(WN18RR,11)
		%			(YAGO3-10,37)
		%		};
	%		\legend{Entities, Relations}
	%	\end{axis}
%\end{tikzpicture}
%\end{center}

\begin{figure}[h]
	\centering
		\label{fig:dataset}
		\begin{tikzpicture}
			[every axis/.style={
				ybar,
				scale only axis,
				ymin=0, ymax= 130000,
				width=0.5\textwidth,
				height=0.4\textwidth,
				legend style={at={(20em,5em)}, anchor=east},
				bar width=1em,
				scaled y ticks=false,
				xtick=data,
				font=\scriptsize\sffamily,
				symbolic x coords={FB15k,FB15k-237,WN18,WN18RR,YAGO3-10},
				nodes near coords,
				nodes near coords align={vertical},
			}]
			\pgfplotsset{
				compat=newest,
				major grid style=blue,
				xlabel near ticks,
				ylabel near ticks
			}
			
			\begin{axis}[]
				\addplot [fill=awesome] coordinates {
					(FB15k,14951)
					(FB15k-237,14541)
					(WN18,40943)
					(WN18RR,40559)
					(YAGO3-10,123182)
				};
				\addplot [fill=azure] coordinates {
					(FB15k,1345)
					(FB15k-237,237)
					(WN18,18)
					(WN18RR,11)
					(YAGO3-10,37)
				};
				\legend{Entities, Relations}
			\end{axis}
		\end{tikzpicture}
	
\end{figure}


In this section, we describe the datasets used for our empirical evaluation, along with a comparison against notable existing methods as reported in \autoref{tab:graphEmbeddingTechCompare}. Additionally, we evaluate our two proposed approaches for injecting new knowledge into the knowledge graph. Specifically, we treat the test set as a batch of new knowledge to be added, and use the validation set to re-evaluate the effectiveness of our method. Detailed results are presented in \autoref{tab:resultOnFreeBase} and \autoref{tab:resultOnWordNet}.


\begin{table}[H]
	\begin{center}
%		\resizebox{\textwidth}{!}{%
			\begin{tabular}{llllll}
				\hline
				&          &           & \multicolumn{3}{l}{\# Edges}    \\ \cline{4-6}
				
				Dataset   & Entities & Relations & Training & Validation & Test    \\ \hline
				FB15k     & 14,951   & 1,345     & 483,142  & 50,000     & 59,071 \\
				FB15k-237 & 14,541   & 237       & 272,115  & 17,535     & 20,466  \\
				WN18      & 40,943   & 18        & 141,442  & 5,000      & 5,000   \\
				WN18RR    & 40,559   & 11        & 86,835   & 3,034       & 3,134    \\
				YAGO3-10    & 123,182   & 37        & 1,079,040   & 5,000       & 5,000  \\
				\hline
			\end{tabular}
%		}
		\caption{Dataset Information}
		\label{tab:datasetInfo}
	\end{center}
\end{table}

%\section{Các tập dữ liệu huấn luyện}
\section{Training Datasets}
\label{sec:DataTraining}

\begin{figure}[H]
	\centering
	\label{fig:dataset_split}
\pgfplotstableread[row sep=\\,col sep=&]{
	Dataset & Entities & Relations & Training & Validation & Test\\
	FB15k & 14951 & 1345 & 483142 & 50000 & 59071  \\
	FB15k-237 & 14541 & 237 & 272115 & 17535 & 20466  \\
	WN18 & 40943 & 18 & 141442 & 5000 & 5000  \\
	WN18RR & 40559 & 11 & 86835 & 3034 & 3134  \\
	YAGO3-10 & 123182 & 37 & 1079040 & 5000 & 5000  \\
}\mydata
\resizebox{\textwidth}{!}{%
	\begin{tikzpicture}
		\begin{axis}[
			xbar stacked,
			tick align = outside, xtick pos = left,
			scale only axis,
			scaled x ticks=false,
			every node near coord/.style={/pgf/number format/fixed},
			xticklabel style={/pgf/number format/fixed},
			width=\textwidth,
			height=0.3\textwidth,
			font=\scriptsize\sffamily,
			legend style={at={(0.5,-0.15)}, anchor=north, legend columns=-1},
			bar width=1.5em,
			ytick=data,
			y dir = reverse,
			yticklabels from table={\mydata}{Dataset},
			]
			\addplot[fill=azure] table [y expr=\coordindex,x=Training]{\mydata};
			\addplot+[fill=awesome] table [y expr=\coordindex,x=Validation]{\mydata};
			\addplot+[fill=amber,
			point meta=x,
			nodes near coords = {\pgfmathprintnumber[precision=1]{\pgfplotspointmeta}},
			nodes near coords align={anchor=west},
			every node near coord/.append style={
				black,
				fill=white,
				fill opacity=0.75,
				text opacity=1,
				outer sep=\pgflinewidth
			}] table [y expr=\coordindex,x=Test]{\mydata};
			\legend{Training, Validation, Test};
		\end{axis}
	\end{tikzpicture}
}
\end{figure}


In our experiments, we evaluate our approach on four widely used benchmark datasets: FB15k, FB15k-237 (\cite{toutanova2015observed}), WN18, and WN18RR (\cite{dettmers2018convolutional}). Each dataset is divided into three subsets: training, validation, and test sets. Detailed statistics for these datasets are presented in \autoref{tab:datasetInfo}. 

Each dataset consists of a collection of triples in the form \(\langle head, relation, tail \rangle\). FB15k and WN18 are derived from the larger knowledge bases FreeBase and WordNet, respectively. However, they contain a large number of inverse relations, which allow most triples to be easily inferred. To address this issue and to better reflect real-world link prediction scenarios, FB15k-237 and WN18RR were constructed by removing such inverse relations.


\subsection{Bộ dữ liệu FB15k}

Bộ dữ liệu này được tạo bởi nhóm nghiên cứu A. Bordes, N. Usunier \cite{bordes2013translating} bằng cách trích xuất từ bộ dữ liệu Wikilinks database \footnote{https://code.google.com/archive/p/wiki-links/}.Wikilinks database thu thập các siêu liên kết (hyperlinks) đến Wikipedia gồm 40 triệu lượt đề cập trên 3 triệu thực thể, họ trích xuất tất cả các dữ kiện liên quan đến một thực thể nhất định có hơn 100 lần được đề cập đến bởi các tài liệu khác cùng với tất cả các dữ kiện liên quan đến thực thể đó (bao gồm cả những thực thể con được nhắc đến trong tài liệu Wikipedia đó), ngoại trừ những thông tin như: ngày tháng, danh từ riêng, v.v ... Họ cũng chuyển đổi các đỉnh có bậc \(n\) được biểu diễn thành các nhóm các cạnh nhị phân tức là liệt kê các cạnh và quan hệ của mọi đỉnh. 

%Tập dữ liệu được chia ngẫu nhiên thành 3 tập: tập training với 1345 relations, 14834 head entities và 14903 tail entities, tập test gồm 916 relations, 11886 head entities, và 11285 tail entities, tập validation gồm 961 relations, 12297 head entities, và 11825 tail entities.

\subsection{Bộ dữ liệu FB15k-237}

Bộ dữ liệu này là một tập hợp con của FB15k được xây dựng bởi Toutanova và Chen \cite{toutanova2015observed} lấy cảm hứng từ quan sát rằng FB15k bao gồm dữ liệu thử nghiệm được các mô hình nhìn thấy tại thời điểm đào tạo (test leekage). Trong FB15k, vấn đề này là do sự hiện diện của các quan hệ gần giống nhau hoặc nghịch đảo của nhau.FB15k-237 được xây dựng để trở thành một tập dữ liệu thách thức hơn: các tác giả đã chọn các dữ kiện liên quan đến 401 quan hệ xuất hiện nhiều nhất và loại bỏ tất cả các quan hệ tương đương hoặc nghịch đảo. Họ cũng đảm bảo rằng không có thực thể nào được kết nối trong tập huấn luyện cũng được liên kết trực tiếp trong tập test và validation.

%Tập training gồm 237 relations, 13781 head entities, và 13379 tail entities, tập test gồm 223 relations, 7652 head entities, và 5804 tail entities, tập vadition gồm 224 relations, 8171 head entities, and 6376 tail entities.

\subsection{Bộ dữ liệu WN18}



\begin{center}
	\resizebox{\textwidth}{!}{%
		\begin{tikzpicture}
			[every axis/.style={
				ybar,
				scale only axis,
				width=\textwidth,
				height=0.4\textwidth,
				xtick=data,
				x tick label style={rotate=45, anchor=east},
				legend style={at={(20em,5em)}, anchor=east},
				bar width=1em,
				scaled y ticks=false,
				font=\scriptsize\sffamily,
				symbolic x coords={
					also\_see,
					derivationally\_related\_form,
					has\_part,
					hypernym,
					hyponym,
					instance\_hypernym,
					instance\_hyponym,
					member\_holonym,
					member\_meronym,
					member\_of\_domain\_region,
					member\_of\_domain\_topic,
					member\_of\_domain\_usage,
					part\_of,
					similar\_to,
					synset\_domain\_region\_of,
					synset\_domain\_topic\_of,
					synset\_domain\_usage\_of,
					verb\_group},
				nodes near coords,
				nodes near coords align={vertical},
			}]
			\pgfplotsset{
				compat=newest,
				major grid style=blue,
				xlabel near ticks,
				ylabel near ticks
			}
			
			\begin{axis}[]
				\addplot [fill=blue] coordinates {
					(also\_see,1299)
					(derivationally\_related\_form,29715)
					(has\_part,4816)
					(hypernym,34796)
					(hyponym,34832)
					(instance\_hypernym,2921)
					(instance\_hyponym,2935)
					(member\_holonym,7382)
					(member\_meronym,7402)
					(member\_of\_domain\_region,923)
					(member\_of\_domain\_topic,3118)
					(member\_of\_domain\_usage,629)
					(part\_of,4805)
					(similar\_to,80)
					(synset\_domain\_region\_of,903)
					(synset\_domain\_topic\_of,3116)
					(synset\_domain\_usage\_of,632)
					(verb\_group,1138)
				};
			\end{axis}
		\end{tikzpicture}
	}
\end{center}


Bộ dữ liệu này được giới thiệu bởi các tác giả của TransE \cite{bordes2013translating}, được trích xuất từ WordNet\footnote{https://wordnet.princeton.edu/}, một bản thể học ngôn ngữ KG có nghĩa là cung cấp một từ điển/từ đồng nghĩa để hỗ trợ NLP và phân tích văn bản tự động. Trong WordNet, các thực thể tương ứng với các tập hợp (\textit{word senses}) và các quan hệ đại diện cho các kết nối từ vựng của chúng (ví dụ: “hypernym”). Để xây dựng WN18, các tác giả đã sử dụng WordNet làm điểm bắt đầu và sau đó lặp đi lặp lại lọc ra các thực thể và mối quan hệ với quá ít lần được đề cập. 

%Tập dữ liệu được chia ngẫu nhiên thành 3 tập: tập training với 18 relations, 40504 head entities, và 40551 tail entities, tập test gồm 18 relations, 4262 head entities, and 4338 tail entities, tập vadiation gồm 18 relations, 4349 head entities, and 4263 tail entities.

\subsection{Bộ dữ liệu WN18RR}

Bộ dữ liệu này là một tập hợp con của WN18 được xây dựng bởi DeŠmers et al.\cite{dettmers2017convolutional}, cũng là người giải quyết vấn đề rò rỉ thử nghiệm (test leakage)trong WN18. Để giải quyết vấn đề đó, họ xây dựng tập dữ liệu WN18RR thách thức hơn nhiều bằng cách áp dụng một phương pháp tương tự được sử dụng cho FB15k-237 \cite{toutanova2015observed}. 

%Training gồm 11 relations, 39610 head entities, và 31881 tail entities, tập test gồm 11 relations, 2958 head entities, và 2619 tail entities, tập vadition gồm 11 relations, 2851 head entities, and 2575 tail entities.

\section{Các độ đo}
Trong phần này chúng tôi mô tả lại các phương pháp đánh giá (độ do), môi trường thực hiện cũng như các tập dữ liệu mà chúng tôi sử dụng để dánh giá phương pháp của mình. Các phương pháp đánh giá (độ do) này cũng phổ biến nó được đánh giá cho hầu hết các mô hình dự đoán liên kết trên đồ thị. Chúng tôi tiến hành so sánh với bốn phương pháp nổi bật khác được báo cáo trong \cite{rossi2020knowledge}.

\subsubsection{Độ đo Hit@K (H@K)}

Đó là tỷ lệ các dự đoán đúng mà rank nhỏ hơn hoặc bằng ngưỡng \(K\):
\[H@K = \frac{\mid {q ~\in ~Q~: rank(q) \leq K} \mid}{\mid Q \mid}\]

\subsubsection{Mean Rank (MR)}

Đây là giá trị trung bình của rank thu được cho một dự đoán chính xác. Càng nhỏ thì mô hình càng chính xác:
\[MR = \frac{1}{\mid Q \mid} \sum_{q ~\in~ Q} rank(q) \]
Trong đó \(\mid Q \mid\) là độ lớn của tập hợp các câu hỏi bằng độ lớn của tập test hoặc vadidation. Khi dự đoán chúng tôi dự đoán cả head và tail cho một dòng tương ứng trong tập dữ liệu thử nghiệm. Ví dụ chún tôi sẽ dự đoán \(\langle ?,~ relation,~ tail \rangle\) và \(\langle head,~ relation,~ ?\rangle\) cho 1 dòng tương ứng, \(q\) thể hiện cho câu hỏi chúng tôi dự đoán và \(rank(q)\) thể hiện cho kết quả đúng của câu hỏi đứng ở vị trí thứ mấy trong xếp hạng của chúng tôi sau đó lấy trung bình rank của các dự đoán head và tail. Rõ ràng độ đo này nằm giữa \([1, \mid \text{số lượng các entity} \mid]\) do có tối da \(n\) cạnh nối 1 đỉnh tới \(n-1\) đỉnh còn lại và thêm cạnh nối tới chính đỉnh nó(cạnh khuyên). Và độ đo này đễ bị ảnh hưởng bởi nhiễu vì có những quan hệ có những thực thể được xếp hạng gần cuối. Để giải quyết vấn đề này nhóm chúng tôi và các nhóm nghiên cứu khác sử dụng thêm độ đo Mean Reciprocal Rank (MMR)

\subsubsection{Mean Reciprocal Rank (MMR)}

Đây là xếp hạng đối ứng trung bình, là nghịch đảo của giá trị trung bình của rank thu được cho một dự đoán chính xác ở trên. Và càng lớn thì mô hình càng chính xác. Do độ đo này lấy nghịch đảo của các rank nên tránh dược vấn đề nhiễu của độ đo MR ở trên.
\[MRR =\frac{1}{\mid Q \mid} \sum_{q~ \in ~Q} \frac{1}{rank(q)}\]

\section{Phương pháp huấn luyện}

%\subsection{Huấn luyện trên mô hình AnyBURL}


\subsection{Huấn luyện trên mô hình KBGAT}

Đầu tiên chúng tôi khởi tạo các vector nhúng bằng mô hình TransE (\cite{bordes2013translating}). Để tạo ra các bộ ba không hợp lệ, chúng tôi thay thế các thực thể đầu và thực thể đuôi bằng một thực thể khác được lấy ngẫu nhiên trong tập thực thể .

Sau đó chúng tôi chia ra làm hai phần huấn luyện, phần đầu tiên được xem như mã hóa (encoder) giúp biến đổi các vector nhúng khởi tạo ban đầu thành các vector nhúng mới tổng hợp thông tin các nút lân cận bằng mô hình KBGAT để tạo ra các vector nhúng của thực thể và quan hệ. Phần thứ hai được xem như quá trình giải mã (decoder) để thực hiện nhiệm vụ dự đoán, bằng cách lấy thêm thông tin của n-hop giúp chúng tôi tổng hợp thêm thông tin từ các thực thể lân cận, ngoài ra chúng tôi còn sử dụng quan hệ phụ trợ để tổng hợp thêm thông tin hàng xóm trong đồ thị thưa. Chúng tôi sử dụng hàm tối ưu Adam với tốc độ học $\mu = 0.001$. Số chiều cuối cùng của cả thực thể và quan hệ đều bằng 200. Cụ thể các siêu tham số ưu được tìm kiếm bằng thuật toán tìm kiếm lưới (grid search) được trình bày ở \autoref{appendix:Appendix1}.

\section{Kết quả thực nghiệm}
\label{sec:Experiment}

Như đã nói trước đây với mô hình dựa trên luật của chúng tôi hoàn toàn có thể thực hiện trên một laptop với cấu hình thông thường. Trong thí nghiệm của chúng tôi cấu hình máy để thực thi như sau: T480, core i5 8th Gen, ram 16GB, 4 core 8 thread. Mã nguồn thực thi được viết bằng ngôn ngữ Python phiên bản 3.6 và dùng các hàm hỗ trợ có sẵn trong Python với không một thư viện bên thứ ba nào. Thí nghiệm được thực hiện với bốn tập dữ liệu phổ biến là FB15k, FB15-237, WN18 và WN18RR. Thông tin chi tiết các bộ dữ liệu này được mô tả ở \autoref{sec:DataTraining} các tập dữ liệu huấn luyện.





Như mô tả ở phần \autoref{alg:GenerateRules}, thuật toán AnyBURL này sẽ học các luật được sinh ra trong một khoảng thời gian nhất định do người dùng cấu hình. Ở đây chúng tôi chọn cấu hình thời gian là 1000 giây tương đương khoảng 17 phút đào tạo, với độ bão hòa (SAT) \(0.85\), độ tin cậy Q \(0.05\), kích thước mẫu S (\(\frac{1}{10}~ \text{tập huấn luyện}\)). Với cấu hình như vậy mô hình phiên bản Python của chúng tối cho kết quả tương đương với phiên bản Java nhóm tác giả Meilicke, Christian et al. \cite{burl} với cấu hình tương tự nhưng thời gian training là 100 giây. Sự khác biệt về thời gian học tập ở đây chủ yếu là do hiệu năng của hai ngôn ngữ Python và Java. Ở đây chúng tôi chọn ngôn ngữ Python vì nó được dùng làm ngôn ngữ chính cho nhiều mô hình trí tuệ nhân tạo gần đây, và cũng thuận tiện cho chúng tôi khi so sánh hiệu năng cũng như đánh giá với các phương pháp học sâu khác đa số được viết bằng Python.


\begin{table}[H]
	\begin{center}
		\caption{Kết quả thực nghiệm trên tập FB15k, FB15k-237}
		\label{tab:resultOnFreeBase}%
		\resizebox{0.9\textwidth}{!}{%
			\begin{tabular}{l|l|l|l|l|l|l|l|l|}
				\cline{2-9}
				& \multicolumn{4}{c|}{\textbf{FB15k}}                   & \multicolumn{4}{c|}{\textbf{FB15k-237}}                   \\ \cline{2-9} 
				& \textbf{H@1} & \textbf{H@10} & \textbf{MR} & \textbf{MRR} & \textbf{H@1} & \textbf{H@10} & \textbf{MR} & \textbf{MRR} \\ \hline
				\multicolumn{1}{|l|}{ComplEx} & 81.56        & 90.53         & 34          & 0.848        & 25.72        & 52.97         & 202        & 0.349        \\ \hline
				\multicolumn{1}{|l|}{TuckER}  & 72.89        & 88.88         & 39          & 0.788        & 25.90        & 53.61         & 162         & 0.352        \\ \hline
				\multicolumn{1}{|l|}{TransE}  & 49.36        & 84.73         & 45          & 0.628        & 21.72        & 49.65         & 209         & 0.31        \\ \hline
				\multicolumn{1}{|l|}{RoteE}   & 73.93        & 88.10         & 42          & 0.791        & 23.83        & 53.06         & 178         & 0.336        \\ \hline
				\multicolumn{1}{|l|}{ConvKB}  & 59.46        & 84.94         & 51         & 0.688        & 21.90        & 47.62         & 281         &0.305        \\ \hline
				\multicolumn{1}{|l|}{\textbf{KBGAT}}     &  70.08            &     91.64    &  38    &   0.784    & 36.06     &    58.32   &  211  &    0.4353  \\ \hline
				\multicolumn{1}{|l|}{\textbf{AnyBURL}}    & 79.13        & 82.30         & 285         & \underline{0.824}        & 20.85        & 42.40         & 490         & 0.311        \\ \hline
			\end{tabular}
		}
	\end{center}
\end{table}


\begin{table}[H]
	\begin{center}
		\caption{Kết quả thực nghiệm trên tập WN18, WN18RR}
		\label{tab:resultOnWordNet}%
		\resizebox{0.9\textwidth}{!}{%
			\begin{tabular}{l|l|l|l|l|l|l|l|l|}
				\cline{2-9}
				& \multicolumn{4}{c|}{\textbf{WN18}}                              & \multicolumn{4}{c|}{\textbf{WN18RR}}                            \\ \cline{2-9} 
				& \textbf{H@1}   & \textbf{H@10}  & \textbf{MR}  & \textbf{MRR}   & \textbf{H@1}   & \textbf{H@10}  & \textbf{MR}  & \textbf{MRR}   \\ \hline
				\multicolumn{1}{|l|}{ComplEx} & 94.53          & 95.50          & 3623         & 0.349          & 42.55          & 52.12          & 4909         & 0.458          \\ \hline
				\multicolumn{1}{|l|}{TuckER}  & 94.64          & 95.80          & 510          & 0.951          & 42.95          & 51.40          & 6239         & 0.459          \\ \hline
				\multicolumn{1}{|l|}{TransE}  & 40.56          & 94.87          & 279          & 0.646          & 2.79           & 94.87          & 279          & 0.646          \\ \hline
				\multicolumn{1}{|l|}{RoteE}   & 94.30          & 96.02          & 274          & 0.949          & 42.60          & 57.35          & 3318         & 0.475          \\ \hline
				\multicolumn{1}{|l|}{ConvKB}  & 93.89          & 95.68          & 413          & 0.945          & 38.99          & 50.75          & 4944         & 0.427          \\ \hline
				\multicolumn{1}{|l|}{\textbf{KBGAT}}     &                &        &        &                &       35.12         &        57.01         &      \underline{1974}       &  0.4301           \\ \hline
				\multicolumn{1}{|l|}{\textbf{AnyBURL}}    &  93.96 & 95.07 & \textbf{230} & \textbf{0.955} & \textbf{44.22} & 54.40 & 2533 & \underline{0.497} \\ \hline
			\end{tabular}
		}
	\end{center}
\end{table}


\autoref{tab:resultOnFreeBase}, và \autoref{tab:resultOnWordNet} mô tả các kết quả thực nghiệm của chúng tôi với các độ đo \(H@K\) cùng với các kết quả thực nghiệm của các phương pháp khác được đề cập trong khảo  sát \cite{rossi2020knowledge}

\begin{table}[H]
	\begin{center}
		\caption{Kết quả độ chính xác hai chiến lược thêm tri thức mới}
		\label{tab:CompareAccuracy}%
		\resizebox{0.8\columnwidth}{!}{%
			\begin{tabular}{ll|l|l|l|}
				\cline{3-5}
				&        & \textbf{AnyBURL} & \textbf{Batch edge AnyBURL} & \textbf{Edge AnyBURL} \\ \hline
				\multicolumn{1}{|l|}{\multirow{3}{*}{\textbf{FB-15k}}}    & hit@10 & 82.22                  & 82.48               & 83.08               \\ \cline{2-5} 
				\multicolumn{1}{|l|}{}                                    & MR     & 285                    & 250                 & 220                 \\ \cline{2-5} 
				\multicolumn{1}{|l|}{}                                    & MRR    & 0.824                  & 0.853               & 0.866               \\ \hline
				\multicolumn{1}{|l|}{\multirow{3}{*}{\textbf{FB15k-237}}} & hit@10 & 42.40                  & 43.40               & 43.51               \\ \cline{2-5} 
				\multicolumn{1}{|l|}{}                                    & MR     & 490                    & 472                 & 441                 \\ \cline{2-5} 
				\multicolumn{1}{|l|}{}                                    & MRR    & 0.311                  & 0.353               & 0.377               \\ \hline
				\multicolumn{1}{|l|}{\multirow{3}{*}{\textbf{WN18}}}      & hit@10 & 95.07                  & 95.09               & 95.19               \\ \cline{2-5} 
				\multicolumn{1}{|l|}{}                                    & MR     & 230                    & 229                 & 228                 \\ \cline{2-5} 
				\multicolumn{1}{|l|}{}                                    & MRR    & 0.955                  & 0.955               & 0.956               \\ \hline
				\multicolumn{1}{|l|}{\multirow{3}{*}{\textbf{WN18RR}}}    & hit@10 & 54.40                  & 54.63               & 54.70               \\ \cline{2-5} 
				\multicolumn{1}{|l|}{}                                    & MR     & 2533                   & 2346                & 2215                \\ \cline{2-5} 
				\multicolumn{1}{|l|}{}                                    & MRR    & 0.497                  & 0.553               & 0.581               \\ \hline
			\end{tabular}
		}
	\end{center}
\end{table}

\autoref{tab:CompareAccuracy} mô tả các kết quả thực nghiệm của chúng tôi với hai chiến lược thêm tri thức mới vào đồ thị. Chúng tôi đánh giá trên tổng số luật được sinh ra, và số luật có độ tin cậy \(>= 50\%\) và \(>= 80\%\).

\begin{table}[H]
	\begin{center}
		\caption{Kết quả đánh giá về số luật hai chiến lược thêm tri thức mới}
		\label{tab:CompareRule}%
		\resizebox{0.8\columnwidth}{!}{%
			\begin{tabular}{ll|l|l|}
				\cline{3-4}
				& & \textbf{Batch edge AnyBURL} & \textbf{Edge AnyBURL}  \\ \hline
				\multicolumn{1}{|c|}{\multirow{3}{*}{\textbf{FB15k}}}     & num rule        & 1011                       & 1367                      \\ \cline{2-4} 
				\multicolumn{1}{|c|}{}& confidence 50\% & 416 (41,14\%)              & 1185 (86,69\%)            \\ \cline{2-4} 
				\multicolumn{1}{|c|}{}& confidence 80\% & 284 (28, 09\%)             & 481 (35,18\%)             \\ \hline
				\multicolumn{1}{|l|}{\multirow{3}{*}{\textbf{FB15k-237}}} & num rule        & 1120                       & 756                       \\ \cline{2-4} 
				\multicolumn{1}{|l|}{}& confidence 50\% & 244 (21,79\%)              & 660 (87,30\%)             \\ \cline{2-4} 
				\multicolumn{1}{|l|}{}& confidence 80\% & 95 (8,48\%)                & 162 (21,43\%)             \\ \hline
				\multicolumn{1}{|l|}{\multirow{3}{*}{\textbf{WN18}}}      & num rule        & 533                        & 260                       \\ \cline{2-4} 
				\multicolumn{1}{|l|}{}& confidence 50\% & 270 (38, 46 \%)            & 252 (96,92\%)             \\ \cline{2-4} 
				\multicolumn{1}{|l|}{}& confidence 80\% & 240 (34,19\%)              & 225 (86,54\%)             \\ \hline
				\multicolumn{1}{|l|}{\multirow{3}{*}{\textbf{WN18RR}}}    & num rule        & 439                        & 106                       \\ \cline{2-4} 
				\multicolumn{1}{|l|}{}& confidence 50\% & 110 (25,05\%)              & 102 (96,22\%)             \\ \cline{2-4} 
				\multicolumn{1}{|l|}{}& confidence 80\% & 83 (18,91\%)               & 85 (81,19\%)              \\ \hline
			\end{tabular}
		}
	\end{center}
\end{table}


%%\chapter{RESULTS AND EVALUATION}
\label{chap:evalution}

\section{Evaluation Methods}

The evaluation process is conducted through two main metrics: Mean Opinion Scores (MOS) and Fréchet Inception Distance (FID).

\subsection{Evaluation Based on Human Perception}

\subsubsection{Mean Opinion Scores (MOS)}

Currently, there is no standard metric for gesture generation, especially for gesture generation from speech. Therefore, this thesis relies on subjective human evaluation to conduct experimental assessments. 
Similar to previous methods \cite{yoon2022genea}, \cite{kucherenko2021large}, speech-driven gesture generation models still lack objective metrics that consistently reflect human subjective perception \cite{alexanderson2022listen}.

MOS is measured through three criteria:

\begin{itemize}
	\item Human-likeness
	\item Gesture-Speech Appropriateness
	\item Gesture-Style Appropriateness
\end{itemize}

One of the contributions of this thesis is the development of the \hyperlink{https://genea-workshop.github.io/leaderboard/}{GENEA Leaderboard} \footnote{ \url{https://genea-workshop.github.io/leaderboard} } \cite{nagy2024towards}. This system includes HEMVIP (\textbf{H}uman \textbf{E}valuation of \textbf{M}ultiple \textbf{V}ideos in \textbf{P}arallel), which is used to compare visual generation results between videos rendered by different models.

\begin{figure}[H]
	\centering
	\includegraphics[width=\textwidth]{hemvip}
	\caption{HEMVIP system used to evaluate rendering results of two models}
\end{figure}

Within the GENEA group (\textbf{G}eneration and \textbf{E}valuation of \textbf{N}on-verbal Behaviour for \textbf{E}mbodied \textbf{A}gents), we hire evaluators on Prolific and conduct a user study based on the rendered video results. Participants rate the results as \textit{Left Better}, \textit{Equal}, or \textit{Right Better}. The updated scores are $-1$, $0$, and $1$ for each model, including ground truth data. The comparative results for all models are updated using the Elo rating system.

The source code of the program is available at \hyperlink{https://github.com/hemvip/hemvip.github.io/}{github.com/hemvip.github.io}
\footnote{HEMVIP 2 \url{https://github.com/hemvip/hemvip.github.io}}.

\subsection{Evaluation Based on Quantitative Metrics}

\subsubsection{Mean Square Error (MSE)}

The mean square error between the predicted gesture sequence $\hat{\mathbf{y}}_i^{1:M \times D}$ and the ground-truth gesture sequence $\mathbf{y}_i^{1:M \times D}$ is computed as follows:

\begin{equation}
\text{MSE} = \frac{1}{n} \sum_{i=1}^n \left\| \mathbf{y}_i^{1:M \times D} - \hat{\mathbf{y}}_i^{1:M \times D} \right\|^2
\end{equation}

Where:
\begin{itemize}
	\item $n$ is the number of data samples.
	\item $\mathbf{y}_i^{1:M \times D}$ is the ground truth of the $i^{th}$ sample, with $M$ being the number of frames and $D$ the number of data dimensions.
	\item $\hat{\mathbf{y}}_i^{1:M \times D}$ is the predicted value of the $i^{th}$ sample, having the same size $M \times D$.
	\item $\left\| \mathbf{y}_i^{1:M \times D} - \hat{\mathbf{y}}_i^{1:M \times D} \right\|^2$ is the squared norm of the difference between the ground truth and the predicted matrix.
\end{itemize}

MSE measures the average squared difference between the actual and predicted gesture sequences. The smaller the value, the more accurate the model's predictions. Evaluation results are presented in \autoref{subsec:MSEResult}.

\subsubsection{Fréchet Gesture Distance (FGD)}

Similar to image generation methods that use the Fréchet Inception Distance (FID) to measure distributional differences between real and generated data, the Fréchet Gesture Distance (FGD) measures the similarity in distribution between generated gesture sequences $\hat{\mathbf{y}}_i^{1:M \times D}$ and real gesture sequences $\mathbf{y}_i^{1:M \times D}$:

\begin{equation}
	\text{FGD} = \left\| \hat{\mu} - \mu \right\|^2 + \operatorname{Tr}\left( \Sigma + \hat{\Sigma} - 2 \sqrt{\Sigma \hat{\Sigma}} \right)
	\label{eq:fidscore}
\end{equation}

Where $n$ is the number of samples, and the parameters are defined as:

\begin{itemize}
	\item $\mu = \frac{1}{n} \sum_{i=1}^n \mathbf{y}_i^{1:M \times D}$ and $\hat{\mu} = \frac{1}{n} \sum_{i=1}^n \hat{\mathbf{y}}_i^{1:M \times D}$ are the mean feature vectors of the real dataset $\mathbf{y}_i^{1:M \times D}$ and generated dataset $\hat{\mathbf{y}}_i^{1:M \times D}$, respectively.
	 
	\item $\Sigma = \frac{1}{n-1} \sum_{i=1}^n \left( \mathbf{y}_i^{1:M \times D} - \mu \right) \left( \mathbf{y}_i^{1:M \times D} - \mu \right)^T$ and
	
	$\hat{\Sigma} = \frac{1}{n-1} \sum_{i=1}^n \left( \hat{\mathbf{y}}_i^{1:M \times D} - \hat{\mu} \right) \left( \hat{\mathbf{y}}_i^{1:M \times D} - \hat{\mu} \right)^T$ are the covariance matrices of the features from the real and generated datasets, respectively.
	
	\item $\operatorname{Tr}(\cdot)$ denotes the matrix trace operator, which sums the diagonal elements.
	
	\item $\sqrt{\Sigma \hat{\Sigma}}$ denotes the matrix square root of the product of the two covariance matrices.
\end{itemize}

A low FGD score indicates that the distribution of generated gestures is close to the real gestures, while a high FGD suggests a large distributional difference and therefore lower gesture generation quality. In this thesis, the evaluation is performed on the predicted gesture sequence $\hat{\mathbf{x}}^{0} \in \mathbb{R}^{1:M \times D}$ and the ground-truth gesture sequence $\mathbf{x}_{0} \in \mathbb{R}^{1:M \times D}$.

\section{Evaluation Results}
\label{sec:result}

\subsection{User Study Evaluation Results}

\subsubsection{MOS Evaluation Results}

This thesis reuses the evaluation results of the baseline model \textbf{DiffuseStyleGesture} \cite{yang2023diffusestylegesture} for human perception metrics, as gesture generation remains a nascent field and the cost of evaluating multiple models is high. Therefore, this thesis does not include evaluation results for the proposed \textbf{OHGesture} model.

\begin{figure}[htbp]
	\centering
	\begin{subfigure}[b]{0.3\textwidth}
		\includegraphics[width=\textwidth]{BoxHumanLikeness.pdf}
		\caption*{(a) Human-likeness}
	\end{subfigure}
	\hfill
	\begin{subfigure}[b]{0.3\textwidth}
		\includegraphics[width=\textwidth]{BoxSpeechAppropriateness.pdf}
		\caption*{\small (b) Speech Appropriateness}
	\end{subfigure}
	\hfill
	\begin{subfigure}[b]{0.3\textwidth}
		\includegraphics[width=\textwidth]{BoxStyleAppropriateness.pdf}
		\caption*{(c) Style Appropriateness}
	\end{subfigure}
	
	\label{fig:compare }
\end{figure}

To understand the visual performance of the proposed method, this thesis conducts a user study comparing gestures generated by the proposed method with those from real motion capture data. The duration of the evaluated video clips ranges from 11 to 51 seconds, with an average length of 31.6 seconds — longer than clips used in the GENEA evaluation \cite{yoon2022genea} (8–10 seconds), as longer durations can provide clearer and more convincing results \cite{yang2022reprgesture}. Participants rated each video on a scale from 5 to 1, labeled $\texttt{excellent}$, $\texttt{good}$, $\texttt{fair}$, $\texttt{poor}$, and $\texttt{bad}$. 

\begin{table}[H]
	\centering
	\begin{tabular}{lcc}
		\hline
		\multicolumn{1}{c}{Name} &
		\begin{tabular}[c]{@{}c@{}}Human\\ likeness \end{tabular}$\uparrow$ &
		\begin{tabular}[c]{@{}c@{}}Gesture-speech\\ appropriateness\end{tabular}$\uparrow$ \\ \hline
		Ground Truth          & 4.15 $\pm$ 0.11          & 4.25 $\pm$ 0.09          \\
		Ours                  & \textbf{4.11 $\pm$ 0.08} & \textbf{4.11 $\pm$ 0.10} \\
		\quad$-$ WavLM             & 4.05 $\pm$ 0.10          & 3.91 $\pm$ 0.11          \\
		\quad$-$ Cross-local attention   & 3.76 $\pm$ 0.09          & 3.51 $\pm$ 0.15          \\
		\quad$-$ Self-attention    & 3.55 $\pm$ 0.13          & 3.08 $\pm$ 0.10          \\
		\quad$-$ Attention + GRU&
		3.10 $\pm$ 0.11 &
		2.98 $\pm$ 0.14 \\
		\quad$+$ Forward attention & 3.75 $\pm$ 0.15          & 3.23 $\pm$ 0.24          \\
		\hline
	\end{tabular}
	\caption{Evaluation results using MOS}
	\label{table:MOSScore}
\end{table}
% Results of the ablation studies. "$-$" indicates removed modules, "$+$" indicates added modules. Bold indicates best performance.

\subsection{Quantitative Evaluation Results}

\subsubsection{Evaluation Results using MSE}
\label{subsec:MSEResult}

In this thesis, the predicted gesture sequence is segmented over $M$ frames. Mean Square Error (MSE) is applied to the gesture sequence $\mathbf{x}^{1:M \times D}$.

\begin{table}[H]
	\centering
	\resizebox{\textwidth}{!}{%
		\begin{tabular}{lcccccc}
			\hline
			\multicolumn{1}{c}{Emotion} & Neutral & Sad & Happy & Relaxed & Elderly & Angry \\ \hline
			DiffuseStyleGesture  & 75.04 & 51.40 & 110.18 & 130.83     & 116.03    & 78.53     \\
			ZeroEGG & 136.33 & 81.22 & 290.47 & 140.24     & 102.44    & 181.07     \\
			\hline
			Proposed Model                     &         &         &         &           &          &                 \\
			\quad \textbf{OHGesture} & 161.22 & 89.58 & 279.95 & 156.93   & 99.86   & 215.24    \\
			\hline
		\end{tabular}%
	}
	\caption{Mean Square Error results across 6 emotion categories}
	\label{table:EvaluationMSE}
\end{table}

\subsubsection{Evaluation Results using FGD}

This thesis proposes Fréchet Gesture Distance (FGD), a gesture-based variant of FID, and develops the open-source tool \hyperlink{https://github.com/GestureScore/GestureScore}{GestureScore} \footnote{Github/GestureScore: \url{https://github.com/GestureScore/GestureScore}}. In GestureScore, an Inception V3 model is implemented to encode the frame sequence $\bx^{1:M \times D}$ into a latent feature vector of size $32 \times 32$, which is then used as input to \autoref{eq:fidscore}. The following \autoref{table:EvalFGD} shows the FGD evaluation results of the OHGesture model using GestureScore.

\begin{table}[H]
	\centering
	\begin{tabular}{lcc}
		\hline
		\multicolumn{1}{c}{Name} & FGD on Feature Vectors & \begin{tabular}[c]{@{}c@{}} FGD on Raw Data \end{tabular} \\ \hline
		Ground Truth             & -       & -          \\
		Ours                     &       & \\
		\quad OHGesture (Feature D=1141) & 2.058      & 9465.546 \\
		\quad OHGesture (Rotations) & 3.513       & 9519.129 \\
		\hline
	\end{tabular}
	\caption{Evaluation results of Fréchet Gesture Distance (FGD) on $\bx^{1:M \times D}$ (from frame 1 to frame M, with D features per frame)}
	\label{table:EvalFGD}
\end{table}

\begin{itemize}[]
	\item \textbf{Feature Vectors:} The BVH files are used to convert the entire skeleton of each frame into a feature vector of size $D = 1141$, as described in \autoref{eq:gesturevector}.
	
	\item \textbf{Rotations:} From the resulting BVH file, this thesis extracts the rotation angles, $D = 225$ ($225 = 75 \times 3$), from the gesture sequence of length $M$ frames for evaluation in the OHGesture row below.
\end{itemize}

\section{Building and Standardizing a Gesture Generation Evaluation System}

Currently, gesture generation is an active research area with many different models. However, there is no shared evaluation metric. Traditional metrics such as FID (Fréchet Inception Distance) or IS (Inception Score) fail to capture the human-likeness, speech appropriateness, and style appropriateness of generated gestures. Moreover, models are trained and evaluated on different datasets, making it difficult to compare results and determine which model is superior or state-of-the-art. This lack of a standardized evaluation protocol hampers progress in gesture generation research.

To address this, the thesis proposes building an online leaderboard system \cite{nagy2024towards} \hyperlink{https://genea-workshop.github.io/leaderboard/}{GENEA Leaderboard} \footnote{GENEA Leaderboard: \url{https://genea-workshop.github.io/leaderboard/}}, which ranks gesture generation models. The thesis collects and processes gesture data from multiple languages and datasets, standardizes them into a unified dataset, and invites authors of various models to train and infer on this standardized set. The resulting generated gestures are then evaluated by hired participants through Prolific. 

The thesis is also building an online system \hyperlink{https://github.com/hemvip/hemvip.github.io}{hemvip/hemvip.github.io} \footnote{HEMVIP2 \url{https://github.com/hemvip/hemvip.github.io}} to support the evaluation process via Prolific-based crowd-sourcing. Evaluation results for the OHGesture model will be added based on this system.

Through this evaluation system, the research aims to establish a common benchmark, thereby fostering advancement in gesture generation.


\chapter{CONCLUSION}
%\chapter{Kết luận}
\label{chap:Conclusion}

%Trong phần này chúng tôi sẽ trình bày các kết quả đạt được của mô hình chúng tôi, cũng như những phân tích của chúng tôi
%trên các kết quả của các tập dữ liệu khác nhau để giải thích những điểm tốt và điểm cần cải thiện trên mô hình của chúng tôi trên tập dữ liệu đó. Từ đó chúng tôi xác định những hướng nghiên cứu để cải tiến trong tương lai.
%% chúng tôi cố gắng tìm hiểu các đặc trưng của các bộ dữ liệu tương ứng để cố gắng lý  giải thích  tại sao mô hình của chúng tôi hoặc các công trình khác có được kết quả tốt trên tập dữ liệu tương ứng đó.
%% Những kết quả của hai đề xuất của chúng tôi cũng như các dịnh hướng nghiên cứu của chúng tôi trong tương lai.
%
%Mặc dù kết quả chúng tôi cho thấy phương pháp dựa trên luật của chúng tôi có hiệu suất tương đương với các mô hình học sâu hiện đại (state-of-art) và có ưu thế vượt trội trong thời gian đào tạo khoảng 17 phút so với thời gian hàng giờ của phương pháp học sâu khác nhưng không phải là các mô hình học sâu này không đáng nghiên cứu. Chúng tôi cũng nhận thấy, với tập dữ liệu có nhiều loại quan hệ khác nhau như FreeBase, mô hình KBGAT nhờ sử dụng cơ chế chú ý đạt được kết quả tốt hơn so với tập WorldNet với số lượng các loại quan hệ ít hơn.


In this section, we presented the results achieved by the proposed model, along with detailed analyses on different datasets to clarify both the strengths and the remaining limitations. From this, we identified potential research directions for improving the model in the future.

Although our rule-based method demonstrates performance comparable to modern deep learning models (state-of-the-art), and clearly outperforms them in terms of training time—only about 17 minutes compared to several hours for deep learning models—this does not imply that deep learning models are not worth studying. On the contrary, through performance analysis across different datasets, we observed that for datasets with diverse relations like FreeBase, the KBGAT model using attention mechanisms yielded significantly better results than on datasets like WordNet, which contain fewer relation types. This highlights the potential of leveraging deep learning mechanisms tailored to the specific characteristics of each dataset.

This shows that the attention mechanism, by incorporating relational embedding information, helps to better capture graph structures in datasets with a wide variety of relations.  
For datasets with many similar and inverse triples such as FB15k and WN18RR, the rule-based model AnyBURL achieved superior results, whereas deep learning methods only achieved average performance compared to other methods.  
The rule-based AnyBURL model performs better on datasets like FB15k and WN18RR; however, for datasets that have removed similar or inverse information, such as FB15k-237 and WN18RR, the rule-based method is less effective, since it relies on previously observed paths or links. In contrast, deep learning models represent relations and entities in a vector space to learn their interactions, allowing them to perform better on datasets like FB15k-237 and WN18RR than on FB15k and WN18.

One of the main advantages of the rule-based approach is that the generated rules are interpretable during training and require significantly less training time compared to other methods. However, after the training phase, the rule-based method must iterate through all learned rules to make predictions. This is an area where deep learning models show better performance, as models like KBGAT can use the learned weights and computational layers to transform inputs into probabilistic predictions much faster. The drawback of deep learning approaches is their lack of interpretability during training, as well as the high computational cost. Regarding our two proposed algorithms for adding new knowledge to the graph, we found them to significantly outperform deep learning methods.




% Ngược lại đối với các phương pháp dựa trên học sâu lại có ưu thế rất lớn trong các tập dữ liệu này do có thể dễ dàng tính toán độ gần của các luật mới cần đánh giá so với các luật đã học từ đó có một kết quả khá tốt.
% Do đó chúng tôi cũng sẽ tiếp tục nghiên cứu các phương pháp học sâu và sẽ dùng phương pháp này làm đường cơ sở (base line) để so sánh với các nghiên cứu của chúng tôi trong tương lai.
% Một điểm yếu nữa của mô hình đựa trên luật của chúng tôi là mặc dù thời gian học là vượt trội nhưng thời gian để tính toán đưa ra đự đoán khá lâu do phải duyệt qua tất cả các luật được sinh ra mới có thể đưa ra dự đoán.
% Không giống như các phương pháp nhúng đồ thị khác thao tác này có thể dễ dàng tính toán.


The graph embedding process helps represent the features of entities, relations, or the characteristics of the knowledge graph as lower-dimensional vectors (\autoref{sec:graphEmbedding}). However, in practice, a piece of knowledge represented by entities and relations is entirely independent between these components; thus, they should be embedded into vectors with different dimensionalities. The ratio of dimensions between entities and relations is also an important issue that requires further investigation. 

Additionally, in the real world, the temporal factor is a critical piece of information that can completely alter the meaning of a piece of knowledge. Therefore, integrating temporal information into the attention mechanism is one of the research directions we aim to pursue to ensure the semantic accuracy of the knowledge graph.

For the rule-based method AnyBURL, the reinforcement learning branch has recently seen significant progress, and the authors Meilicke, Christian and Chekol \cite{meilicke2020reinforced} have recently proposed a study to optimize the AnyBURL method using reinforcement learning. We also intend to explore this direction and aim to report our findings in the near future.


\pagebreak 
% TÀI LIỆU TRÍCH DẪN
% REFERENCES
\renewcommand{\bibname}{TÀI LIỆU TRÍCH DẪN} % Đổi tên phần tiêu đề của tài liệu tham khảo
%
\bibliographystyle{plain}
%\bibliographystyle{custombibstyle}
\bibliography{References/references}



%\renewcommand{\bibname}{DANH MỤC CÔNG TRÌNH CỦA HVCH}
%\bibliographystyle{plain}
%\bibliography{References/author}
%\printbibliography[heading=subbibliography,
%title={Web Related},
%keyword=web]

\appendix
\renewcommand{\chaptername}{Appendix}
%\chapter{Parameters in Diffusion}
%\label{appendix:Appendix1}

\chapter{Optimal Hyperparameters}
\label{appendix:Appendix1}

In this section, we present the set of optimal hyperparameters used for both models: the attention-based model and the ConvKB model. The hyperparameter optimization process was conducted using grid search based on the Hits@10 evaluation metric. For the attention-based model, we trained on the entire dataset without any data splitting. In contrast, for the ConvKB model, we applied the same hyperparameter configuration across all datasets. The detailed hyperparameters are shown in the following table:


\begin{table}[htbp]
	\begin{center}
		\resizebox{\textwidth}{!}{%
			\begin{tabular}{llllllllll}
				\hline
				& $\mu$  & Weight decay & Epochs & negative ratio  & Dropouts  & $\alpha_{\text{LeakyRELU}}$ & $N_{\text{head}}$ & $D_{\text{final}}$ & $\gamma$ \\
				\hline
				FB15k     & 1e-3 & 1e-5 & 3000 & 2 & 0.3 & 0.2 & 2 & 200 & 1 \\
				FB15k-237 & 1e-3 & 1e-5 & 3000 & 2 & 0.3 & 0.2 & 2 & 200 & 1 \\
				WN18      & 1e-3 & 5e-6 & 3600 & 2 & 0.3 & 0.2 & 2 & 200 & 5 \\
				WN18RR    & 1e-3 & 5e-6 & 3600 & 2 & 0.3 & 0.2 & 2 & 200 & 5 \\
				\hline
			\end{tabular}
		}
	\end{center}
\end{table}


\end{document}