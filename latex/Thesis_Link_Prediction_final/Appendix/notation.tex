\begin{table}[hbbp]
\begin{center}
	\caption{Các ký hiệu sử dụng trong báo cáo}\label{notation}
	\resizebox{\textwidth}{!}{%
	\begin{tabularx}{\textwidth}{sb}
		\toprule
		Ký hiệu & Mô tả \\
		\hline
		$\mathcal{G}$ & Đồ thị \\
		$\mathcal{G}_{\text{mono}}$ & Đồ thị đồng nhất \\
		$\mathcal{G}_{\text{hete}}$ & Đồ thị không đồng nhất \\
		$\mathcal{G}_{\text{know}}$ & Đồ thị tri thức \\
		$V, E$ & Tập hợp đỉnh, Tập hợp cạnh  \\
		$e, e_i$ & Thực thể, thực thể thứ i  \\
		$r, r_k$ & Quan hệ, quan hệ thứ k  \\
		$t_{ijk}, t_{ij}^k$ & Một cạnh/bộ ba  \\
		$\overrightarrow{e}, \overrightarrow{r}$ & Thực thể nhúng, quan hệ nhúng  \\
		$\langle h, r, t \rangle$ & Bộ ba gồm thực thể đỉnh (head), quan hệ (relation), thực thể đuôi (tail) \\
		$T^v, T^e$ & Tập hợp loại đỉnh, loại cạnh  \\
		$N_e, N_r$ & Số lượng tập thực thể, tập quan hệ \\
		$N_{\text{head}}$ & Số lượng đỉnh tự chú ý \\
		$\mathbb{R}$ & Số thực  \\
		$\mathbf{E}, \mathbf{R}$ & Ma trận nhúng thực thể, quan hệ  \\
		$\mathbf{S}$ & Tập dữ liệu huấn luyện  \\
		$\ast$ & Phép tính tích chập  \\
		$\sigma$ & Hàm biến đổi phi tuyến tính  \\
		$\mathbf{W}$ & Ma trận trọng số  \\
		$\bigparallel_{k=1}^{K}$ & Phép ghép chồng từ 1 đến K lớp \\
		$||$ & Phép ghép chồng  \\
		${.}^T$ & Phép chuyển vị  \\
		$|| W ||^2_2$ & Chuẩn hóa L2  \\
		$\vee, \wedge$ & Phép hội, phép giao  \\
		$\oplus$ & Phép toán hai ngôi  \\
		$\cap$ & Phép hợp  \\
		$\neg$ & Phép phủ định  \\
		$\bigwedge^n_{i=1}$ & Phép nối liền  \\
		$\gamma$ & Biên lề  \\
		$\mu$ & Tốc độ học  \\
		$\omega$ & Số lượng lớp tích chập  \\
		$\Omega$ & Bộ lọc tính tích chập \\
		\bottomrule
	\end{tabularx}
	}
	
\end{center}
\end{table}