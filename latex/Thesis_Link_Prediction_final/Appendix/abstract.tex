\begin{center}
\begingroup
\section*{Tóm tắt}
\label{chap:Abstract}

\begin{adjustwidth}{1.5cm}{1.5cm}
	Đồ thị tri thức là cấu trúc giúp biểu diễn thông tin trong thế giới thực đã được Google nghiên cứu và phát triển cho công cụ tìm kiếm của mình rất thành công \cite{googlekg:2020}. Việc khai thác đồ thị tri thức không chỉ có truy vấn, phân tích mà còn hoàn thiện những thông tin còn thiếu cũng như dự đoán liên kết dựa trên những thông tin sẵn có trên đồ thị tri thức. Chính vì vậy, trong báo cáo này chúng tôi sẽ trình bày cơ bản về đồ thị tri thức và hai phương pháp để dự đoán liên kết trong đồ thị là phương pháp dựa trên luật và phương pháp dựa trên học sâu.
	Với phương pháp dựa trên luật, chúng tôi dựa trên mô hình AnyBURL và cung cấp thêm 2 chiến lược thêm tri thức mới vào đồ thị.
	
	Với phương pháp học sâu, chúng tôi trình bày lại cơ chế chú ý trong xử lý ngôn ngữ tự nhiên, từ đó được áp dụng vào đồ thị tri thức và trình bày lại đầy đủ mô hình cải tiến là mô hình KBGAT. Bằng cách ghép chồng các lớp với nhau mà những đỉnh có thể chú ý với những đặc trưng lân cận mà không tốn chi phí tính toán nào hoặc phụ thuộc vào việc biết trước cấu trúc đồ thị trước đó.
	%Cải tiến của chúng tôi dựa trên việc ghép cho một ma trận nhúng ở giữa
	Hai mô hình của chúng tôi đạt được kết quả tốt hơn đáng kể so với các phương pháp dự đoán liên kết khác được áp dụng trên bốn tập dữ liệu chuẩn.
\end{adjustwidth}
\endgroup
\end{center}