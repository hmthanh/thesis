\chapter{Kết luận}
\label{Chapter5}

Trong phần này chúng tôi đưa ra các kết quả đạt được của mô hình chúng tôi, chúng tôi cố gắng tìm hiểu các đặc trưng của các bộ dữ liệu tương ứng để cố gắng lý giải thích tại sao mô hình của chúng tôi hoặc các công trình khác có được kết quả tốt trên tập dữ liệu tương ứng đó. Những kết quả của hai đề xuất của chúng tôi cũng như các dịnh hướng nghiên cứu của chúng tôi trong tương lai.

Mặc dù kết quả chúng tôi cho thấy phương pháp của chúng tôi có hiệu suất tương đương với các mô hình học sâu hiện đại(state-of-art) và có ưu thế vượt trội trong thời gian đào tạo khoảng 17 phút so với thời gian hàng giờ của phương pháp học sâu khác nhưng không phải là các mô hình deep learning này không đáng nghiên cứu. Chúng tôi cũng nhận thấy đối với những tập dữ liệu khó như FB15-237 hay WN18RR phương pháp của chúng tôi thường cho kết quả không tốt do các quan hệ tương tự hay nghịch đảo không không xuất hiện trong ví dụ đào tạo nên chúng tôi khó tạo ra các luật đủ tốt để có thể khái quát hóa trên toàn bộ đồ thị đẫn đến các kết quả không tốt. Ngược lại đối với các phương pháp dựa trên học sâu lại có ưu thế rất lớn trong các tập dữ liệu này do có thể dễ đàng tính toán độ gần của các luật mới cần đánh giá so với các luật đã học từ đó có một kết quả khá tốt. Do đó chúng tôi cũng sẽ tiếp tục nghiên cứu các phương pháp học sâu và sẽ dùng phương pháp này làm đường cơ sở(base line) để so sánh với các nghiên cứu của chúng tôi trong tương lai. Một điểm yếu nữa của mô hình đựa trên luật của chúng tôi là mặc dù thời gian học là vượt trội nhưng thời gian để tính toán đưa ra đự đoán khá lâu do phải duyệt qua tất cả các luật được sinh ra mới có thể đưa ra dự đoán. Không giống như các phương pháp nhúng đồ thị khác thao tác này có thể dễ dàng tính toán.

Đối với hai thuật toán mở rộng của chúng tôi trong việc thêm tri thức mới vào đồ thị chúng tôi nhận thấy rằng là vượt trội hoàn toàn so với các phương pháp học sâu. Ở các phương pháp học sâu điều này đường như chưa được ai chú trọng nghiên cứu mặc dù thời gian đào tạo một mô hình là tương đối mất thời gian. Khi có tri thức mới hầu hết các mô hình phải đào tạo lại toàn bộ điều này khá lãng phí. Chúng tôi cũng xem đây là mục tiêu tiếp theo cho chúng tôi khi nghiên cứu các mô hình học sâu trong tương lai. Gần đây nhánh học tăng cường(Reinforcement learning) khá phát triển và nhóm tác giả Meilicke, Christian and Chekol \cite{meilicke2020reinforced} gần đây cũng đã có 1 nghiên cứu để tối ưu hóa lại phương pháp AnyBURL này. Chúng tôi cũng có ý định nghiên cứu về hướng này và cố gắng báo cáo lại trong một tương lai gần. 