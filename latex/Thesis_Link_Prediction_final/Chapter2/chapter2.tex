\chapter{Các công trình liên quan}
\label{Chapter2}

Hầu hết các nghiên cứu hiện tại về việc dự đoán liên kết của đồ thị tri thức đều liên quan đến các phương pháp tiếp cận tập trung vào khái niệm nhúng một đồ thị đã cho trong một không gian vectơ có số chiều thấp. Ngược lại với các tiếp cận này là một phương pháp đựa trên luật được nghiên cứu trong \cite{burl}. Thuật toán cốt lõi của nó dựa trên lấy mẫu một luật bất kỳ, sau đó khái quát  thành các quy tắc Horn\cite{wiki:Horn}. Tiếp đó dùng thống kê để tính độ tin cậy của các luật được khái quát. Khi dự đoán một liên kết mới (cạnh mới) của đồ thị chúng ta dự đoán một đỉnh có cạnh nối với một quan hệ cụ thể (label) với đỉnh còn lại hay không. Cũng đã có rất nhiều phương pháp được nghiên cứu, đề xuất để học các các luật trong đồ thị chẳng hạn như trong  RuDiK\cite{ortona2018robust}, AMIE\cite{galarraga2015fast}, RuleN\cite{meilicke2018fine}. 
Như đã nói trong phần trước có hai cách tiếp cận chính cho bài toán này một là tối ưu hóa hàm mục tiêu. Tìm ra một bộ quy tắc nhỏ bao gồm phần lớn các ví dụ là đúng và ít sai sót nhất có thể như được ngiên cứu trong RuDiK\cite{ortona2018robust}. Còn cách tiếp cận còn lại cũng là cách tiếp cận mà chúng tôi chọn nghiên cứu là cố gắng tìm hiểu mọi quy tắc khả thi có thể sau đó tạo xếp hạng \(k\) ứng viên tiềm năng với một độ tin cậy nhất định được đo trên tập huấn luyện.

Phương pháp đựa trên luật của chúng tôi phần lớn dựa vào phương pháp Anytime Bottom-Up Rule Learning for Knowledge Graph Completion \cite{meilicke2019anytime} mà sau đây chúng tôi gọi là \textbf{AnyBURL}. Như tên của phương pháp này phương pháp chủ yếu chú trọng vào vấn đề hoàn thành đồ thị, điền những phần còn thiếu vào đồ thị. Vấn đề tồn đọng lại ở mô hình này khi có một cạnh mới hay một tri thức mới được thêm vào đồ thị sẽ phải đào tạo lại toàn bộ mô hình. Chúng tôi giải quyết vẫn đề này theo hai chiến lược offline-to-online tức là khi thêm vào đồ thị tập hợp các cạnh thì mới thực hiện lại quá trình đào tạo lại một phần của đồ thị và chiến lược thứ 2 là online-to-online  khi thêm một cạnh mới sẽ thực hiện đào tạo lại ngay một phần có liên quan tới cạnh vừa thêm vào.

% GAT
Trong nhánh các phương pháp về học sâu, rất nhiều kỹ thuật học sâu thành công trong xử lý ảnh và xử lý ngôn ngữ tự nhiên được áp dụng vào đồ thị tri thức như : Mạng Neural Tích Chập (Convolution Neural Network - CNN \cite{lecun1999object}), Mạng Neural Hồi Quy (Recurrent Neural Network\cite{hopfield2007hopfield}), và gần đây như Transformer (\cite{yang2019xlnet}), Mạng Neural Bao Bọc (Capsule Neural Network - CapsNet \cite{sabour2017dynamic}). Bên cạnh đó các nghiên cứu còn sử dụng một số kỹ thuật khác như Random Walks, các mô hình dựa trên cấu trúc phân cấp, .. Ưu điểm chung của nhóm các phương pháp học sâu trên đồ thị tri thức đó là tự động rút trích các đặc trưng và có thể khái quát hóa cấu trúc phức tạp của đồ thị dựa trên một lượng lớn dữ liệu huấn luyện. Tuy nhiên, một số phương pháp chỉ chủ yếu tập trung vào cấu trúc dạng lưới mà không giữ được đặc trưng không gian của đồ thị tri thức. 
Cơ chế chú ý hay lớp chú ý đa đỉnh (multi-head attention layer) đã được áp dụng vào đồ thị bằng mô hình Mạng Đồ Thị Chú Ý (Graph Attention Network - GAT \cite{velivckovic2017graph}) giúp tổng hợp thông tin của một thực thể dựa vào trọng số chú ý của thực thể gốc đối với các thực thể lân cận. Tuy nhiên, mô hình đồ thị chú ý lại thiếu thông tin của vector nhúng quan hệ cũng như các vector nhúng lân cân của một thực thể gốc, một phần rất quan trọng giúp thể hiện vai trò của từng thực thể. Vấn đề đó đã được giải quyết trong báo cáo Learning Attention-based Embeddings for Relation Prediction in
Knowledge Graphs (\textbf{KBGAT} \cite{nathani2019learning}), mô hình được chúng tôi chọn làm cơ sở nghiên cứu.
Cơ chế chú ý đang là một trong những cấu trúc học sâu đạt được hiệu quả nhất hiện nay (state-of-the-art) vì nó đã được chứng minh là thay thế cho bất kỳ phương pháp tính tích chập nào \cite{cordonnier2019relationship},
hơn nữa nó cũng nằm trong cấu trúc cơ bản để áp dụng trên các mô hình mới nhất trên ngôn ngữ tự nhiên như mô hình Megatron-LM \cite{shoeybi2019megatron}, và trên phân đoạn hình ảnh như mô hình HRNet-OCR (Hierarchical Multi-Scale Attention \cite{tao2020hierarchical}). Một số phương pháp thú vị \cite{cordonnier2020multi} đã cải tiến dựa trên cơ chế chú ý, tuy nhiên nó lại chưa được áp dụng vào đồ thị tri thức, vì vậy chúng tôi chọn nhóm phương pháp này để áp dụng các cải tiến mới nhất vào đồ thị tri thức.