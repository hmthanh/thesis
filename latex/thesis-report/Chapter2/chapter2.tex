\chapter{Các công trình liên quan}
\label{Chapter2}
Hầu hết các nghiên cứu hiện tại về việc dự đoán liên kết của đồ thị tri thức đều liên quan đến các phương pháp tiếp cận tập trung vào khái niệm nhúng một đồ thị đã cho trong một không gian vectơ có số chiều thấp. Ngược lại với các tiếp cận này là một phương pháp đựa trên luật được nghiên cứu trong \cite{burl}. Thuật toán cốt lõi của nó dựa trên lấy mẫu một luật bất kỳ, sau đó khái quát  thành các quy tắc Horn\cite{wiki:Horn}. Tiếp đó dùng thống kê để tính độ tin cậy của các luật được khái quát. Khi dự đoán một liên kết mới (cạnh mới) của đồ thị chúng ta dự đoán một đỉnh có cạnh nối với một quan hệ cụ thể (label) với đỉnh còn lại hay không. Cũng đã có rất nhiều phương pháp được nghiên cứu, đề xuất để học các các luật trong đồ thị chẳng hạn như trong  RuDiK\cite{ortona2018robust}, AMIE\cite{galarraga2015fast}, RuleN\cite{meilicke2018fine}. 
Như đã nói trong phần trước có hai cách tiếp cận chính cho bài toán này một là tối ưu hóa hàm mục tiêu. Tìm ra một bộ quy tắc nhỏ bao gồm phần lớn các ví dụ là đúng và ít sai sót nhất có thể như được ngiên cứu trong RuDiK\cite{ortona2018robust}. Còn phương pháp của chúng tôi cố gắng tìm hiểu mọi quy tắc khả thi có thể sau đó tạo xếp hạng \(k\) ứng viên tiềm năng với một độ tin cậy nhất định được đo trên tập huấn luyện.

Phương pháp của chúng tôi phần lớn dựa vào phương pháp Anytime Bottom-Up Rule Learning for Knowledge Graph Completion \cite{meilicke2019anytime} mà sau đây chúng tôi gọi là \textbf{AnyBURL}. Như tên của phương pháp này phương pháp chủ yếu chú trọng vào vấn đề hoàn thành đồ thị, điền những phần còn thiếu vào đồ thị. Vấn đề tồn đọng lại ở mô hình này khi có một cạnh mới hay một tri thức mới được thêm vào đồ thị sẽ phải đào tạo lại toàn bộ mô hình. Chúng tôi giải quyết vẫn đề này theo hai chiến lược offline-to-online tức là khi thêm vào đồ thị tập hợp các cạnh thì mới thực hiện lại quá trình đào tạo lại một phần của đồ thị và chiến lược thứ 2 là online-to-online  khi thêm một cạnh mới sẽ thực hiện đào tạo lại ngay một phần có liên quan tới cạnh vừa thêm vào.