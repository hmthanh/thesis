\chapter{Phương pháp dựa trên luật}
\label{Chapter2}

\section{Giới Thiệu}

Hiện nay các bài toán liên quan đến dự đoán liên kết đồ thị tri thức lớn rất được quan tâm có khoảng bốn nhánh nghiên cứu chính như được nhắc đến trong nghiên cứu \cite{ampligraph} một trong số đó là phương pháp dựa trên luật logic. Vói mong muốn được tiếp cận các phương pháp từ đơn giản đến phức tạp nên chúng tôi chọn phương pháp này làm chủ để chính cho các báo cáo và nghiên cứu trong chương này trong chương này.

Hầu hết các nghiên cứu hiện tại về việc dự đoán liên kết của đồ thị tri thức đều liên quan đến các phương pháp tiếp cận tập trung vào khái niệm nhúng một đồ thị đã cho trong một không gian vectơ có số chiều thấp. Ngược lại với các tiếp cận này là một phương pháp đựa trên luật được nghiên cứu trong \cite{burl}.Thuật toán cốt lõi của nó dựa trên lấy mẫu một luật bất kỳ, sau đó khái quát  thành các quy tắc Horn. Tiếp đó dùng thống kê để tính độ tin cậy của các luật được khái quát. Khi dự đoán một liên kết mới (cạnh mới) của đồ thị chúng ta dự đoán một đỉnh có cạnh nối với một quan hệ cụ thể (label) với đỉnh còn lại hay không.

\section{Nội dung}
Trong phần này chúng tôi mô tả lại cách mô hình hóa lại bài toán theo phương pháp dựa trên luật, chiến lược đào tạo, cũng như một số kết quả thực nghiệm.
\subsection{Horn rule}
Trong logic toán học, một công thức nguyên tử - \textbf{atomic formula}\cite{wiki:Atomic} (còn được gọi đơn giản là một nguyên tử-\textbf{atom}) là một công thức không có cấu trúc mệnh đề, nghĩa là một công thức không chứa các liên kết logic (\(\vee, ~ \wedge\)) hoặc tương đương (\(\leftrightarrow\)) là một công thức không có các mẫu con nghiêm ngặt. Do đó, các nguyên tử là công thức đơn giản nhất để hình thành luật của logic. Các công thức hợp chất được hình thành bằng cách kết hợp các công thức nguyên tử bằng cách sử dụng các kết nối logic.

Một \textbf{literal}\cite{wiki:Literal} là một công thức nguyên tử (nguyên tử) hoặc phủ định của nó. Định nghĩa chủ yếu xuất hiện trong lý thuyết logic cổ điển. \textbf{Literal} có thể được chia thành hai loại: Một \textbf{positive literal} chỉ là một nguyên tử (ví dụ: \(x\)). Một \textbf{negative literal} là phủ định của một nguyên tử (ví dụ: \(\neg x\)). Sự phân chia của \textbf{literal} là \textbf{positive literal} hay \textbf{negative literal} tùy thuộc vào việc nó là \textbf{literal} được định nghĩa. Trong ngữ cảnh của một công thức ở dạng nối rời (\(\vee\)), một \textbf{literal} là thuần túy nếu phần bổ sung theo \textbf{literal} không xuất hiện trong công thức.

Một mệnh đề (clause) là một literal hoặc nối rời của hai hoặc nhiều literal. Ở dạng \textbf{Horn} một mệnh đề có nhiều nhất một positive literal. Lưu ý: Không phải mọi công thức trong logic mệnh đề đều có thể đưa về dạng Horn.Mệnh đề xác định không có literal đôi khi được gọi là mệnh đề đơn vị (unit clause) và một mệnh đề đơn vị không có biến đôi khi được gọi là \textit{facts}\cite{wiki:Horn}.Một công thức nguyên tử được gọi là \textit{ground} hoặc \textit{atoms ground} nếu nó được xây dựng hoàn toàn từ các mệnh đề đơn vị; tất cả các \textit{atoms ground} có thể ghép lại từ một tập hợp hàm và các ký hiệu vị ngữ nhất định tạo nên cơ sở Herbrand cho các bộ ký hiệu này\cite{wiki:Term}.

\subsection{Định nghĩa đồ thị logic}
Một đồ thị tri thức \(\mathbb{G}\) được định nghĩa trên một bộ từ vựng \(\langle \mathbb{C}, \mathbb{R} \rangle\) trong đó \(\mathbb{C}\) là tập hợp các hằng số và \(\mathbb{R}\) là tập hợp các vị từ nhị phân.Khi đó, \(\mathbb{G} = \{r (a, b) \mid r \in \mathbb{R}, a, b \in \mathbb{C}\}\) là tập hợp các \textit{ground atoms} hoặc \textit{facts}.

Nội dung báo cáo được phân thành các chương. Số thứ tự của các chương, mục được đánh số bằng hệ thống số Ả-rập, không dùng số La mã. Các mục và tiểu mục được đánh số bằng các nhóm hai hoặc ba chữ số, cách nhau một dấu chấm: số thứ nhất chỉ số chương, chỉ số thứ hai chỉ số mục, số thứ ba chỉ số tiểu mục.


%Báo cáo cần dùng LaTEX để viết và trình bày theo mẫu đã được cung cấp.

 Báo cáo trình bày sử dụng khổ giấy với việc canh lề như sau: Lề trên 3 cm, lề dưới 2,5 cm, lề trái 3 cm, lề phải 2 cm. Đánh số trang ở giữa bên dưới. Đánh số trang ở giữa bên dưới.

Font chữ dùng trong báo cáo (Times New Roman) với kích cỡ (size) 13-14pt, sử dụng chế độ dãn dòng (line spacing) chế độ 1.5 lines.

%Các bảng biểu trình bày theo chiều ngang khổ giấy thì đầu bảng là lề trái của trang. 


\section{Bố cục của báo cáo}

Nội dung của báo cáo tối thiểu 50 trang khổ A4 và không nên vượt quá 100 trang (không kể các trang bìa, lời cám ơn, mục lục, tài liệu tham khảo \ldots) theo trình tự như sau:

\begin{itemize}
\item MỞ ĐẦU (thường đặt tên là ``Giới thiệu''): Trình bày lý do chọn đề tài, mục đích, đối tượng và phạm vi nghiên cứu.
Mô tả bài toán mà đề tài giải quyết.
Bài toán này có gì hay?
Tại sao lại cần giải quyết bài toán này?
Bài toán này có gì khó?
Có những hướng nào để giải quyết bài toán này?
Những hướng giải quyết trước đây có những vấn đề gì chưa giải quyết được?
Các câu hỏi nghiên cứu mà đề tài trả lời hoặc những vấn đề mà đề tài sẽ giải quyết.
Các đóng góp của đề tài.

\item TỔNG QUAN (thường đặt tên là ``Các công trình liên quan''): Phân tích đánh giá các hướng nghiên cứu đã có của các tác giả trong và ngoài nước liên quan đến đề tài; nêu những vấn đề còn tồn tại (những vấn đề nào mà các công trình khác chưa giải quyết được); chỉ ra những vấn đề mà đề tài cần tập trung, nghiên cứu giải quyết.

\item NGHIÊN CỨU THỰC NGHIỆM HOẶC LÝ THUYẾT (thường đặt tên là ``Phương pháp đề xuất''): Trình bày cơ sở lý thuyết, lý luận, giả thiết khoa học và phương pháp nghiên cứu đã được sử dụng trong đề tài.

Nếu đề xuất hướng giải quyết mới, mô hình mới thì cần mô tả chi tiết cách giải quyết của mình (chi tiết tới mức người khác có thể dựa vào phần này mà cài đặt lại được đúng hoàn toàn phương pháp của mình đề ra).

\item TRÌNH BÀY, ĐÁNH GIÁ BÀN LUẬN VỀ CÁC KẾT QUẢ (thường đặt tên là ``Kết quả thí nghiệm''): Mô tả các kết quả nghiên cứu khoa học hoặc kết quả thực nghiệm.
Đối với  đề tài ứng dụng có kết quả là sản phẩm phần mềm phải có hồ sơ thiết kế, cài đặt,\ldots theo một trong các mô hình đã học (UML,\ldots).

Thông thường cần mô tả môi trường thí nghiệm trước như sử dụng dữ liệu nào, dùng độ đo nào để đánh giá, môi trường chạy thí nghiệm (cấu hình máy nếu cần phân tích thông tin về thời gian chạy thực nghiệm). Sau đó, nêu kết quả thực nghiệm, bàn luận và giải thích kết quả.

\item KẾT LUẬN VÀ HƯỚNG PHÁT TRIỂN (thường đặt tên là ``Kết luận''): Trình bày những kết quả đạt được, những đóng góp mới và những đề xuất mới, kiến nghị về những hướng nghiên cứu tiếp theo.

\item DANH MỤC TÀI LIỆU THAM KHẢO: Chỉ bao gồm các tài liệu được trích dẫn, sử dụng và đề cập tới để bàn luận trong báo cáo.
Phần này các bạn chuẩn bị 1 file BIB để lưu các tài liệu trích dẫn.
Khi các bạn trích dẫn một tài liệu nào đó, LaTeX sẽ tự động thêm vào danh mục tài liệu tham khảo giúp các bạn.
Các bạn xem hướng dẫn cách trích dẫn ở chương sau.

\item PHỤ LỤC: Phần này bao gồm nội dung cần thiết nhằm minh họa hoặc hỗ trợ cho nội dung báo cáo như số liệu, mẫu biểu, tranh ảnh,\ldots Phụ lục không được dày hơn phần chính của báo cáo.
Nếu có công trình công bố thì để vào phần phụ lục này.
\end{itemize}

\section{Bảng biểu, hình vẽ, phương trình}

%Những qui định dưới này các bạn có thể bỏ qua hoặc đọc để hiểu thêm.
%Những định dạng này LaTeX đều tự động giúp các bạn.
%Các bạn xem hướng dẫn chi tiết hơn ở chương sau.

Việc đánh số bảng biểu, hình vẽ, phương trình phải gắn với số chương; ví dụ hình 3.4 có nghĩa là hình thứ 4 trong Chương 3.
Mọi đồ thị, bảng biểu, hình vẽ lấy từ các nguồn khác phải được trích dẫn đầy đủ.

\subsection{Bảng biểu, hình vẽ}



Đầu đề của bảng biểu ghi phía trên bảng, đầu đề của hình vẽ ghi phía dưới hình.

Thông thường, những bảng ngắn và đồ thị phải đi liền với phần nội dung đề cập tới các bảng và đồ thị này ở lần thứ nhất.
Các bảng dài có thể để ở những trang riêng nhưng cũng phải tiếp theo ngay phần nội dung đề cập tới bảng này ở lần đầu tiên.
Các bảng rộng vẫn nên trình bày theo chiều đứng dài 297mm của trang giấy, chiều rộng của trang giấy có thể hơn 210mm.
Chú ý gấp trang giấy sao cho số và đầu đề của hình vẽ hoặc bảng vẫn có thể nhìn thấy ngay mà không cần mở rộng tờ giấy.
Tuy nhiên hạn chế sử dụng các bảng quá rộng này.

Đối với những trang giấy có chiều đứng hơn 297mm (bản đồ, bản vẽ,\ldots) thì có thể để trong một phong bì cứng đính bên trong bìa sau của báo cáo.

Các hình vẽ phải sạch sẽ bằng mực đen để có thể sao chụp lại; có đánh số và ghi đầy đủ đầu đề, cỡ chữ phải bằng cỡ chữ sử dụng trong báo cáo.

Khi đề cập đến các bảng biểu và hình vẽ phải nêu rõ số của hình và bảng biểu đó, ví dụ ``... được nêu trong Bảng 4.1'' hoặc ``xem Hình 3.2'' mà không được viết ``… được nêu trong bảng dưới đây'' hoặc ``trong đồ thị của X và Y sau''.

\subsection{Phương trình toán học}

Việc trình bày phương trình toán học trên một dòng đơn hoặc dòng kép tùy ý, tuy nhiên phải thống nhất trong toàn báo cáo.

Khi ký hiệu xuất hiện lần đầu tiên thì phải giải thích và đơn vị tính phải đi kèm ngay trong phương trình có ký hiệu đó.
Nếu cần thiết, danh mục của tất cả các ký hiệu, chữ viết tắt và nghĩa của chúng cần được liệt kê và để ở phần đầu của báo cáo.

Tất cả các phương trình cần được đánh số và để trong ngoặc đơn đặt bên phía lề phải.
Nếu một nhóm phương trình mang cùng một số thì những số này cũng được để trong ngoặc, hoặc mỗi phương trình trong nhóm phương trình (5.1) có thể được đánh số là (5.1.1), (5.1.2), (5.1.3).

\section{Viết tắt}

\textbf{Không lạm dụng việc viết tắt} trong báo cáo.
Chỉ viết tắt những từ, cụm từ hoặc thuật ngữ được sử dụng nhiều lần trong báo cáo.
Không viết tắt những cụm từ  dài, những mệnh đề; không viết tắt những cụm từ ít xuất hiện trong báo cáo.
Nếu cần viết tắt những từ thuật ngữ, tên các cơ quan, tổ chức,\ldots thì được viết tắt sau lần viết thứ nhất có kèm theo chữ viết tắt trong ngoặc đơn.
Nếu báo cáo có nhiều chữ viết tắt thì phải có bảng danh mục các chữ viết tắt (xếp theo thứ tự ABC) ở phần đầu báo cáo.

Nhắc lại: \textbf{không lạm dụng việc viết tắt} trong báo cáo.
Khi các bạn sử dụng từ viết tắt, người đọc sẽ phải lật lại những phần đã đọc, để tìm lại xem từ viết tắt đó nghĩa là gì.
Việc này sẽ làm chậm tốc độ đọc và sẽ khiến người đọc khó theo dõi báo cáo của bạn hơn.
Nếu có thể, hạn chế hoàn toàn việc dùng viết tắt.

\section{Tài liệu tham khảo và cách trích dẫn}

Mọi ý kiến, khái niệm có ý nghĩa, mang tính chất gợi ý không phải của riêng tác giả và mọi tham khảo khác phải được trích dẫn và chỉ ra nguồn trong danh mục Tài liệu tham khảo của báo cáo. Nguồn được trích dẫn phải được liệt kê chính xác trong danh mục Tài liệu tham khảo.

Việc trích dẫn, tham khảo chủ yếu nhằm thừa nhận nguồn của những ý tưởng có giá trị giúp người đọc theo được mạch suy nghĩ của tác giả, không làm trở ngại việc đọc.


Không trích dẫn những kiến thức phổ biến, mọi người đều biết cũng như không làm báo cáo nặng nề với những tham khảo trích dẫn.

Nếu không có điều kiện tiếp cận được một tài liệu gốc mà phải trích dẫn thông qua một tài liệu khác thì phải nêu ra trích dẫn này, đồng thời tài liệu gốc đó không được liệt kê trong danh mục tài liệu tham khảo của báo cáo.