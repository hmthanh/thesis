\chapter{Phương pháp dựa trên luật}
\label{Chapter2}

\section{Giới Thiệu}

Hiện nay các bài toán liên quan đến dự đoán liên kết đồ thị tri thức lớn rất được quan tâm có khoảng bốn nhánh nghiên cứu chính như được nhắc đến trong nghiên cứu \cite{ampligraph} một trong số đó là phương pháp dựa trên luật logic. Với mong muốn được tiếp cận các phương pháp từ đơn giản đến phức tạp nên chúng tôi chọn phương pháp này làm chủ để chính cho các báo cáo và nghiên cứu trong chương này. Với phương pháp này đưa ra một xếp hạng \(k\) ứng viên với một số điểm nhất định biểu thị cho sự chắc chán của dự đoán nó phù hợp với các hệ thống gợi ý(recommender system).

\section{Các công trình liên quan}
Hầu hết các nghiên cứu hiện tại về việc dự đoán liên kết của đồ thị tri thức đều liên quan đến các phương pháp tiếp cận tập trung vào khái niệm nhúng một đồ thị đã cho trong một không gian vectơ có số chiều thấp. Ngược lại với các tiếp cận này là một phương pháp đựa trên luật được nghiên cứu trong \cite{burl}.Thuật toán cốt lõi của nó dựa trên lấy mẫu một luật bất kỳ, sau đó khái quát  thành các quy tắc Horn\cite{wiki:Horn} Tiếp đó dùng thống kê để tính độ tin cậy của các luật được khái quát. Khi dự đoán một liên kết mới (cạnh mới) của đồ thị chúng ta dự đoán một đỉnh có cạnh nối với một quan hệ cụ thể (label) với đỉnh còn lại hay không.Cũng đã có rất nhiều phương pháp được nghiên cứu, đề xuất để học các các luật trong đồ thị chẳng hạn như trong   RuDiK \cite{dettmers2018convolutional}, AMIE\cite{galarraga2015fast}, RuleN\cite{meilicke2018fine}. Trong phương pháp RuDiK học các luật có thể học cả quy tắc positive và negative. Trong khi phương pháp của chúng tôi chỉ học các quy tắc positive. RuDiK tìm ra một bộ quy tắc nhỏ bao gồm phần lớn các ví dụ là đúng và ít sai sót nhất có thể. Điều này khác với mục tiêu của chúng tôi, chúng tôi cố gắng tìm hiểu mọi quy tắc khả thi có thể sau đó tạo xếp hạng \(k\) ứng viên tiềm năng với một độ tin cậy nhất định đuọc đo trên tập huấn luyện.

Phương pháp của chúng tôi phần lớn dựa vào phương pháp Anytime Bottom-Up Rule Learning for Knowledge Graph Completion \cite{meilicke2019anytime} mà sau đây chúng tôi gọi là \textbf{AnyBURL}. Vấn đề tồn đọng lại ở mô hình này khi có một cạnh mới hay một tri thức mới được thêm vào đồ thị sẽ phải đào tạo lại toàn bộ mô hình. Chúng tôi giải quyết vẫn đề này theo hai chiến lược offline-to-online tức là khi thêm vào đồ thị tập hợp các cạnh thì mới thực hiện lại quá trình đào tạo lại một phần của đồ thị và chiến lược thứ 2 là online-to-online  khi thêm một cạnh mới sẽ thực hiện đào tạo lại ngay một phần có liên quan tới cạnh vừa thêm vào.
\section{Phương pháp đề xuất}
Trong phần này chúng tôi mô tả lại cách mô hình hóa lại bài toán theo phương pháp dựa trên luật AnyBURL,thuật toán lấy mẫu luật (đường đẫn) và thuật toán khái quát hóa một luật để lưu trữ trở thành tri thức của mô hình. Cùng với những cải tiến của chúng tôi trong quá trình đào tạo khi có một tri thức mới được thêm vào đồ thị(thêm cạnh).
\subsection{Horn rule}
Trong logic toán học, một công thức nguyên tử - \textbf{atomic formula}\cite{wiki:Atomic} (còn được gọi đơn giản là một nguyên tử-\textbf{atom}) là một công thức không có cấu trúc mệnh đề, nghĩa là một công thức không chứa các liên kết logic (\(\vee, ~ \wedge\)) hoặc tương đương (\(\Leftrightarrow\)) là một công thức không có các mẫu con nghiêm ngặt (tức là atom không thể chia nhỏ ra thành các atom con nữa). Do đó, các nguyên tử là công thức đơn giản nhất để hình thành luật của logic. Các công thức hợp được hình thành bằng cách kết hợp các công thức nguyên tử bằng cách sử dụng các liên kết logic.

Một \textbf{literal}\cite{wiki:Literal} là một công thức nguyên tử (nguyên tử) hoặc phủ định của nó. Định nghĩa chủ yếu xuất hiện trong lý thuyết logic cổ điển. \textbf{Literal} có thể được chia thành hai loại: Một \textbf{positive literal} chỉ là một nguyên tử (ví dụ: \(x\)). Một \textbf{negative literal} là phủ định của một nguyên tử (ví dụ: \(\neg x\)). Sự phân chia của \textbf{literal} là \textbf{positive literal} hay \textbf{negative literal} tùy thuộc vào việc \textbf{literal} được định nghĩa.

Một mệnh đề (clause) là một literal hoặc nối rời của hai hoặc nhiều literal. Ở dạng \textbf{Horn} một mệnh đề có nhiều nhất một positive literal. Lưu ý: Không phải mọi công thức trong logic mệnh đề đều có thể đưa về dạng Horn.Mệnh đề xác định không có literal đôi khi được gọi là mệnh đề đơn vị (unit clause) và một mệnh đề đơn vị không có biến đôi khi được gọi là \textit{facts}\cite{wiki:Horn}.Một công thức nguyên tử được gọi là \textit{ground} hoặc \textit{ground atoms} nếu nó được xây dựng hoàn toàn từ các mệnh đề đơn vị; tất cả các \textit{ground atoms} có thể ghép lại từ một tập hợp hàm và các ký hiệu vị từ nhất định tạo nên cơ sở Herbrand cho các bộ ký hiệu này\cite{wiki:Term}.

\subsection{Định nghĩa đồ thị  tri thức}
Một đồ thị tri thức \(\mathbb{G}\) được định nghĩa trên một bộ từ vựng \(\langle \mathbb{C}, \mathbb{R} \rangle\) trong đó \(\mathbb{C}\) là tập hợp các hằng số và \(\mathbb{R}\) là tập hợp các vị từ nhị phân.Khi đó, \(\mathbb{G} = \{r (a, b) \mid r \in \mathbb{R}, a, b \in \mathbb{C}\}\) là tập hợp các \textit{ground atoms} hoặc \textit{facts}. Một vị từ nhị phân được gọi là quan hệ và hằng số (hoặc hằng số được đề cập đến) được gọi là thực thể (entity) tương ứng với một dòng dữ liệu trong tập huấn luyện. Sau đây chúng tôi sử dụng các chữ cái viết thường cho các hằng và chữ in hoa cho các biến cho các thảo luận dưới đây. Vì chúng ta không học các quy tắc Horn tùy ý, và chỉ học đối với loại quy tắc nào có thể được khái quát hóa như được thảo luận dưới đây.

Chúng ta định nghĩa một quy tắc là \(h(c_0, c_n) \gets b_1(c_0, c_1) ,\dots ,b_n(c_{n}, c_{n + 1})\) là một đường dẫn ground atoms có chiều dài \(n\). Trong đó \(h(\dots)\) được gọi là \textit{head atoms} và \( b_1(c_0, c_1) ,\dots ,b_n(c_{n}, c_{n + 1})\) được gọi là \textit{body atoms}. Chúng tôi sẽ phân biệt dưới đây ba loại quy tắc mà chúng tôi gọi là: \textit{quy tắc nhị phân} \((\mathbf{B})\) là quy tắc trong head atoms chứa 2 biến, quy tắc đơn nguyên kết thúc bằng một đỉnh treo  và atom này chỉ chứa biến không chứa hằng số\((\mathbf{U_d})\) và head atoms chỉ chứa 1 biến. Còn quy tắc đơn nguyên kết thúc bằng một atom \((\mathbf{U_c)}\) và head atoms cũng chỉ chứa 1 biến.\((\mathbf{U_c)}\)  có thể là một đỉnh treo tới một hằng số bất kì nếu hằng số này trùng mới hằng số trong head atom thì tạo thành một đường đẫn có chu trình.

\[B \hspace{3.7cm} h(A_0,A_n) \gets  \bigwedge^n_{i=1} b_i(A_{i-1}, A_i)\]
\[U_d \hspace{3.8cm} h(A_0,c) \gets  \bigwedge^n_{i=1} b_i(A_{i-1}, A_i)\]
\[U_c \hspace{1cm} h(A_0,c) \gets  \bigwedge^{n-1}_{i=1} b_i(A_{i-1}, A_i) \wedge b_n(A_{n-1}, c^{\prime})\]

Chúng tôi gọi các quy tắc của các loại này là quy tắc đường đi (path rules), bởi vì các body atoms (phần sau đấu \(\gets\)) tạo thành một đường đi. Lưu ý rằng nó cũng bao gồm các biến thể quy tắc với các biến được đảo ngược trong các nguyên tử: được đưa ra trong đồ thị tri thức \(\mathbb{G}\), đường dẫn có độ dài \(n\) là một chuỗi gồm \(n\) bộ ba \(p_i (c_i, c_i + 1)\) với \(p_i (c_i, c_i + 1) \in \mathbb{G}\) hoặc \(p_i (c_i + 1, c_i) \in \mathbb{G}\) với \(0 \geq i \leq n\). Các mẫu quy tắc trừu tượng (abstract rule patterns) được cho ở trên có độ dài \(n\) vì body atoms của chúng có thể được khởi tạo thành một đường dẫn có độ dài \(n\).

Ngoài ra Quy tắc \(B\) và quy tắc \(U_c\) cũng được gọi là quy tắc kết nối kín. Chúng có thể được học bởi hệ thống khai thác AMIE được mô tả trong \cite{AMIE,galarraga2015fast}. Quy tắc \(U_d\) là quy tắc không đóng hay đường đi không tạo thành chu trình vì \(A_n\) là biến chỉ xuất hiện một lần. Ví dụ:
\[
\begin{matrix}
\textit{speaks}(X, Y ) & \gets & \textit{lives}(X, Y) & \quad (1) \\
\textit{lives\_in\_city}(X, Y ) & \gets & \textit{lives}(X, A),\textit{within}(Y, A)  & \quad  (2) \\
\textit{gen}(X, female) & \gets & \textit{married}(X, A), \textit{gen}(A, male)  & \quad  (3) \\
\textit{profession}(X, actor) &  \gets & \textit{acted\_in}(X, A)  & \quad (4)
\end{matrix}
\]
Quy tắc (1) là quy tắc \textbf{B}(quy tắc nhị phân) quy tắc này nói rằng nếu một người (thực thể) \(X\) nói nguôn ngữ \(Y\) nếu người \(X\) sống  ở đất nước \(Y\).Rõ ràng quy tắc này là một quy tắc khái quát miễn khi nào thực thể \(X\) có cạnh nối với thực thể \(Y\) với nhãn là \textit{lives} thì có thể kết thêm 1 cạnh với nhãn \textit{speaks} giữa \(X\) và \(Y\). Quy tắc (2), (3) điều là quy tắc \(U_c\) ,quy tắc (2) nói rằng người \(X\) sống ở thành phố \(Y\) nếu người \(X\) sống ở quốc gia \(A\) và thành phố \(Y\) nằm trong quốc gia \(A\), quy tắc (3) nói rằng nếu một người \(X\) là nữ nếu họ kết hôn với một người \(A\) và người \(A\) có giới tính nam. Ở quy tắc (3) không tạo thành chu trình trên đồ thị như quy tắc (2) đỉnh (Y) lặp lại  ở \textit{head atom} và đỉnh cuối cùng trong \textit{body atoms}. Quy tắc (4) nói rằng người \(X\) là một điễn viên nếu người \(X\) đóng trong một bộ phim \(A\).

Tất cả các quy tắc được xem xét sẽ được lọc lại đựa trên điểm được gọi là độ tin cậy của quy tắc là được đo trên tập dữ liệu huấn luyện. Độ tin cậy này được đo bằng tỷ lệ body atoms dẫn đến head atoms chia cho tất cả các đường đãn chứa body atoms.Ví dụ khi ta có quy tắc sau:
\(\textit{gen}(X, female) \gets \textit{married}(X, A), \textit{gen}(A, male) \). Khi đó chúng ta thực hiện đếm tất cả các cặp thực thể có quan hệ  \(\textit{married}(X, A), \textit{gen}(A, male) \) được gọi là số đường dẫn chứa body atoms, sau đó thực hiện đếm tất cả các  thực thể thỏa quan hệ \(\textit{gen}(X, female) \gets textit{married}(X, A), \textit{gen}(A, male) \) được gọi là số body atoms dẫn đến head atoms. Sau đó chia số body atoms dẫn đến head atoms cho  đường dẫn chứa body atoms được gọi là độ tin cậy của quy tắc.
\subsection{Thuật toán}
\subsubsection{Thuật toán AnyBURL}
\begin{algorithm}
\caption{Anytime Bottom-up Rule Learning}\label{euclid}
\begin{algorithmic}[1]
\Procedure{AnyBURL($\mathbb{G}$, s, sat, Q, ts)}{}
\State $\textit{n} = \text{2}$
\State $R = \emptyset$
\Loop
\State $R_s = \emptyset$
\State $start = currentTime()$
\Repeat
\State $p = samplePath(\mathbb{G}, n)$
\State $R_p = generateRules(p)$
\For {$r \in R_p$}
\State $score(r, s)$
\If {$Q(r)$}
	\State $R_s = R_s \cup \{r\}$
\EndIf
\EndFor
\Until {$currentTime() > start + ts$}
\State $R^{\prime}_s = R_s \cap R$
\If {$ \mid R^{\prime} \mid / \mid R \mid > SAT$}
	\State $n = n + 1$
\EndIf
\State $R = R_s \cap R$
\EndLoop
\Return R
\EndProcedure
\end{algorithmic}
\end{algorithm}


\subsubsection{Thuật toán khái quát hóa}
\begin{algorithm}
\caption{generateRules(p)}\label{euclid}
\begin{algorithmic}[1]
\Procedure{MyProcedure}{}
\State $\textit{stringlen} \gets \text{length of }\textit{string}$
\State $i \gets \textit{patlen}$
\BState \emph{top}:
\If {$i > \textit{stringlen}$} \Return false
\EndIf
\State $j \gets \textit{patlen}$
\BState \emph{loop}:
\If {$\textit{string}(i) = \textit{path}(j)$}
\State $j \gets j-1$.
\State $i \gets i-1$.
\State \textbf{goto} \emph{loop}.
\State \textbf{close};
\EndIf
\State $i \gets i+\max(\textit{delta}_1(\textit{string}(i)),\textit{delta}_2(j))$.
\State \textbf{goto} \emph{top}.
\EndProcedure
\end{algorithmic}
\end{algorithm}

Nội dung báo cáo được phân thành các chương. Số thứ tự của các chương, mục được đánh số bằng hệ thống số Ả-rập, không dùng số La mã. Các mục và tiểu mục được đánh số bằng các nhóm hai hoặc ba chữ số, cách nhau một dấu chấm: số thứ nhất chỉ số chương, chỉ số thứ hai chỉ số mục, số thứ ba chỉ số tiểu mục.

%Báo cáo cần dùng LaTEX để viết và trình bày theo mẫu đã được cung cấp.

 Báo cáo trình bày sử dụng khổ giấy với việc canh lề như sau: Lề trên 3 cm, lề dưới 2,5 cm, lề trái 3 cm, lề phải 2 cm. Đánh số trang ở giữa bên dưới. Đánh số trang ở giữa bên dưới.

Font chữ dùng trong báo cáo (Times New Roman) với kích cỡ (size) 13-14pt, sử dụng chế độ dãn dòng (line spacing) chế độ 1.5 lines.

%Các bảng biểu trình bày theo chiều ngang khổ giấy thì đầu bảng là lề trái của trang.

\section{Kết quả thí nghiệm}
Trong phần này chúng tôi mô tả lại các bộ dữ liệu mà chúng tôi đùng để thực nghiệm đánh giá phương pháp của chúng tôi cùng với so sánh với các kết quả khác của các công trình nổi bật khác. Cùng với đó chúng tôi cố gắng tìm hiểu các đặc trưng của các bộ dữ liệu tương ứng để cố gắng lý giải thích tại sao mô hình của chúng tôi hoặc các công trình khác có được kết quả tốt trên tập dữ liệu tương ứng đó.
\subsection{Các tập dữ liệu huấn luyện}
Trong thí nghiệm của chúng tôi, chúng tôi thực hiện trên bốn tập dữ liệu phổ biến là FB15k, FB15-237, WN18 và WN18RR.Các bộ dữ liệu này là tập hợp các bộ ba (triple) \(\langle head, relation, tail \rangle\) biểu thị cho thực thể đầu có một mối quan hệ với thực thể cuối.

Bộ dữ liệu FB15k: Bộ dữ liệu này được tạo bởi nhóm nghiên cứu A. Bordes, N. Usunier, A. Garcia-Duran, J. Weston, and O. Yakhnenko \cite{bordes2013translating} bằng cách trích xuất từ bộ dữ liệu Wikilinks database \footnote{https://code.google.com/archive/p/wiki-links/}.Wikilinks database thu thập các siêu liên kết(hyperlinks) đến Wikipedia gồm 40 triệu lượt đề cập trên 3 triệu thực thể, họ trích xuất tất cả các dữ kiện liên quan đến một thực thể nhất định cùng với tất cả các dữ kiện liên quan đến thực thể đó (bao gồm cả những thực thể con được nhắc đến trong trang Wikipedia đó), ngoại trừ những thonng tin như: ngày tháng, danh từ riêng, v.v ... Họ cũng chuyển đổi các đỉnh có bậc \(n\) được biểu diễn với sự biến đổi thành các nhóm các cạnh nhị phân.

Bộ dữ liệu FB15k-237: là một tập hợp con của FB15k được xây dựng bởi Toutanova và Chen \cite{toutanova2015observed} lấy cảm hứng từ quan sát rằng FB15k thu được từ rò rỉ thử nghiệm, bao gồm dữ liệu thử nghiệm được các mô hình nhìn thấy tại thời điểm đào tạo. Trong FB15k, vấn đề này là do sự hiện diện của các quan hệ gần giống nhau hoặc nghịch đảo của nhau.FB15k-237 được xây dựng để trở thành một tập dữ liệu thách thức hơn: các tác giả đã chọn các dữ kiện đầu tiên từ FB15k liên quan đến 401 quan hệ xuất hiện nhiều nhất và loại bỏ tất cả các quan hệ tương đương hoặc nghịch đảo. Họ cũng đảm bảo rằng không có thực thể nào được kết nối trong tập huấn luyện cũng được liên kết trực tiếp trong tập xác nhận và kiểm tra.

Bộ dữ liệu WN18: được giới thiệu bởi các tác giả của TransE \cite{bordes2013translating}, được trích xuất từ WordNet\footnote{https://wordnet.princeton.edu/}, một bản thể học ngôn ngữ KG có nghĩa là cung cấp một từ điển / từ đồng nghĩa để hỗ trợ NLP và phân tích văn bản tự động. Trong WordNet, các thực thể tương ứng với các tập hợp (\textit{word senses}) và các quan hệ đại diện cho các kết nối từ vựng của chúng (ví dụ: “hypernym”). Để xây dựng WN18, các tác giả đã sử dụng WordNet làm điểm bắt đầu và sau đó lặp đi lặp lại lọc ra các thực thể và mối quan hệ với quá ít đề cập

Bộ dữ liệu WN18RR: là một tập hợp con của WN18 được xây dựng bởi DeŠmers et al. \cite{dettmers2017convolutional}, cũng là người quan sát rò rỉ thử nghiệm trong WN18. Chúng tôi chứng minh mức độ nghiêm trọng của sự rò rỉ nói trên bằng cách chỉ ra rằng một mô hình dựa trên quy tắc đơn giản dựa trên phát hiện quan hệ nghịch đảo, được gọi là Mô hình nghịch đảo, đạt được kết quả hiện đại ở cả WN18 và FB15k. Để giải quyết vấn đề đó, họ xây dựng tập dữ liệu WN18RR thách thức hơn nhiều bằng cách áp dụng một đường ống tương tự như đường ống được sử dụng cho FB15k-237 [57]. Gần đây đã được các tác giả thừa nhận [56] rằng tập kiểm tra bao gồm 212 thực thể không xuất hiện trong tập huấn luyện, khiến cho việc dự đoán hợp lý khoảng \(6,7\%\) dữ kiện thử nghiệm là không thể.

\subsection{Kết quả thực nghiệm trên}
Như đã nói trước đây với mô hình dựa trên luật của chúng tôi hoàn toàn có thể thực hiện trên một laptop với cấu hình thông thường. Trong thí nghiệm của chúng tôi cấu hình máy để thực thi như sau: T480, core i5 8th Gen, ram 16Gb, 4 core 8 thread. Mã nguồn thực thi được viết bằng ngôn ngữ Python phiên bản 3.6 và dùng các hàm hỗ trợ có sẵn trong Python với không một thư viện bên thứ ba nào. Thí nghiệm được thực hiện với bốn tập dữ liệu phổ biến là FB15k, FB15-237, WN18 và WN18RR. Thông tin chi tiết các bộ dữ liệu này được mô tả ở phần [?].

Bảng ? bên dưới mô tả các kết quả thực nghiệm của chúng tôi với các độ đo \(h@k\) cùng với các kết quả thực nghiệm của các phương pháp khác được đề cập trong khảo  sát \cite{rossi2020knowledge}

\begin{table}[ht]
\resizebox{\textwidth}{!}{%
\begin{tabular}{lllllllllllllllll}
\cline{2-17}
\multicolumn{1}{l|}{} &
  \multicolumn{4}{c|}{\textbf{FB15k}} &
  \multicolumn{4}{c|}{\textbf{FB15-237}} &
  \multicolumn{4}{c|}{\textbf{WN18}} &
  \multicolumn{4}{c|}{\textbf{WN18RR}} \\ \cline{2-17}
\multicolumn{1}{l|}{} &
  \multicolumn{1}{l|}{\textbf{H@1}} &
  \multicolumn{1}{l|}{\textbf{H@10}} &
  \multicolumn{1}{l|}{\textbf{MR}} &
  \multicolumn{1}{l|}{\textbf{MRR}} &
  \multicolumn{1}{l|}{\textbf{H@1}} &
  \multicolumn{1}{l|}{\textbf{H@10}} &
  \multicolumn{1}{l|}{\textbf{MR}} &
  \multicolumn{1}{l|}{\textbf{MRR}} &
  \multicolumn{1}{l|}{\textbf{H@1}} &
  \multicolumn{1}{l|}{\textbf{H@10}} &
  \multicolumn{1}{l|}{\textbf{MR}} &
  \multicolumn{1}{l|}{\textbf{MRR}} &
  \multicolumn{1}{l|}{\textbf{H@1}} &
  \multicolumn{1}{l|}{\textbf{H@10}} &
  \multicolumn{1}{l|}{\textbf{MR}} &
  \multicolumn{1}{l|}{\textbf{MRR}} \\ \hline
\multicolumn{1}{|l|}{\textbf{ComplEx}} &
  \multicolumn{1}{l|}{81.56} &
  \multicolumn{1}{l|}{90.53} &
  \multicolumn{1}{l|}{34} &
  \multicolumn{1}{l|}{0.848} &
  \multicolumn{1}{l|}{94.53} &
  \multicolumn{1}{l|}{95.50} &
  \multicolumn{1}{l|}{3623} &
  \multicolumn{1}{l|}{0.949} &
  \multicolumn{1}{l|}{25.72} &
  \multicolumn{1}{l|}{52.97} &
  \multicolumn{1}{l|}{202} &
  \multicolumn{1}{l|}{0.349} &
  \multicolumn{1}{l|}{42.55} &
  \multicolumn{1}{l|}{52.12} &
  \multicolumn{1}{l|}{4907} &
  \multicolumn{1}{l|}{0.458} \\ \hline
\multicolumn{1}{|l|}{\textbf{TuckER}} &
  \multicolumn{1}{l|}{72.89} &
  \multicolumn{1}{l|}{88.88} &
  \multicolumn{1}{l|}{39} &
  \multicolumn{1}{l|}{0.788} &
  \multicolumn{1}{l|}{94.64} &
  \multicolumn{1}{l|}{95.80} &
  \multicolumn{1}{l|}{510} &
  \multicolumn{1}{l|}{0.951} &
  \multicolumn{1}{l|}{25.90} &
  \multicolumn{1}{l|}{53.61} &
  \multicolumn{1}{l|}{162} &
  \multicolumn{1}{l|}{0.352} &
  \multicolumn{1}{l|}{42.95} &
  \multicolumn{1}{l|}{51.40} &
  \multicolumn{1}{l|}{6239} &
  \multicolumn{1}{l|}{0.459} \\ \hline
\multicolumn{1}{|l|}{\textbf{TransE}} &
  \multicolumn{1}{l|}{49.36} &
  \multicolumn{1}{l|}{84.73} &
  \multicolumn{1}{l|}{45} &
  \multicolumn{1}{l|}{0.628} &
  \multicolumn{1}{l|}{40.56} &
  \multicolumn{1}{l|}{94.87} &
  \multicolumn{1}{l|}{279} &
  \multicolumn{1}{l|}{0.646} &
  \multicolumn{1}{l|}{21.72} &
  \multicolumn{1}{l|}{49.65} &
  \multicolumn{1}{l|}{209} &
  \multicolumn{1}{l|}{0.31} &
  \multicolumn{1}{l|}{2.79} &
  \multicolumn{1}{l|}{49.65} &
  \multicolumn{1}{l|}{3936} &
  \multicolumn{1}{l|}{0.206} \\ \hline
\multicolumn{1}{|l|}{\textbf{RotatE}} &
  \multicolumn{1}{l|}{73.93} &
  \multicolumn{1}{l|}{88.10} &
  \multicolumn{1}{l|}{42} &
  \multicolumn{1}{l|}{0.791} &
  \multicolumn{1}{l|}{94.30} &
  \multicolumn{1}{l|}{96.02} &
  \multicolumn{1}{l|}{274} &
  \multicolumn{1}{l|}{0.949} &
  \multicolumn{1}{l|}{23.83} &
  \multicolumn{1}{l|}{53.06} &
  \multicolumn{1}{l|}{178} &
  \multicolumn{1}{l|}{0.336} &
  \multicolumn{1}{l|}{42.60} &
  \multicolumn{1}{l|}{57.35} &
  \multicolumn{1}{l|}{3318} &
  \multicolumn{1}{l|}{0.475} \\ \hline
\multicolumn{1}{|l|}{\textbf{ConvKB}} &
  \multicolumn{1}{l|}{11.44} &
  \multicolumn{1}{l|}{40.83} &
  \multicolumn{1}{l|}{324} &
  \multicolumn{1}{l|}{0.211} &
  \multicolumn{1}{l|}{94.89} &
  \multicolumn{1}{l|}{52.89} &
  \multicolumn{1}{l|}{202} &
  \multicolumn{1}{l|}{0.709} &
  \multicolumn{1}{l|}{13.98} &
  \multicolumn{1}{l|}{41.46} &
  \multicolumn{1}{l|}{309} &
  \multicolumn{1}{l|}{0.230} &
  \multicolumn{1}{l|}{5.63} &
  \multicolumn{1}{l|}{52.50} &
  \multicolumn{1}{l|}{3429} &
  \multicolumn{1}{l|}{0.249} \\ \hline
\multicolumn{1}{|l|}{\textbf{GAT}} &
  \multicolumn{1}{l|}{} &
  \multicolumn{1}{l|}{} &
  \multicolumn{1}{l|}{} &
  \multicolumn{1}{l|}{} &
  \multicolumn{1}{l|}{} &
  \multicolumn{1}{l|}{} &
  \multicolumn{1}{l|}{} &
  \multicolumn{1}{l|}{} &
  \multicolumn{1}{l|}{} &
  \multicolumn{1}{l|}{} &
  \multicolumn{1}{l|}{} &
  \multicolumn{1}{l|}{} &
  \multicolumn{1}{l|}{} &
  \multicolumn{1}{l|}{} &
  \multicolumn{1}{l|}{} &
  \multicolumn{1}{l|}{} \\ \hline
\multicolumn{1}{|l|}{\textbf{BURL}} &
  \multicolumn{1}{l|}{79.13} &
  \multicolumn{1}{l|}{82.30} &
  \multicolumn{1}{l|}{} &
  \multicolumn{1}{l|}{} &
  \multicolumn{1}{l|}{20.85} &
  \multicolumn{1}{l|}{42.40} &
  \multicolumn{1}{l|}{} &
  \multicolumn{1}{l|}{} &
  \multicolumn{1}{l|}{\textbf{93.86}} &
  \multicolumn{1}{l|}{\textbf{94.07}} &
  \multicolumn{1}{l|}{} &
  \multicolumn{1}{l|}{} &
  \multicolumn{1}{l|}{\textbf{44.22}} &
  \multicolumn{1}{l|}{54.23} &
  \multicolumn{1}{l|}{} &
  \multicolumn{1}{l|}{} \\ \hline
Rank &
   &
   &
   &
   &
   &
   &
   &
   &
   &
   &
   &
   &
   &
   &
   &

\end{tabular}}
\end{table}


\section{Bố cục của báo cáo}

Nội dung của báo cáo tối thiểu 50 trang khổ A4 và không nên vượt quá 100 trang (không kể các trang bìa, lời cám ơn, mục lục, tài liệu tham khảo \ldots) theo trình tự như sau:

\begin{itemize}
\item MỞ ĐẦU (thường đặt tên là ``Giới thiệu''): Trình bày lý do chọn đề tài, mục đích, đối tượng và phạm vi nghiên cứu.
Mô tả bài toán mà đề tài giải quyết.
Bài toán này có gì hay?
Tại sao lại cần giải quyết bài toán này?
Bài toán này có gì khó?
Có những hướng nào để giải quyết bài toán này?
Những hướng giải quyết trước đây có những vấn đề gì chưa giải quyết được?
Các câu hỏi nghiên cứu mà đề tài trả lời hoặc những vấn đề mà đề tài sẽ giải quyết.
Các đóng góp của đề tài.

\item TỔNG QUAN (thường đặt tên là ``Các công trình liên quan''): Phân tích đánh giá các hướng nghiên cứu đã có của các tác giả trong và ngoài nước liên quan đến đề tài; nêu những vấn đề còn tồn tại (những vấn đề nào mà các công trình khác chưa giải quyết được); chỉ ra những vấn đề mà đề tài cần tập trung, nghiên cứu giải quyết.

\item NGHIÊN CỨU THỰC NGHIỆM HOẶC LÝ THUYẾT (thường đặt tên là ``Phương pháp đề xuất''): Trình bày cơ sở lý thuyết, lý luận, giả thiết khoa học và phương pháp nghiên cứu đã được sử dụng trong đề tài.

Nếu đề xuất hướng giải quyết mới, mô hình mới thì cần mô tả chi tiết cách giải quyết của mình (chi tiết tới mức người khác có thể dựa vào phần này mà cài đặt lại được đúng hoàn toàn phương pháp của mình đề ra).

\item TRÌNH BÀY, ĐÁNH GIÁ BÀN LUẬN VỀ CÁC KẾT QUẢ (thường đặt tên là ``Kết quả thí nghiệm''): Mô tả các kết quả nghiên cứu khoa học hoặc kết quả thực nghiệm.
Đối với  đề tài ứng dụng có kết quả là sản phẩm phần mềm phải có hồ sơ thiết kế, cài đặt,\ldots theo một trong các mô hình đã học (UML,\ldots).

Thông thường cần mô tả môi trường thí nghiệm trước như sử dụng dữ liệu nào, dùng độ đo nào để đánh giá, môi trường chạy thí nghiệm (cấu hình máy nếu cần phân tích thông tin về thời gian chạy thực nghiệm). Sau đó, nêu kết quả thực nghiệm, bàn luận và giải thích kết quả.

\item KẾT LUẬN VÀ HƯỚNG PHÁT TRIỂN (thường đặt tên là ``Kết luận''): Trình bày những kết quả đạt được, những đóng góp mới và những đề xuất mới, kiến nghị về những hướng nghiên cứu tiếp theo.

\item DANH MỤC TÀI LIỆU THAM KHẢO: Chỉ bao gồm các tài liệu được trích dẫn, sử dụng và đề cập tới để bàn luận trong báo cáo.
Phần này các bạn chuẩn bị 1 file BIB để lưu các tài liệu trích dẫn.
Khi các bạn trích dẫn một tài liệu nào đó, LaTeX sẽ tự động thêm vào danh mục tài liệu tham khảo giúp các bạn.
Các bạn xem hướng dẫn cách trích dẫn ở chương sau.

\item PHỤ LỤC: Phần này bao gồm nội dung cần thiết nhằm minh họa hoặc hỗ trợ cho nội dung báo cáo như số liệu, mẫu biểu, tranh ảnh,\ldots Phụ lục không được dày hơn phần chính của báo cáo.
Nếu có công trình công bố thì để vào phần phụ lục này.
\end{itemize}

\section{Bảng biểu, hình vẽ, phương trình}

%Những qui định dưới này các bạn có thể bỏ qua hoặc đọc để hiểu thêm.
%Những định dạng này LaTeX đều tự động giúp các bạn.
%Các bạn xem hướng dẫn chi tiết hơn ở chương sau.

Việc đánh số bảng biểu, hình vẽ, phương trình phải gắn với số chương; ví dụ hình 3.4 có nghĩa là hình thứ 4 trong Chương 3.
Mọi đồ thị, bảng biểu, hình vẽ lấy từ các nguồn khác phải được trích dẫn đầy đủ.

\subsection{Bảng biểu, hình vẽ}



Đầu đề của bảng biểu ghi phía trên bảng, đầu đề của hình vẽ ghi phía dưới hình.

Thông thường, những bảng ngắn và đồ thị phải đi liền với phần nội dung đề cập tới các bảng và đồ thị này ở lần thứ nhất.
Các bảng dài có thể để ở những trang riêng nhưng cũng phải tiếp theo ngay phần nội dung đề cập tới bảng này ở lần đầu tiên.
Các bảng rộng vẫn nên trình bày theo chiều đứng dài 297mm của trang giấy, chiều rộng của trang giấy có thể hơn 210mm.
Chú ý gấp trang giấy sao cho số và đầu đề của hình vẽ hoặc bảng vẫn có thể nhìn thấy ngay mà không cần mở rộng tờ giấy.
Tuy nhiên hạn chế sử dụng các bảng quá rộng này.

Đối với những trang giấy có chiều đứng hơn 297mm (bản đồ, bản vẽ,\ldots) thì có thể để trong một phong bì cứng đính bên trong bìa sau của báo cáo.

Các hình vẽ phải sạch sẽ bằng mực đen để có thể sao chụp lại; có đánh số và ghi đầy đủ đầu đề, cỡ chữ phải bằng cỡ chữ sử dụng trong báo cáo.

Khi đề cập đến các bảng biểu và hình vẽ phải nêu rõ số của hình và bảng biểu đó, ví dụ ``... được nêu trong Bảng 4.1'' hoặc ``xem Hình 3.2'' mà không được viết ``… được nêu trong bảng dưới đây'' hoặc ``trong đồ thị của X và Y sau''.

\subsection{Phương trình toán học}

Việc trình bày phương trình toán học trên một dòng đơn hoặc dòng kép tùy ý, tuy nhiên phải thống nhất trong toàn báo cáo.

Khi ký hiệu xuất hiện lần đầu tiên thì phải giải thích và đơn vị tính phải đi kèm ngay trong phương trình có ký hiệu đó.
Nếu cần thiết, danh mục của tất cả các ký hiệu, chữ viết tắt và nghĩa của chúng cần được liệt kê và để ở phần đầu của báo cáo.

Tất cả các phương trình cần được đánh số và để trong ngoặc đơn đặt bên phía lề phải.
Nếu một nhóm phương trình mang cùng một số thì những số này cũng được để trong ngoặc, hoặc mỗi phương trình trong nhóm phương trình (5.1) có thể được đánh số là (5.1.1), (5.1.2), (5.1.3).

\section{Viết tắt}

\textbf{Không lạm dụng việc viết tắt} trong báo cáo.
Chỉ viết tắt những từ, cụm từ hoặc thuật ngữ được sử dụng nhiều lần trong báo cáo.
Không viết tắt những cụm từ  dài, những mệnh đề; không viết tắt những cụm từ ít xuất hiện trong báo cáo.
Nếu cần viết tắt những từ thuật ngữ, tên các cơ quan, tổ chức,\ldots thì được viết tắt sau lần viết thứ nhất có kèm theo chữ viết tắt trong ngoặc đơn.
Nếu báo cáo có nhiều chữ viết tắt thì phải có bảng danh mục các chữ viết tắt (xếp theo thứ tự ABC) ở phần đầu báo cáo.

Nhắc lại: \textbf{không lạm dụng việc viết tắt} trong báo cáo.
Khi các bạn sử dụng từ viết tắt, người đọc sẽ phải lật lại những phần đã đọc, để tìm lại xem từ viết tắt đó nghĩa là gì.
Việc này sẽ làm chậm tốc độ đọc và sẽ khiến người đọc khó theo dõi báo cáo của bạn hơn.
Nếu có thể, hạn chế hoàn toàn việc dùng viết tắt.

\section{Tài liệu tham khảo và cách trích dẫn}

Mọi ý kiến, khái niệm có ý nghĩa, mang tính chất gợi ý không phải của riêng tác giả và mọi tham khảo khác phải được trích dẫn và chỉ ra nguồn trong danh mục Tài liệu tham khảo của báo cáo. Nguồn được trích dẫn phải được liệt kê chính xác trong danh mục Tài liệu tham khảo.

Việc trích dẫn, tham khảo chủ yếu nhằm thừa nhận nguồn của những ý tưởng có giá trị giúp người đọc theo được mạch suy nghĩ của tác giả, không làm trở ngại việc đọc.


Không trích dẫn những kiến thức phổ biến, mọi người đều biết cũng như không làm báo cáo nặng nề với những tham khảo trích dẫn.

Nếu không có điều kiện tiếp cận được một tài liệu gốc mà phải trích dẫn thông qua một tài liệu khác thì phải nêu ra trích dẫn này, đồng thời tài liệu gốc đó không được liệt kê trong danh mục tài liệu tham khảo của báo cáo.