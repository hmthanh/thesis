% Template KLTN cho SV trường ĐHKHTN
% Liên hệ: nqminh@fit.hcmus.edu.vn
% Last update: 08/06/2016

% Chú ý: đọc các phần chú ý đóng khung của file này và chỉnh lại cho phù hợp.
% Trước khi build, xóa hết các file được tạo ra trong quá trình build trước đó, và build theo thứ tự: BIB > PDF > PDF.
% Nếu cập nhật tài liệu tham khảo, cũng cần build lại theo cách trên.

\documentclass[oneside,a4paper,14pt]{extreport}

% Font tiếng Việt
\usepackage[T5]{fontenc}
\usepackage[utf8]{inputenc}
\DeclareTextSymbolDefault{\DH}{T1}

% Tài liệu tham khảo
\usepackage[
sorting=nty,
backend=bibtex,
defernumbers=true]{biblatex}
\usepackage[unicode]{hyperref} % Bookmark tiếng Việt
\addbibresource{References/references.bib}

\makeatletter
\def\blx@maxline{77}
\makeatother

% Chèn hình, các hình trong luận văn được để trong thư mục Images/
\usepackage{graphicx}
\graphicspath{ {./images/} }

% Chèn và định dạng mã nguồn
\usepackage{listings}
\usepackage{color}
\definecolor{codegreen}{rgb}{0,0.6,0}
\definecolor{codegray}{rgb}{0.5,0.5,0.5}
\definecolor{codepurple}{rgb}{0.58,0,0.82}
\definecolor{backcolour}{rgb}{0.95,0.95,0.92}
\lstdefinestyle{mystyle}{
	backgroundcolor=\color{backcolour},   
	commentstyle=\color{codegreen},
	keywordstyle=\color{magenta},
	numberstyle=\tiny\color{codegray},
	stringstyle=\color{codepurple},
	basicstyle=\footnotesize,
	breakatwhitespace=false,         
	breaklines=true,                 
	captionpos=b,                    
	keepspaces=true,                 
	numbers=left,                    
	numbersep=5pt,                  
	showspaces=false,                
	showstringspaces=false,
	showtabs=false,                  
	tabsize=2
}
\lstset{style=mystyle}

% Chèn và định dạng mã giả
\usepackage{amsmath}
\usepackage{algorithm}
\usepackage[noend]{algpseudocode}
\makeatletter
\def\BState{\State\hskip-\ALG@thistlm}
\makeatother

% Bảng biểu
\usepackage{multirow}
\usepackage{array}
\newcolumntype{L}[1]{>{\raggedright\let\newline\\\arraybackslash\hspace{0pt}}m{#1}}
\newcolumntype{C}[1]{>{\centering\let\newline\\\arraybackslash\hspace{0pt}}m{#1}}
\newcolumntype{R}[1]{>{\raggedleft\let\newline\\\arraybackslash\hspace{0pt}}m{#1}}

% Đổi tên mặc định
\renewcommand{\chaptername}{Chương}
\renewcommand{\figurename}{Hình}
\renewcommand{\tablename}{Bảng}
\renewcommand{\contentsname}{Mục lục}
\renewcommand{\listfigurename}{Danh sách hình}
\renewcommand{\listtablename}{Danh sách bảng}
\renewcommand{\appendixname}{Phụ lục}

% Dãn dòng 1.5
\usepackage{setspace}
\onehalfspacing

% Thụt vào đầu dòng
\usepackage{indentfirst}

% Canh lề
\usepackage[
top=30mm,
bottom=25mm,
left=30mm,
right=20mm,
includefoot]{geometry}

% Trang bìa
\usepackage{tikz}
\usetikzlibrary{calc}
\newcommand\HRule{\rule{\textwidth}{1pt}}
% ========================================================================================= %
% error citation
\usepackage[strings]{underscore}
\usepackage{url}

% Math
\usepackage{amsfonts}

\usepackage{stmaryrd}
\newcommand*\concat{\mathbin{\|}} % parallel

% Vẽ cây
\usetikzlibrary{trees}

% subfigure
\usepackage{subcaption}
\usepackage[justification=centering]{caption}

% Định nghĩa
\newtheorem{definition}{Định nghĩa}

% Bảng thuật ngữ
\usepackage[acronym,nomain]{glossaries}

% mindmap
\usetikzlibrary{shapes.geometric, arrows}

% ========================================================================================= %
% CHÚ Ý: Thông tin chung về KLTN - sinh viên điền vào đây để tự động update các trang khác  %
% ========================================================================================= %
\newcommand{\tenSV}{Hoàng~Minh~Thanh~-~Phan~Minh~Tân} % Dấu ~ là khoảng trắng không được tách (các chữ nối với nhau bằng dấu ~ sẽ nằm cùng 1 dòng
\newcommand{\mssv}{18424062~-~18424059}
\newcommand{\tenKL}{Đự~đoán~liên~kết~trong~đồ~thị~phức} % Chú ý dấu ~ trong tên khóa luận
\newcommand{\tenGVHD}{Th.S~Lê~Ngọc~Thành}
\newcommand{\tenBM}{Khoa~học~máy~tính}

\begin{document}
	\begin{titlepage}

\begin{center}
%ĐẠI HỌC QUỐC GIA THÀNH PHỐ HỒ CHÍ MINH\\
TRƯỜNG ĐẠI HỌC KHOA HỌC TỰ NHIÊN\\
\textbf{KHOA CÔNG NGHỆ THÔNG TIN}\\[2cm]


{ \Large \bfseries Hoàng Minh Thanh - Phan Minh Tâm\\[2cm] } 

%Tên đề tài Khóa luận tốt nghiệp/Đồ án tốt nghiệp

{ \Large \bfseries DỰ ĐOÁN LIÊN KẾT TRONG\\ĐỒ THỊ PHỨC\\KNOWLEDGE GRAPH EMBEDDING FOR LINKING PREDICTION\\[2cm] } 


%Chọn trong các dòng sau
\large KHÓA LUẬN TỐT NGHIỆP CỬ NHÂN\\
%\large ĐỒ ÁN TỐT NGHIỆP CỬ NHÂN\\
%\large THỰC TẬP TỐT NGHIỆP CỬ NHÂN\\
%Đưa vào dòng này nếu thuộc chương trình Chất lượng cao, hoặc lớp Cử nhân tài năng
\large CHƯƠNG TRÌNH HOÀN CHỈNH ĐẠI HỌC\\
%\large CHƯƠNG TRÌNH CHẤT LƯỢNG CAO\\
%\large CHƯƠNG TRÌNH CỬ NHÂN TÀI NĂNG\\[2cm]

\begin{tikzpicture}[remember picture, overlay]
  \draw[line width = 2pt] ($(current page.north west) + (2cm,-2cm)$) rectangle ($(current page.south east) + (-1.5cm,2cm)$);
\end{tikzpicture}

\vfill
Tp. Hồ Chí Minh, tháng 09/2020

\end{center}

\pagebreak



\begin{center}

TRƯỜNG ĐẠI HỌC KHOA HỌC TỰ NHIÊN\\
\textbf{KHOA CÔNG NGHỆ THÔNG TIN}\\[2cm]


{\large \bfseries Hoàng Minh Thanh - 18424062\\} 
{\large \bfseries Phan Minh Tâm - 18424059\\[2cm]}

%Tên đề tài Khóa luận tốt nghiệp/Đồ án tốt nghiệp

{ \Large \bfseries DỰ ĐOÁN LIÊN KẾT TRONG\\ĐỒ THỊ PHỨC\\KNOWLEDGE GRAPH EMBEDDING FOR LINKING PREDICTION\\[2cm] } 


%Chọn trong các dòng sau
\large KHÓA LUẬN TỐT NGHIỆP CỬ NHÂN\\
%\large ĐỒ ÁN TỐT NGHIỆP CỬ NHÂN\\
%Đưa vào dòng này nếu thuộc chương trình Chất lượng cao, hoặc lớp Cử nhân tài năng
\large CHƯƠNG TRÌNH HOÀN CHỈNH ĐẠI HỌC\\[2cm]
%\large CHƯƠNG TRÌNH CHẤT LƯỢNG CAO\\[2cm]
%\large CHƯƠNG TRÌNH CỬ NHÂN TÀI NĂNG\\[2cm]

\textbf{GIÁO VIÊN HƯỚNG DẪN}\\
Th.S Lê Ngọc Thành\\

\begin{tikzpicture}[remember picture, overlay]
  \draw[line width = 2pt] ($(current page.north west) + (2cm,-2cm)$) rectangle ($(current page.south east) + (-1.5cm,2cm)$);
\end{tikzpicture}

\vfill
Tp. Hồ Chí Minh, tháng 09/2020

\end{center}

\end{titlepage}
	% Sasu trang Title, các bạn chèn nhận xét gủa GVHD và GVPB. Nhận xét sẽ được giáo vụ phát sau buổi bảo vệ để các bạn đóng quyển.
	
	\pagenumbering{roman} % Đánh số i, ii, iii, ...
	
	%\addcontentsline{toc}{chapter}{Lời cam đoan}
	%\chapter*{Lời cam đoan}
\label{reassurances}

Tôi xin cam đoan đây là công trình nghiên cứu của riêng chúng tôi. Các số liệu và kết quả nghiên cứu trong luận văn này là trung thực và không trùng lặp với các đề tài khác.s
	
	\addcontentsline{toc}{chapter}{Lời cảm ơn}
	\chapter*{Lời cảm ơn}
\label{thanks}

Tôi xin chân thành cảm ơn ...
	
	% Mục lục, danh sách hình, danh sách bảng
	\addcontentsline{toc}{chapter}{Mục lục}
	\tableofcontents
	\listoffigures
	\listoftables
	
	
	\begin{center}
\begingroup
\section*{Tóm tắt}
\label{chap:Abstract}

\begin{adjustwidth}{1.5cm}{1.5cm}
	Đồ thị tri thức là cấu trúc giúp biểu diễn thông tin trong thế giới thực đã được Google nghiên cứu và phát triển cho công cụ tìm kiếm của mình rất thành công \cite{googlekg:2020}. Việc khai thác đồ thị tri thức không chỉ có truy vấn, phân tích mà còn hoàn thiện những thông tin còn thiếu cũng như dự đoán liên kết dựa trên những thông tin sẵn có trên đồ thị tri thức. Chính vì vậy, trong báo cáo này chúng tôi sẽ trình bày cơ bản về đồ thị tri thức và hai phương pháp để dự đoán liên kết trong đồ thị là phương pháp dựa trên luật và phương pháp dựa trên học sâu.
	Với phương pháp dựa trên luật, mô hình cải tiến của chúng tôi AnyBURL giúp ... 
	
	Với phương pháp học sâu, chúng tôi cải tiến lại mô hình KBGAT bằng mô hình mà chúng tôi gọi là GCAT. Bằng cách ghép chồng các lớp với nhau mà những nút có thể chú ý với những đặc trưng lân cận mà không tốn chi phí tính toán nào hoặc phụ thuộc vào việc biết trước cấu trúc đồ thị trước đó.
	%Cải tiến của chúng tôi dựa trên việc ghép cho một ma trận nhúng ở giữa
	Hai mô hình của chúng tôi đạt được kết quả tốt hơn đáng kể so với các phương pháp dự đoán liên kết khác được áp dụng trên bốn tập dữ liệu chuẩn.
\end{adjustwidth}
\endgroup
\end{center}
	
	\clearpage
	
	\pagenumbering{arabic} % Đánh số 1, 2, 3, ...
	
	% Các chương nội dung
	% 1. Giới thiệu
	\chapter{Giới thiệu}
\label{introduction}

Ngôn ngữ để viết và trình bày báo cáo khóa luận tốt nghiệp, đồ án tốt nghiệp, thực tập tốt nghiệp (sau đây gọi chung là báo cáo) là tiếng Việt hoặc tiếng Anh. 
Trường hợp chọn ngôn ngữ tiếng Anh để viết và trình bày báo cáo,  sinh viên cần có đơn đề nghị, được cán bộ hướng dẫn (CBHD) đồng ý và nộp cho bộ phận Giáo vụ của Khoa vào thời điểm đăng ký đề tài để xin ý kiến.
Báo cáo viết và trình bày bằng tiếng Anh phải có bản tóm tắt viết bằng tiếng Việt.


%Tóm tắt luận văn được trình bày nhiều nhất trong 24 trang in trên hai mặt giấy, cỡ chữ Times New Roman 11 của hệ soạn thảo Winword hoặc phần mềm soạn thảo Latex đối với các chuyên ngành thuộc ngành Toán.

%Mật độ chữ bình thường, không được nén hoặc kéo dãn khoảng cách giữa các chữ.
%Chế độ dãn dòng là Exactly 17pt.
%Lề trên, lề dưới, lề trái, lề phải đều là 1.5 cm.
%Các bảng biểu trình bày theo chiều ngang khổ giấy thì đầu bảng là lề trái của trang.
%Tóm tắt luận án phải phản ảnh trung thực kết cấu, bố cục và nội dung của luận án, phải ghi đầy đủ toàn văn kết luận của luận án.
%Mẫu trình bày trang bìa của tóm tắt luận văn (phụ lục 1).
	
	% 2. Tổng quan (Cơ sở lý thuyết và Các nghiên cứu liên quan)
	\chapter{Tổng quan}
\label{overview}

sdfsdf

% chapter 2, main file
\section{This is chapter two}

sdfsdfsdfsdf


\section{Các nghiên cứu liên quan}

Sau thời kỳ ngủ đông của AI, các kỹ thuật học sâu có một bước tiến đang kể với các mô hình CNN \cite{lecun1999object}, RNN \cite{hopfield2007hopfield}, Attention \cite{vaswani2017attention}, Transformer \cite{yang2019xlnet} đã đạt được những kết quả rất lớn với độ chính xác cao cả trong nghiên cứu cũng như khi áp dụng vào công nghiệp. Các mô hình học sâu, cụ thể là CNN đã rất thành công để giải quyết các vấn đề về phân loại hình ảnh \cite{he2016deep} , phân đoạn ngữ nghĩa \cite{jegou2017one} và dịch máy , chúng tạo ra một bản đồ đặc trưng biểu diễn cấu trúc thông tin dưới dạng lưới. CNN cũng đã được áp dụng vào trong cấu trúc đồ thị với mô hình GCN \cite{kipf2016semi}. 
% \begin{document}
% \section{Cơ sở lý thuyết}
% % chapter 2, main file
\section{This is chapter two}

sdfsdfsdfsdf


% Báo cáo phải được trình bày ngắn gọn, rõ ràng, mạch lạc, sạch sẽ, không được tẩy xóa, có đánh số trang, đánh số bảng biểu, hình vẽ, đồ thị. 

% Nội dung báo cáo được phân thành các chương. Số thứ tự của các chương, mục được đánh số bằng hệ thống số Ả-rập, không dùng số La mã. Các mục và tiểu mục được đánh số bằng các nhóm hai hoặc ba chữ số, cách nhau một dấu chấm: số thứ nhất chỉ số chương, chỉ số thứ hai chỉ số mục, số thứ ba chỉ số tiểu mục.


% %Báo cáo cần dùng LaTEX để viết và trình bày theo mẫu đã được cung cấp.

%  Báo cáo trình bày sử dụng khổ giấy với việc canh lề như sau: Lề trên 3 cm, lề dưới 2,5 cm, lề trái 3 cm, lề phải 2 cm. Đánh số trang ở giữa bên dưới. Đánh số trang ở giữa bên dưới.

% Font chữ dùng trong báo cáo (Times New Roman) với kích cỡ (size) 13-14pt, sử dụng chế độ dãn dòng (line spacing) chế độ 1.5 lines.

% %Các bảng biểu trình bày theo chiều ngang khổ giấy thì đầu bảng là lề trái của trang. 


% \section{Các nghiên cứu liên quan}
% \section{Các nghiên cứu liên quan}

Sau thời kỳ ngủ đông của AI, các kỹ thuật học sâu có một bước tiến đang kể với các mô hình CNN \cite{lecun1999object}, RNN \cite{hopfield2007hopfield}, Attention \cite{vaswani2017attention}, Transformer \cite{yang2019xlnet} đã đạt được những kết quả rất lớn với độ chính xác cao cả trong nghiên cứu cũng như khi áp dụng vào công nghiệp. Các mô hình học sâu, cụ thể là CNN đã rất thành công để giải quyết các vấn đề về phân loại hình ảnh \cite{he2016deep} , phân đoạn ngữ nghĩa \cite{jegou2017one} và dịch máy , chúng tạo ra một bản đồ đặc trưng biểu diễn cấu trúc thông tin dưới dạng lưới. CNN cũng đã được áp dụng vào trong cấu trúc đồ thị với mô hình GCN \cite{kipf2016semi}. 

% Nội dung của báo cáo tối thiểu 50 trang khổ A4 và không nên vượt quá 100 trang (không kể các trang bìa, lời cám ơn, mục lục, tài liệu tham khảo \ldots) theo trình tự như sau:
% \end{document}
	
	% 3. Phương pháp đề xuất
	\chapter{Phương pháp đề xuất}
\label{approachs}

sdfsdf

\section{AnyBURL}

Anytime Bottom-Up Rule Learning for Knowledge Graph Completion \cite{meilicke2019anytime}
sdfsdf

Dùng lệnh để trích dẫn một hoặc nhiều tài liệu
Lưu ý khi trích dẫn tài liệu tham khảo, cần viết câu sao cho bỏ phần trong cặp ngoặc vuông đi thì câu vẫn đầy đủ ý nghĩa.
Ví dụ, thay 
Một ví dụ khác, thay vì viết ``... như trong công trình nghiên viết ``... nh'

Để chèn mã nguồn, cần dùng package listings
\section{CGAT}

Ở phần này chúng tôi sẽ giới thiệu tóm lược về mạng đồ thị chú ý (GATs \cite{velivckovic2017graph}) và cải tiến của chúng tôi trên lớp GAT mà chúng tôi gọi là \textit{mạng cộng tác đồ thị chú ý} CGAT, và sau đó chúng tôi áp dụng vào để xây dựng đồ thị theo mô hình KGAT \cite{nathani2019learning} để tối ưu quá trình dự đoán các mối quan hệ. 

\subsection{Mô hình GAT}

Mạng đồ thị tích chập (GCNs \cite{schlichtkrull2018modeling}) giúp tổng hợp thông tin bằng cách tính trung bình thông tin từ các thực thể lân cận, tuy nhiên cách này sẽ làm cho các thực thể có trọng số ngang bằng nhau không biểu diễn đúng thông tin trong thế giới thực. Để giải quyết vấn đề đó, GATs \cite{velivckovic2017graph} ra đời để đối xử với các node lân cận bằng sự quan trọng của chúng.

Đầu vào của mô hình là vector biểu diễn đặc trưng của từng thực thể (entity) $E = \overrightarrow{e_1} + \overrightarrow{e_2} + ... + \overrightarrow{e_N}$. Và mục tiêu của chúng ta là biến đổi thành một đặc trưng đầu ra mới $E'' = \overrightarrow{e''_1} + \overrightarrow{e''_2} + ... + \overrightarrow{e''_N}$; với $\overrightarrow{e_i}$ và $\overrightarrow{e'_i} \in \mathcal{R}^k$ tương ứng là vector nhúng đầu vào và vector đầu ra của của thực thể $e_i$, N là số lượng của các thực thể (nodes), k là đặc trưng đầu vào.

Mô hình sẽ đi qua hai quá trình biến đổi vector đặc trưng $\overrightarrow{e_i}$ và có thể tóm lược như sau :
\begin{align}
\overrightarrow{e_i} \longrightarrow \overrightarrow{e'_i} \longrightarrow \overrightarrow{e''_i}
\end{align}

Ở quá trình biên đổi đầu tiên, mô hình sẽ tổng hợp thông tin từ các thực thể lân cận và ghép chồng lên nhau để tạo ra vector $\overrightarrow{e'_i}$ sau đó mô hình sẽ dùng vector $\overrightarrow{e'_i}$ để coi là vector nhúng của thực thể cho lớp mới và tiếp tục quá trình tổng hợp từ các thông tin lân cận và tạo ra vector $\overrightarrow{e''_i}$ cuối cùng.

Đầu tiên để tham số hóa quá trình biến đổi tuyến tính, ta cần một trọng số $W \in \mathbb{R}^{N_e \times k}$ ánh xạ vector đầu vào thành một vector mới với miền không gian lớn hơn và một hàm chú ý $a$ chúng ta tùy chọn :

\begin{align}
\centering
{e_{ij}}&={a(W \overrightarrow{e_i}, W \overrightarrow{e_i})}
\end{align}

trong đó $e_{ij}$ là giá trị chú ý của một cạnh $(e_i, e_j)$ trong đồ thị $\mathcal{G}$ hay $e_{ij}$ thể hiện sự quan trọng của đặc trưng cạnh $(e_i, e_j)$ so với thực thể $e_i$. Sau đó, chúng ta áp dụng hàm \textit{softmax} qua tất cả các giá trị nhúng của hàng xóm để tạo ra $\alpha_{ij}$ . Quá trình tổng hợp các sự chú ý được thể hiện ở biểu thức sau : 

\begin{align}
\centering
{\overrightarrow{a_{ij}}}&={\sigma\left(\sum_{j\in \mathcal{N}(i)} {\alpha_{ij} \mathbf{W} \overrightarrow{e_j} }\right)}
\end{align}

Tiếp theo mô hình GAT sẽ đi qua \textit{lớp chú ý đa đỉnh}(multi-head attention) để ổn định quá trình học bằng cách ghép $A$ đỉnh chú ý với nhau :

\begin{align}
{\overrightarrow{x'_i}}&={\bigparallel_{a=1}^{A}\sigma\left(\sum_{j\in \mathcal{N}(i)}\alpha_{ij}^{a} \mathbf{W}^{a} \overrightarrow{x_{j}} \right)}
\end{align}

trong đó phép $||$ biểu diễn quá trình ghép chồng lên nhau và $\sigma$ là bất kỳ hàm biến đổi phi tuyến tính nào, $\alpha_{ij}^a$ là hệ số chú ý được chuẩn hóa của cạnh $(e_i, e_j)$ được tính từ lớp thứ $a^{th}$ cơ chế chú ý. Cuối cùng $\overrightarrow{x'_i}$ được coi là vector thực thể nhúng mới và cho vào lớp chú ý đa đỉnh với đầu ra thay vì ghép chồng như trên thì được tính trung bình như công thức sau :

\begin{align}
{\overrightarrow{x''_i}}&={\sigma\left(\frac{1}{A} \sum_{a=1}^{A}\sum_{j\in \mathcal{N}(i)}\alpha_{ij}^{a} \mathbf{W}^{a} \overrightarrow{x'_{j}} \right)}
\end{align}


 \cite{nathani2019learning}

GAT \cite{velivckovic2017graph}

TransE \cite{bordes2013translating}

Attention \cite{vaswani2017attention}

CAttention \cite{cordonnier2020multi}

ConvKB \cite{nguyen2017novel}


	
	% 4. Kết quả thực nghiệm và phân tích
	\chapter{Kết quả thực nghiệm và phân tích}
\label{results}

\section{Datasets}

sdfsdf



	
	% 5. Kết luận
	\chapter{Kết luận}
\label{conclusions}

EXPERIMENTS RESULTS AND ANALYSIS
\section{AnyBURL}

Dùng lệnh $\textbackslash cite$ để trích dẫn một hoặc nhiều tài liệu tham khảo.
Tài liệu tham khảo có thể là trang web~\cite{Listings,HDLVThS}, bài báo khoa học~\cite{1994-Cavnar}, sách~\cite{1984-TeX-Knuth,2006-DDien,2006-NPTV}, bài tạp chí~\cite{1989-TED} hoặc các nguồn tham khảo khác. 
Lưu ý khi trích dẫn tài liệu tham khảo, cần viết câu sao cho bỏ phần trong cặp ngoặc vuông đi thì câu vẫn đầy đủ ý nghĩa.
Ví dụ, thay vì viết ``Nghiên cứu~\cite{1989-TED} chỉ ra rằng ... '' thì nên viết ``Nghiên cứu của Zhang~\cite{1989-TED} chỉ ra rằng ...''.
Một ví dụ khác, thay vì viết ``... như trong công trình nghiên cứu~\cite{1994-Cavnar}.'' thì nên viết ``... như trong công trình nghiên cứu của Cavnar và Trenkle~\cite{1994-Cavnar}.''

	
	% Công trình của tác giả (nếu không có thì comment 02 dòng dưới)
	\addcontentsline{toc}{chapter}{Danh mục công trình của tác giả}
	\chapter*{Danh mục công trình của tác giả}
\label{Publish}

\begin{enumerate}
\item Tạp chí ABC
\item Tạp chí XYZ
\end{enumerate}
	
	% In tài liệu tham khảo
	\addcontentsline{toc}{chapter}{Tài liệu tham khảo}
	\printbibheading[title={Tài liệu tham khảo}]
	
	\DeclareNameAlias{sortname}{last-first}
	\DeclareNameAlias{default}{last-first}
	
	\printbibliography[heading=subbibliography, title={Đồ thị tri thức}, keyword=KG, resetnumbers=true]
	
	\printbibliography[heading=subbibliography, title={Phương pháp AnyBURL}, keyword=AnyBURL, resetnumbers=6]
	
	\printbibliography[heading=subbibliography, title={Phương pháp CGAT}, keyword=CGAT, resetnumbers=15]
	
	
	% ===================================================================== %
	% CHÚ Ý: phải gán lại resetnumbers=số tài liệu tham khảo tiếng Việt + 1 %
	% ===================================================================== %
	% Phần phụ lục
	\appendix 

\chapter{Các siêu tham số tối ưu}
\label{Appendix1}

Trong phần này chúng tôi sẽ báo cáo lại tập hợp các siêu tham số tối ưu trên cả mô hình Chú ý và mô hình ConvKB.
Chúng tôi sử dụng tìm kiếm lưới theo Hits@10 để tìm các tham số tối ưu. Chúng tôi không chia nhỏ tập dữ liệu trong quá trình huấn luyện mô hình chú ý,
đối với mô hình dự đoán, chúng tôi sử dụng kích thước cho tất cả các tập dữ liệu. Cụ thể các siêu tham số chúng tôi sử dụng được trình bày trong bảng sau :

\begin{table}[htbp]
	\begin{center}
		\resizebox{\textwidth}{!}{%
	\begin{tabular}{llllllllll}
		\hline
		& $\mu$  & Weight decay & Epochs & negative ratio  & Dropouts  & $\alpha_{\text{LeakyRELU}}$ & $N_{\text{head}}$ & $D_{\text{final}}$ & $\gamma$ \\
		\hline
		FB15k     & 1e-3 & 1e-5 & 3000 & 2 & 0.3 & 0.2 & 2 & 200 & 1 \\
		FB15k-237 & 1e-3 & 1e-5 & 3000 & 2 & 0.3 & 0.2 & 2 & 200 & 1 \\
		WN18      & 1e-3 & 5e-6 & 3600 & 2 & 0.3 & 0.2 & 2 & 200 & 5 \\
		WN18RR    & 1e-3 & 5e-6 & 3600 & 2 & 0.3 & 0.2 & 2 & 200 & 5 \\
		\hline
	\end{tabular}
}
\end{center}
\end{table}
	\chapter{Bảng thuật ngữ}
\label{Appendix2}

\printglossary[type=\acronymtype]
	
\end{document} 