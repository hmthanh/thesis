\Chapter{Tổng quan}
\label{overview}

sdfsdf
% \begin{document}
% \section{Cơ sở lý thuyết}
% % chapter 2, main file
\section{This is chapter two}

sdfsdfsdfsdf


% Báo cáo phải được trình bày ngắn gọn, rõ ràng, mạch lạc, sạch sẽ, không được tẩy xóa, có đánh số trang, đánh số bảng biểu, hình vẽ, đồ thị. 

% Nội dung báo cáo được phân thành các chương. Số thứ tự của các chương, mục được đánh số bằng hệ thống số Ả-rập, không dùng số La mã. Các mục và tiểu mục được đánh số bằng các nhóm hai hoặc ba chữ số, cách nhau một dấu chấm: số thứ nhất chỉ số chương, chỉ số thứ hai chỉ số mục, số thứ ba chỉ số tiểu mục.


% %Báo cáo cần dùng LaTEX để viết và trình bày theo mẫu đã được cung cấp.

%  Báo cáo trình bày sử dụng khổ giấy với việc canh lề như sau: Lề trên 3 cm, lề dưới 2,5 cm, lề trái 3 cm, lề phải 2 cm. Đánh số trang ở giữa bên dưới. Đánh số trang ở giữa bên dưới.

% Font chữ dùng trong báo cáo (Times New Roman) với kích cỡ (size) 13-14pt, sử dụng chế độ dãn dòng (line spacing) chế độ 1.5 lines.

% %Các bảng biểu trình bày theo chiều ngang khổ giấy thì đầu bảng là lề trái của trang. 


% \section{Các nghiên cứu liên quan}
% \section{Các nghiên cứu liên quan}

Sau thời kỳ ngủ đông của AI, các kỹ thuật học sâu có một bước tiến đang kể với các mô hình CNN \cite{lecun1999object}, RNN \cite{hopfield2007hopfield}, Attention \cite{vaswani2017attention}, Transformer \cite{yang2019xlnet} đã đạt được những kết quả rất lớn với độ chính xác cao cả trong nghiên cứu cũng như khi áp dụng vào công nghiệp. Các mô hình học sâu, cụ thể là CNN đã rất thành công để giải quyết các vấn đề về phân loại hình ảnh \cite{he2016deep} , phân đoạn ngữ nghĩa \cite{jegou2017one} và dịch máy , chúng tạo ra một bản đồ đặc trưng biểu diễn cấu trúc thông tin dưới dạng lưới. CNN cũng đã được áp dụng vào trong cấu trúc đồ thị với mô hình GCN \cite{kipf2016semi}. 

% Nội dung của báo cáo tối thiểu 50 trang khổ A4 và không nên vượt quá 100 trang (không kể các trang bìa, lời cám ơn, mục lục, tài liệu tham khảo \ldots) theo trình tự như sau:
% \end{document}